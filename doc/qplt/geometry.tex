% -*- mode: LaTeX; coding: utf-8 -*-
\documentclass[12pt]{article}
\usepackage[unicode,colorlinks]{hyperref}
\usepackage[T2A]{fontenc}
\usepackage[utf8]{inputenc}
\usepackage[russian]{babel}
\usepackage{amsmath}
\usepackage{amssymb}
\usepackage{eufrak}
\usepackage{epsfig}
%\usepackage[mathscr]{eucal}
\usepackage{psfrag}
\usepackage{tabularx}
\usepackage{wrapfig}
%\usepackage{eucal}
\usepackage{euscript}
\usepackage{cite}

\usepackage[usenames]{color}
\usepackage{colortbl} 

\definecolor{codegreen}{rgb}{0,0.6,0}
\definecolor{codegray}{rgb}{0.5,0.5,0.5}
\definecolor{codeblack}{rgb}{0.1,0.,0.3}
\definecolor{codeemph}{rgb}{0.5,0.1,0.5}
\definecolor{codepurple}{rgb}{0.58,0,0.82}
\definecolor{backcolour}{rgb}{0.95,0.95,0.92}

\usepackage{listings}\lstset{
	basicstyle=\ttfamily\fontsize{10pt}{10pt}\selectfont\color{codeblack},
  commentstyle=\color{codegray},
	keywordstyle=\tt\bf\color{codeemph},
	belowskip=0pt
    }

\setlength{\topmargin}{-0.5in}
\setlength{\oddsidemargin}{-5.mm}
\setlength{\evensidemargin}{-5.mm}
\setlength{\textwidth}{7.in}
\setlength{\textheight}{9.in}

\def\dfdx#1#2{\frac{\partial #1}{\partial #2}}
\def\hm#1{#1\nobreak\discretionary{}{\hbox{\m@th$#1$}}{}}
\newcommand{\Frac}[2]{\displaystyle\frac{#1}{#2}}

\def\sr#1{{\left<#1\right>}}
\def\r{\mathbf r{}}
\def\R{\mathbf R{}}
\def\s{\mathbf s{}}
\def\S{\mathbf S{}}

\def\m{\mathbf m{}}
\def\n{\mathbf n{}}
\def\a{\mathbf a{}}
\def\b{\mathbf b{}}
\def\h{\mathbf h{}}
\def\d{\mathbf d{}}
\def\p{\mathbf p{}}
\def\f{\mathbf f{}}
\def\F{\mathbf F{}}
\def\q{\mathbf q{}}

\begin{document}
\begin{center}
\Large\bf Геометрия qplt
\end{center}

Введем следующие системы координат:
\begin{enumerate}
\item 3D система связанная с фрагментом отображаемой равномерной прямоугольной сетки, $\r\in[0,\R]$,
  где $\R$~--- размер фрагмента сетки;
\item 2D система связанная с плоскостью сцены, $\s\in[0,\S]$, где $\S$~--- число пикселей изображения;  
\end{enumerate}

Пусть $\varphi\in[0,2\pi]$, $\theta\in[0,\pi]$ --- углы задающие ориентацию плоскости сцены (при $\varphi=0$ и $\theta=\pi/2$ взгляд направлен вдоль вектора
$(0,1,0)$ в сторону увеличения компоненты $r_y$), тогда
базис $\n_{XYZ}$ в системе координат $\r$
\begin{itemize}
\item $\n_X = (\cos\varphi,\, \sin\varphi,\, 0)$ --- горизонтальный вектор в плоскости сцены,
\item $\n_Y = (-\cos\theta\sin\varphi,\, \cos\theta\cos\varphi,\, \sin\theta)$ --- вертикальный вектор в плоскости сцены,
\item $\n_Z = (-\sin\theta\sin\varphi,\, \sin\theta\cos\varphi,\, -\cos\theta)$ --- вектор направленный из плоскости сцены в начало координат.
\end{itemize}
Переход $\r\to\s$
\begin{equation}
  \s = \frac\S2 + \frac1{1 + p\r'\cdot\n_Z}\Big(\r'\cdot\n_X ,\, \r'\cdot\n_Y \Big), %=
%\frac \S2 - \left(\frac{\R\cdot\n_X}2,\, \frac{\R\cdot\n_Y}2 \right) + (\r\n_x,\,\r\n_Y) -  
\end{equation}
где $\r' = \r - \R/2$,  $p \geq 0$~--- параметр задающий перспективу.

Для обратного перехода $\s\to\r$ надо задать срез, то есть зафиксировать одну из компонент~$\r$, например $r_k$, что позволяет найти компоненты $r_{i,j}$: 
\begin{equation}
(1 + p\r'\cdot\n_Z)\s'= \Big(\r'\cdot\n_X ,\, \r'\cdot\n_Y \Big),
\end{equation}
или
\begin{align}
s_X + p  n_{iZ} s_X r_i' + p  n_{jZ} s_X r_j' + p  n_{kZ} s_X r_k' &= r'_i n_{iX} + r'_j n_{jX} + r'_k n_{kX}, \\
s_Y + p  n_{iZ} s_Y r_i' + p  n_{jZ} s_Y r_j' + p  n_{kZ} s_Y r_k' &= r'_i n_{iY} + r'_j n_{jY} + r'_k n_{kY}, 
\end{align}
\begin{align}
 (p  n_{iZ} s_X - n_{iX})r_i' + (p  n_{jZ} s_X - n_{jX})r_j'  &= r'_k n_{kX} - s_X -  p  n_{kZ} s_X r_k', \\
 (p  n_{iZ} s_Y - n_{iY})r_i' + (p  n_{jZ} s_Y - n_{jY})r_j'  &= r'_k n_{kY} - s_Y - p  n_{kZ} s_Y r_k', 
\end{align}
\begin{multline}
  \left(\begin{matrix} r_i' \\ r_j' \end{matrix}\right) =
  \left(\begin{matrix}
      p  n_{iZ} s_X - n_{iX} &  p  n_{jZ} s_X - n_{jX} \\
      p  n_{iZ} s_Y - n_{iY} & p  n_{jZ} s_Y - n_{jY} \end{matrix}\right)^{-1}
  \left(\begin{matrix}
      r'_k n_{kX} - s_X -  p  n_{kZ} s_X r_k' \\
      r'_k n_{kY} - s_Y - p  n_{kZ} s_Y r_k'  \end{matrix}\right)
  =\\=
  \left[\begin{matrix}
      p  n_{iZ} s_X - n_{iX} &  p  n_{jZ} s_X - n_{jX} \\
      p  n_{iZ} s_Y - n_{iY} & p  n_{jZ} s_Y - n_{jY} \end{matrix}\right]^{-1}
  \left(\begin{matrix}
      r'_k n_{kX} - s_X (1+  p  n_{kZ} r_k') \\
      r'_k n_{kY} - s_Y (1+ p  n_{kZ}  r_k')  \end{matrix}\right).
\end{multline}


\end{document}
