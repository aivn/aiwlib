% -*- mode: LaTeX; coding: utf-8 -*-
\documentclass[12pt]{article}
\usepackage[unicode,colorlinks]{hyperref}
\usepackage[T2A]{fontenc}
\usepackage[utf8]{inputenc}
\usepackage[russian]{babel}
\usepackage{amsmath}
\usepackage{amssymb}
\usepackage{eufrak}
\usepackage{epsfig}
%\usepackage[mathscr]{eucal}
\usepackage{psfrag}
\usepackage{tabularx}
\usepackage{wrapfig}
%\usepackage{eucal}
\usepackage{euscript}
\usepackage[usenames]{color}
\usepackage{colortbl} 

\definecolor{codegreen}{rgb}{0,0.6,0}
\definecolor{codegray}{rgb}{0.5,0.5,0.5}
\definecolor{codeblack}{rgb}{0.1,0.,0.3}
\definecolor{codeemph}{rgb}{0.5,0.1,0.5}
\definecolor{codepurple}{rgb}{0.58,0,0.82}
\definecolor{backcolour}{rgb}{0.95,0.95,0.92}

\usepackage{listings}\lstset{
	basicstyle=\ttfamily\fontsize{10pt}{10pt}\selectfont\color{codeblack},
    commentstyle=\color{codegray},
	keywordstyle=\tt\bf\color{codeemph},
	belowskip=0pt
  }

\setlength{\topmargin}{-0.5in}
\setlength{\oddsidemargin}{-5.mm}
\setlength{\evensidemargin}{-5.mm}
\setlength{\textwidth}{7.in}
\setlength{\textheight}{9.in}

\def\dfdx#1#2{\frac{\partial #1}{\partial #2}}
\def\hm#1{#1\nobreak\discretionary{}{\hbox{\m@th$#1$}}{}}
\newcommand{\Frac}[2]{\displaystyle\frac{#1}{#2}}

\def\sr#1{{\left<#1\right>}}
\def\m{\mathbf m{}}

\begin{document}
\begin{center}
  {\bf Про ZAMR}

  Антон Иванов
\end{center}

\tableofcontents

%%%%%%%%%%%%%%%%%%%%%%%%%%%%%%%%%%%%%%%%%%%%%%%%%%%%%%%%%%%%%%%%%%%%%%%%%%%%%%%%
\section{Создание и настройка экзмепляра класса ZAMR}
 В настоящий момент ZAMR не поддерживается в Python, работа с ZAMR возможна только из C++.
 Простейший пример кода:
 % \begin{lstlisting}[language=c]
\begin{verbatim}
#include <aiwlib/zamr>
using namespace aiw;

int main(){
    ZipAdaptiveMesh<float, 2> zamr1(
        ind(1,2),  // размеры леса
        3          // начальный ранг разбиения каждого дерева
    );
}
\end{verbatim}
%\end{lstlisting}
 здесь создается 2D zamr1 из float с размерами леса $1\times2$  и начальным рангом разбиения каждого дерева 3,
 всего сетка будет содержать $8\times16$ ячеек (два дерева по $8^2$ ячеек).

 Как и в {\sf aiw::Mesh} можно задавать размеры сетки в некоторой глобальной системе координат при помощи дополнительных аргументов конструктора 
\begin{verbatim}
ZipAdaptiveMesh<float, 2> zamr2(
    ind(1,2),    // размеры леса
    3,           // начальный ранг разбиения каждого дерева
    vec(4., 5.), // координаты левого нижнего угла сетки
    vec(6., 7.)  // координаты правого верхнего угла сетки
);
\end{verbatim}
 Если координаты углов сетки не заданы, по умолчанию полагается что левый нижний угол расположен в точке (0, 0, ...) а правый верхний в точке $(N_x, N_y ...)$
 где $N_i$~--- размеры сетки в ячейках.

 
Альтернативным вариантом является создание леса с деревьями разного ранга разбиения:
\begin{verbatim}
Mesh<int, 2> forest(ind(1,2));
forest[ind(0,0)] = 2;
forest[ind(0,1)] = 3;
ZipAdaptiveMesh<float, 2> zamr3(forest);
\end{verbatim}
Здесь создается объект zamr3 из двух деревьев разного ранга~--- $4^2$ и $8^2$ ячеек.
Координаты углов сетки zamr берутся из координат углов сетки forest.

Кроме настройки объекта zamr при помощи конструктора, можно настроить объект после созания в любой момент при помощи аналогичных методов init:
\begin{verbatim}
zamr.init(ind(1,2), 3);
zamr.init(ind(1,2), 3, vec(4., 5.), vec(6., 7.));
zamr.init(forest);
\end{verbatim}

%%%%%%%%%%%%%%%%%%%%%%%%%%%%%%%%%%%%%%%%%%%%%%%%%%%%%%%%%%%%%%%%%%%%%%%%%%%%%%%%
\section{Работа с ячейкой сетки}
Ячейка сетки \verb'ZipAdaptiveMesh::Cell' является ключевым понятием, интерфейсом обеспечивающим реализацию основных операций.  
С точки зрения программирования ячейку можно рассматривать как итератор, но у пользователя нет возможности переключить ячейку на следующую позицию самостоятельно.

Ячейка сетки \verb'C' имеет следующие методы.
\begin{itemize}
\item \verb'bool(C)' и \verb'!C' --- проверяют валидность ячейки. Например ячейка оказавшаяся за границей сетки невалидна, попытка вызова любых других ее методов
  приводит к ошибке сегментирования.
\item \verb'C.rank()' --- возвращает целочисленный ранг разбиения.
\item \verb'C.step()' --- возвращает размер ячейки в виде \verb'aiw::Vec<D>'.
\item \verb'C.bmin()' --- возвращает координаты левого нижнего угла ячейки в виде \verb'aiw::Vec<D>'.
\item \verb'*C' и \verb'C->...' --- обеспечивает доступ к данным ячейки, например для
\begin{verbatim}
struct MyCell{ 
    float u, tmp_u, v, tmp_v; 
    ...
};
ZipAdaptiveMesh<MyCell, ...> zamr;
\end{verbatim}
  вызов \verb'*C' вернет \verb'MyCell&', вызов \verb'C->u' предоставит доступ к полю \verb'u'.
\item \verb'C[ind(di, dj, dk)]' и \verb'C(di, dj, dk)' --- обеспечивает доступ к ячейке сетки, смещенной относительно \verb'C' на \verb'(di, dj, dk)'.
  Если соседняя ячейка с таким же рангом отсутствует, возвращенная ячейка будет невалидной\footnote{Это не устоялось и может быть изменено по запросам пользователей}.
\item \verb'C.is_boud(int faceID)' --- проверяет является ли ячейка граничной через грань\verb'faceID',  \verb'faceID'
  отвечает номеру оси координат ортогональной грани начиная с единицы со знаком,
  знак задает направление смещения (влево/вправо). Например 1 отвечает проверке справа по оси $x$, -2 отвечает проверке по оси $y$ снизу.
  \item \verb'C.face(...)' --- обходит соседей ячейки по граням, см. раздел~\ref{C:face:sec}.
\end{itemize}

%%%%%%%%%%%%%%%%%%%%%%%%%%%%%%%%%%%%%%%%%%%%%%%%%%%%%%%%%%%%%%%%%%%%%%%%%%%%%%%% 
\section{Обход сетки}
Для обхода сетки используются методы
\begin{verbatim}
zamr.foreach(F &&f, int min_rank=0, int ti=0);
zamr.foreach2Xdt(F1 &&f1, F2 &&f2);
\end{verbatim}
Метод \verb'foreach' однократно применяет \verb'f' к каждой ячейке с рангом разбиения не меньше \verb'max_rank'.
Метод \verb'foreach2Xdt' обходит все ячейки с учетом измельчения шага по времени и применяет к каждой ячейке две стадии численной схемы.
Аргументы \verb'f[12]'
должны быть вызываемыми объектами (функциями, $\lambda$--функциями или функторами) принимающими\linebreak \verb'ZipAdaptiveMesh::Cell&' и номер временного шага.
Удобнее всего работать с $\lambda$--функциями:
\begin{verbatim}
ZipAdaptiveMesh<int, 2> zamr(...);
...
int i = 0;
zamr.foreach([&](decltype(zamr)::Cell &C, int){ *C = i++; });
\end{verbatim}
проинициализирует все ячейки их номерами в порядке обхода сетки. Здесь
\begin{verbatim}
[&](decltype(zamr)::Cell &C, int){ *C = i++; }
\end{verbatim}
является $\lambda$--функцией, выражение \verb'decltype(zamr)::Cell' эквивалентно выражению\\
\verb'ZipAdaptiveMesh<int, 2>::Cell' но имеет важное преимущество~--- при изменении параметров шаблона \verb'zamr'
изменения придется вносить только в одном месте кода.


%%%%%%%%%%%%%%%%%%%%%%%%%%%%%%%%%%%%%%%%%%%%%%%%%%%%%%%%%%%%%%%%%%%%%%%%%%%%%%%%
\section{Обход соседей ячейки сетки}\label{C:face:sec}
Для обхода соседей ячейки сетки предназначен метод ячейки
\begin{verbatim}
void face(F&& f, int faceID, int ifdR=0);
\end{verbatim}
Здесь \verb'f' это $\lambda$--функция пользователя принимающая по ссылке соседнюю ячейку сетки,
\verb'faceID' задает грань по которой производится обход аналогично методу \verb'Cell::is_bound'
(если \verb'axe==0' то производится обход по всем граням), \verb'ifdR' задает допустимую
разницу рангов с соседними ячейками (0~--- любая, 1~--- соседи того же уровня или крупнее, 2~--- только крупнее).

Например вот так можно посчитать интеграл по граням всех ячеек для значения \verb'u':
\begin{verbatim}
struct MyCell{ 
    float u, u_sum; 
    ...
};
...
ZipAdaptiveMesh<MyCell, ...> zamr;
...
zamr.foreach([](decltype(zamr)::Cell &C, int){
    C->u_sum = 0;
    C.face([&](decltype(zamr)::Cell &nbC){ С->u_sum += nbC->u*nbC.face_area(); });
});
\end{verbatim}

%%%%%%%%%%%%%%%%%%%%%%%%%%%%%%%%%%%%%%%%%%%%%%%%%%%%%%%%%%%%%%%%%%%%%%%%%%%%%%%%
\section{Сохранение сетки на диск}
Для сохранения сетки на диск в текстовом формате используйте метод
\begin{verbatim}
zamr.out2dat(File(path2file, "w"));
\end{verbatim}
в этом случае в файл сохраняются координаты центров ячеек, ранги разбиения ячеек и данные ячеек, по одной строке на каждую ячейку.

Метод
\begin{verbatim}
zamr.out2dat(File(path2file, "w"), true);
\end{verbatim}
сохраняет координаты ячеек в виде 2D квадратиков. 

Для сохранения ZAMR содержащих ячейки пользовательского типа
\begin{verbatim}
struct MyCell{ 
    float u, tmp_u, v, tmp_v; 
    ...
};
ZipAdaptiveMesh<MyCell, ...> zamr;
\end{verbatim}
необходимо определить оператор вывода этого типа в поток:
\begin{verbatim}
std::ostream& operator << (std::ostream &S, const MyCell &C){ return S<<C.u<<' '<<C.v; }
\end{verbatim}

Для сохранения/восстановления данных в бинарном формате используются методы
\begin{verbatim}
zamr.dump(File(path2file, "w"));
zamr.load(File(path2file, "r"));
\end{verbatim}
Если в один фалй пишется несколько кадров:
\begin{verbatim}
File fout(path2file, "w");
for(int ti=0; ti<ti_max; ti++){
    ...
    zamr.dump(fout);
}
...
File fin(path2file, "r");
while(zamr.dump(fin)){ ... }
\end{verbatim}
Запись в бинарном формате гораздо быстрее и занимает меньше места, но в настоящий момент бинарный формат еще не устоялся. 
К концу года формат будет зафиксирован и  будет реализован вьювер для бинарного формата.


%%%%%%%%%%%%%%%%%%%%%%%%%%%%%%%%%%%%%%%%%%%%%%%%%%%%%%%%%%%%%%%%%%%%%%%%%%%%%%%%
\section{Перестроение сетки}
{\it Пока не отлажено.}

%%%%%%%%%%%%%%%%%%%%%%%%%%%%%%%%%%%%%%%%%%%%%%%%%%%%%%%%%%%%%%%%%%%%%%%%%%%%%%%%
\section{Сборка кода}
В случае использования связки C++ c Python сборка проводится традиционным образом. Пусть есть следующие файлы.

Makefile:
\begin{verbatim}
name=mycode
headers=model.hpp
modules=model.cpp

include aiwlib/user.mk
\end{verbatim}

model.hpp:
\begin{verbatim}
#pragma once
#include <aiwlib/zamr>

class Model{
...
   ZipAdaptiveMesh<...> zamr;
   void calc(); 
...
};
\end{verbatim}

model.cpp:
\begin{verbatim}
#include "model.hpp"

void Model::calc(){
    zamr.foreach(...);
}
\end{verbatim}
тогда достаточно выполнить команду 
\begin{verbatim}
$ make
\end{verbatim}
для сборки обычной версии или
\begin{verbatim}
$ make debug=on
\end{verbatim}
для сборки версии с отладкой.

Если Python не задействован то для сборки можно использовать команду 
\begin{verbatim}
$ g++ -std=c++11 -Wall -O3 -g -o mycode mycode.cpp ... -laiw
\end{verbatim}
Для сбокри отладочной версии необходимо добавить опцию \verb'-DEBUG', при этом включается много всяких проверок. Отладочная версия может быть существенно медленнее боевой.
Если что то ломается и при отладке в \verb'gdb' непонятно что происходит, рекомендуется убрать опцию \verb'-O3'~--- при оптимизации порядок выполнения
некоторых фрагментов кода может изменяться.

Для отладочной опции дополнительно рекомендуются опции
\begin{verbatim}
-D_GLIBCXX_DEBUG -D_GLIBCXX_DEBUG_PEDANTIC -D_FORTIFY_SOURCE=2 
-fsanitize=address -fsanitize=undefined -fno-sanitize-recover -fstack-protector
\end{verbatim}
 
 

\end{document}
