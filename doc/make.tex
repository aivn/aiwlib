\section{Сборка и установка библиотеки}
\subsection{Получение исходных кодов}
Для получения исходных кодов библиотеки \verb'aiwlib' удобнее всего
использовать утилиту \verb'git':
\begin{verbatim}
    git clone https://github.com/aivn/aiwlib
\end{verbatim}
при этом будет создана директория \verb'aiwlib' содержащая последнюю
версию исходных кодов. Если Вы хотите извлечь исходные коды в директорию с
другим именем \verb'my-aiwlib-specific-path' используйте команду
\begin{verbatim}
    git clone https://github.com/aivn/aiwlib my-aiwlib-specific-path
\end{verbatim}

В дальнейшем, для получения обновлений,
достаточно будет перейти в эту директорию и набрать
\begin{verbatim}
    git pull
\end{verbatim}

Если по каким то причинам использование \verb'git' невозможно,
перейдите по ссылке
\begin{verbatim}
    https://github.com/aivn/aiwlib
\end{verbatim}
нажмите на открывшей странице расположенную справа в центре зеленую
кнопку <<Clone or download>> и выберите в открывшемся окне пункт
<<Download ZIP>>. Однако при этом получение обновленных версий
библиотеки будет довольно неудобным~--- Вам придется каждый раз
получать новый архив и фактически собирать и устанавливать библиотеку заново. 

\subsection{Сборка библиотеки}
Для сборки библиотеки используется утилита \verb'GNU Make'.
Желательно использование компилятора \verb'g++' с версией не старше 4.8. 
Все пакеты и утилиты на языке \verb'Python' рассчитаны на
использование версии \verb'Python2.7' (хотя некоторые из них могут
работать и на более старших версиях), язык \verb'Python3' {\bf не} поддерживается. 

Процесс сборки организован нетрадиционным образом~---
отсутствует скрипт \verb'./configure'. Для настройки сборки необходимо
вручную отредактировать файл
\verb'include/aiwlib/config.mk',\footnote{
Предполагается, что пользователь занимающийся численным моделированием
обладает достаточной квалификацией для этого действия}
либо задать необходимые параметры сборки через аргументы командной
строки \verb'make'~--- во втором случае это придется делать при
каждой новой сборке.

Если Вы получили исходные коды \verb'aiwlib' через \verb'git', после
внесения изменений в файл \verb'include/aiwlib/config.mk'
рекомендуется создать новую ветку и переключится на нее, для этого выполните
команды
\begin{verbatim}
  git commit include/aiwlib/config.mk
  git branch my-custom-config
  git checkout my-custom-config
\end{verbatim}
это позволит не потерять настройки конфигурации после получения обновлений.

Рассмотрим подробно часть файла \verb'include/aiwlib/config.mk'
предназначенную для редактирования (приводятся значения по
умолчанию).

Фрагмент
\begin{verbatim}
PYTHONDIR=/usr/lib/python2.7
LIBDIR=/usr/lib
INCLUDEDIR=/usr/include
BINDIR=/usr/bin
BIN_LIST=racs approx isolines gplt uplt splt mplt fplt
\end{verbatim}
задает целевые пути и список утилит \verb'aiwlib' и играет роль только
при установке уже собранной библиотеки.

Фрагмент 
\begin{verbatim}
zlib=on
swig=on
png=on
pil=on
bin=on
ezz=on
# mpi=on
# mpi=off
\end{verbatim}
задает флаги определяющие список модулей (частей библиотеки) которые будут собираться
и использоваться в проектах. Если в системе отсутствуют необходимые для
сборки модуля зависимости\footnote{Например пакет {\tt zlib1g-dev}
  необходимый для сборки кода с поддержкой сжатых файлов} или
функциональность предоставляемая модулем избыточна, модуль может быть
исключен из сборки по умолчанию при комментировании
(раскомментировании для \verb'mpi') соответствующей
строки фрагмента, либо при помощи аргумента командной строки при
вызове \verb'make'.


\newpage
Фрагмент 
\begin{verbatim}
CXX:=g++
MPICXX:=mpiCC
SWIG:=swig

PYTHON_H_PATH:=/usr/include/python2.7
override CXXOPT:=$(CXXOPT) -std=c++11 -Wall -fopenmp -O3 -fPIC -g 
override MPICXXOPT:=$(MPICXXOPT) $(CXXOPT)
override LINKOPT:=$(LINKOPT) -lgomp 
override SWIGOPT:=$(SWIGOPT) -Wall -python -c++ 
\end{verbatim}
задает компиляторы, опции компиляции и линковки, и путь к файлу \verb'Python.h'
(актуален в случае \verb'swig=on'). Если Вы не знаете где находится
\verb'Python.h'
Вы можете воспользоваться командой
\begin{verbatim}
    locate Python.h
\end{verbatim}
Опции компиляции и линковки могут быть дополнены через аргументы
командной строки \verb'make'.

Содержимое библиотеки с точки зрения особенностей сборки можно условно разбить на следующие части:
\begin{itemize}
  \item ядро~--- модули на языке \verb'C++' в результате сборки дающие файл
    \verb'libaiw.a' и компонуемые с этим файлом утилиты на языке \verb'C++';
  \item модули для работы со структурами данных \verb'C++' из
    \verb'Python' требующие сборки и зависящие от утилиты \verb'swig' и пакета \verb'python-dev';
  \item модули для визуализации на языках \verb'C++', \verb'OpenGL' и
    \verb'Python' требующие сборки и зависящие от довольно
    значительного числа пакетов;
  \item модули и утилиты на языке \verb'Python' не требующие сборки
    (но возможно требующие установки).
\end{itemize}

\begin{table}
\begin{center}
\begin{tabular}{lll}
  \hline
  флаг & зависимости & функциональность \\
  \hline
  \verb'zlib=on' & \verb'zlib...-dev' & работа со сжатыми
  файлами \\[3mm]
  \verb'png=on' & \verb'libpng...-dev' & построение изображений в
  формате \verb'.png' \\[3mm]
  \verb'pil=on' & \verb'python-pil' & построение изображений для {\tt Python Imaging Library}, \\  
                & \verb'python-dev' & работает только при \verb'swig=on' \\[3mm]  
  \verb'mpi=off' &  & сборка с поддержкой \verb'MPI' всегда отключена \\[3mm]  
  \verb'mpi=on' & \verb'mpi...-bin' & для сборки всегда используется \verb'$(MPICXX)', как для  \\  
                & \verb'mpi...-dev' & {\tt aiwlib.a} так и для пользовательских проектов \\[3mm]  
  \verb'mpi='   & опционально       & для сборки некоторых частей {\tt aiwlib.a} используется  \\  
                & \verb'mpi...-bin' & \verb'$(MPICXX)' (при условии что такой компилятор есть, \\
                & \verb'mpi...-dev' & иначе поддержка \verb'MPI' отключается и используется \verb'$(CXX)'), \\
                &                   & при необходимости можно указывать \verb'mpi=on' для проектов \\
                &                   & пользователя  \\[3mm]
  \verb'bin=on' &                   & собирать по умолчанию бинарные утилиты \verb'aiwlib' \\
  \hline
\end{tabular}
\end{center}
\caption{Зависимости ядра, функциональность и влияние флагов сборки}\label{make:core:flags:table}
\end{table}

Зависимости ядра \verb'aiwlib', функциональность и влияние флагов сборки показаны  таблице~\ref{make:core:flags:table}.

Модули для работы со структурами данных \verb'C++' из \verb'Python'
требуют наличия утилиты \verb'swig' (версии не ниже 2.0) и пакета
\verb'python-dev'. Для сборки модулей необходимо установить флаг \verb'swig=on'.
%Более подробно сборка таких модулей описано в разделе~\ref{make:swig:sec}.!!!

Библиотека \verb'aiwlib' предоставляет целое семейство вьюверов:
\begin{itemize}
  \item \verb'gplt' --- построение графиков типографского качества на
    основе {\tt gnuplot};
  \item \verb'uplt' ---  построение срезов многомерных и
    сферических сеток;
  \item\verb'splt' --- визуализация поверхностей, заданных треугольными неструктурированными сетками;
  \item\verb'mplt' --- визуализация распределений магнитных моментов
    и векторных полей;
  \item\verb'fplt' --- воксельная визуализация трехмерных равномерных
    сеток.
\end{itemize}

Вьювер \verb'gplt' не требует сборки, но требует установки библиотеки (настройки
путей к другим питоновским модулям) и требует как минимум пакета
\verb'gnuplot'.
Для рисования графиков на экране требуются пакеты \verb'gnuplot-x11'
или \verb'gnuplot-qt'. Для построения графиков типографского качества
дополнительно необходим \LaTeX, в том числе приложения \verb'pdflatex'
и \verb'pdfcrop'.

Вьювер \verb'uplt' требует сборки \verb'aiwlib' с флагами
\verb'swig=on', \verb'pil=on', кроме того необходим пакет
\verb'python-tk'.

Вьюверы \verb'splt', \verb'mplt' и \verb'fplt' собираются при
установке флага \verb'ezz=on' и требуют пакеты
\verb'glm', \verb'glu', \verb'glew', \verb'python2-pillow', \verb'glut', \verb'python2-glut', \verb'python2-opengl', \verb'readline'.


\subsection{Установка библиотеки}
Для установки библиотеки используйте команды (из под \verb'root')
\begin{verbatim}
    make install
\end{verbatim}
для копирования собранных модулей и заголовочных файлов в системные директории, либо
\begin{verbatim}
    make install-links|links-install
\end{verbatim}
для создания мягких ссылок на собранные модули и заголовочные файлы в системных директориях.
Второй вариант предпочтительней с точки зрения простоты обновления библиотеки, 
но может иметь некоторые уязвимости с точки зрения безопасности
при работе в многопользовательском режиме~--- 
гипотетически пользователь установивший у себя библиотеку может подсунуть другим пользователям вредоносный код.

Предыдущая версия, библиотека {\tt aivlib}, имела два существенных недостатка~--- сложную установку и проблемы при
работе с несколькими версиями библиотеки на одной машине. Текущая версия допускает локальную установку
произвольного числа версий/копий библиотеки, более того это рекомендуется для  
упрощения переноса проектов на другие машины, где библиотека \verb'aiwlib' не установлена~---
в этом случае вместо копирования всей библиотоеки достаточно создать мягкую ссылку на библиотеку в директории проекта.

Таким образом Вы можете не устанавливать \verb'aiwlib' глобально в
систему~--- достаточно собрать библиотеку и использовать для своих
проектов систему сборки \verb'aiwlib' описанную ниже. При этом доступ
к утилитам командной строки \verb'aiwlib' и вьюверам (расположенным в директории
\verb'aiwlib/bin') Вам придется обеспечивать самостоятельно.  

\section{Сборка проектов с использованием aiwlib}
Для упрощения сборки проекта рекомендуется использовать файл \verb'include/aiwlib/user.mk'.
При этом пользовательский \verb'Makefile' должен иметь вид
\begin{verbatim}
name=NAME #имя проекта
headers=... #список хидеров обрабатываемых SWIG-ом
modules=... #список .cpp модулей проекта
sources=... #другие исходные файлы проекта

include aiwlib/user.mk
\end{verbatim}
или
\begin{verbatim}
include local-path-to-aiwlib/include/aiwlib/user.mk
\end{verbatim}
При этом заголовочные файлы библиотеки всегда включаются как \verb'<aiwlib/...>',
путь к локально установленной библиотеке при необходимости определяется автоматически на основе пути к файлу \verb'user.mk'.

По умолчанию, такой \verb'Makefile' собирает модуль для питона, и предоставляет еще ряд целей
\begin{itemize}
\item \verb'sources' --- выводит список исходных файлов проекта включая \verb'make'--файл, хидеры определяются автоматически 
при помощи вызова \verb'g++ -M' на основе переменной \verb'modules', заголовочные файлы библиотеки \verb'aiwlib' 
{\bf НЕ} включаются в список. Если при вызове \verb'make' указать опцию \verb'with=swig', в список будут включены файлы 
\verb'NAME.i', \verb'NAME.py' и \verb'NAME_wrap.cxx'
\item \verb'all_sources' --- выводит список всех исходных файлов проекта включая \verb'make'--файл, 
хидеры определяются автоматически 
при помощи вызова \verb'g++ -M' на основе переменной \verb'modules', заголовочные файлы библиотеки \verb'aiwlib' 
включаются в список {\bf ПРИ УСЛОВИИ ЕЕ ЛОКАЛЬНОЙ УСТАНОВКИ}. 
Если библиотека \verb'aiwlib' установлена локально и при вызове \verb'make' указана опция \verb'with=aiwlib', в список включаются  
необходимые файлы \verb'.py' и \verb'.i' библиотеки \verb'aiwlib'.
Если библиотека \verb'aiwlib' установлена локально и при вызове \verb'make' указана опция \verb'with=aiwlib,swig',
(последовательность не имеет значения) в список включаются
все файлы \verb'.py', \verb'.i' и \verb'_wrap.cxx' библиотеки \verb'aiwlib', а так же файлы 
\verb'NAME.i', \verb'NAME.py' и \verb'NAME_wrap.cxx'.
\item \verb'clean' --- удаляет все созданные объектные файлы и файл \verb'_NAME.so'.
\item \verb'cleanall' --- удаляет все созданные объектные и \verb'.so' модули, а так же файлы 
\verb'NAME.i', \verb'NAME.py' и \verb'NAME_wrap.cxx'.
\item \verb'NAME.tgz' \verb'NAME.md5' --- создает сжатый архив и файл с контрольной суммой {\bf несжатого} архива
со списком файлов  проекта \verb'sources', опция \verb'with' влияет на список файлов.
\item \verb'tar NAME-all.tgz' --- создает сжатый архив \verb'NAME-all.tgz'
со списком файлов  проекта \verb'all_sources', опция \verb'with' влияет на список файлов.
\item \verb'export to=...' --- переносит проект в другую директорию (если опция \verb'to' имеет вид \verb'to=PATH')
либо на другую машину (если опция \verb'to' имеет вид \verb'to=HOST:PATH'). Для переноса используется список 
файлов \verb'all_sources', опция \verb'with' влияет на список файлов. Перенос на другую машину осуществляется при помощи 
утилиты \verb'SSH', если у Вас не настроена авторизация по открытому ключу то пароль придется вводить трижды.
Директория \verb'PATH' создается при необходимости, включая родительские каталоги.
Корректный перенос файлов библиотеки \verb'aiwlib' возможен только при условии ее установки в поддиректорию проекта.
\end{itemize}
Таким образом, опция \verb'with=swig' необходима, если на целевой машине отсутствует утилита \verb'SWIG'. Опция \verb'with=aiwlib'
необходима, если библиотека \verb'aiwlib' была установлена локально в поддиректорию проекта.

В питоне, при импорте модулей библиотеки \verb'aiwlib' будут импортироваться модули из локальной копии библиотеки,
при условии что до этого был импортирован собранный пользовательский модуль. 

Если модуль для питона собрать не требуется, не указывайте переменную
\verb'name'. Если требуется собирать некоторое количество исполняемых
файлов из \verb'C++' кода, их можно перечислить в переменной
\verb'cxxmain' через пробел, указываются \verb'C++'--файлы содержащие
\verb'main()'--функции.
Для каждого такого файла будет собран исполняемый файл, при этом
производится линковка со всеми объектными файлами из переменных
\verb'modules' и  \verb'sources'.

