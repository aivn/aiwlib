\section{Сборка и установка библиотеки}
Для сборки библиотеки используется утилита \verb'GNU Make'. Для работы в \verb'C++' сборка не требуется,
достаточно заголовочных файлов. Для работы в \verb'Python' необходимо собрать
ряд модулей с применением утилиты \verb'SWIG', что требует вызова команды 
\begin{verbatim}
    make
\end{verbatim}
из директории библиотеки.

Для установки библиотеки используйте команды (из под \verb'root')
\begin{verbatim}
    make install
\end{verbatim}
для копирования собранных модулей и заголовочных файлов в системные директории, либо
\begin{verbatim}
    make install-links|links-install
\end{verbatim}
для создания мягких ссылок на собранные модули и заголовочные файлы в системных директориях.
Второй вариант предпочтительней с точки зрения простоты обновления библиотеки, 
но может иметь некоторые уязвимости с точки зрения безопасности
при работе в многопользовательском режиме~--- 
гипотетически пользователь установивший у себя библиотеку может подсунуть другим пользователям вредоносный код.

Предыдущая версия, библиотека {\tt aivlib}, имела два существенных недостатка~--- сложную установку и проблемы при
работе с несколькими версиями библиотеки на одной машине. Текущая версия допускает локальную установку
произвольного числа версий/копий библиотеки, более того это рекомендуется для  
упрощения переноса проектов на другие машины, где библиотека \verb'aiwlib' не установлена~---
в этом случае вместо копирования всей библиотоеки достаточно создать мягкую ссылку на библиотеку в директории проекта.

Для упрощения сборки проекта рекомендуется использовать файл \verb'include/aiwlib/user.mk'.
При этом пользовательский \verb'Makefile' должен иметь вид
\begin{verbatim}
name=NAME #имя проекта
headers=... #список хидеров обратаываемых SWIG-ом
modules=... #список .cpp модулей проекта
sources=... #другие исходные файлы проекта

include local-path-to-aiwlib/include/aiwlib/user.mk
\end{verbatim}
При этом заголовочные файлы библиотеки всегда включаются как \verb'<aiwlib/...>',
путь к локально установленной библиотеке при необходимости определяется автоматически на основе пути к файлу \verb'user.mk'.

По умолчанию, такой \verb'Makefile' собирает модуль для питона, и предоставляет еще ряд целей
\begin{itemize}
\item \verb'sources' --- выводит список исходных файлов проекта включая \verb'make'--файл, хидеры определяются автоматически 
при помощи вызова \verb'g++ -M' на основе переменной \verb'modules', заголовочные файлы библиотеки \verb'aiwlib' 
{\bf НЕ} включаются в список. Если при вызове \verb'make' указать опцию \verb'with=swig', в список будут включены файлы 
\verb'NAME.i', \verb'NAME.py' и \verb'NAME_wrap.cxx'
\item \verb'all_sources' --- выводит список всех исходных файлов проекта включая \verb'make'--файл, 
хидеры определяются автоматически 
при помощи вызова \verb'g++ -M' на основе переменной \verb'modules', заголовочные файлы библиотеки \verb'aiwlib' 
включаются в список {\bf ПРИ УСЛОВИИ ЕЕ ЛОКАЛЬНОЙ УСТАНОВКИ}. 
Если библиотека \verb'aiwlib' установлена локально и при вызове \verb'make' указана опция \verb'with=aiwlib', в список включаются  
необходимые файлы \verb'.py' и \verb'.i' библиотеки \verb'aiwlib'.
Если библиотека \verb'aiwlib' установлена локально и при вызове \verb'make' указана опция \verb'with=aiwlib,swig',
(последовательность не имеет значения) в список включаются
все файлы \verb'.py', \verb'.i' и \verb'_wrap.cxx' библиотеки \verb'aiwlib', а так же файлы 
\verb'NAME.i', \verb'NAME.py' и \verb'NAME_wrap.cxx'.
\item \verb'clean' --- удаляет все созданные объектные файлы и файл \verb'_NAME.so'.
\item \verb'cleanall' --- удаляет все созданные объектные и \verb'.so' модули, а так же файлы 
\verb'NAME.i', \verb'NAME.py' и \verb'NAME_wrap.cxx'.
\item \verb'NAME.tgz' \verb'NAME.md5' --- создает сжатый архив и файл с контрольной суммой {\bf несжатого} архива
со списком файлов  проекта \verb'sources', опция \verb'with' влияет на список файлов.
\item \verb'tar NAME-all.tgz' --- создает сжатый архив \verb'NAME-all.tgz'
со списком файлов  проекта \verb'all_sources', опция \verb'with' влияет на список файлов.
\item \verb'export to=...' --- переносит проект в другую директорию (если опция \verb'to' имеет вид \verb'to=PATH')
либо на другую машину (если опция \verb'to' имеет вид \verb'to=HOST:PATH'). Для переноса используется список 
файлов \verb'all_sources', опция \verb'with' влияет на список файлов. Перенос на другую машину осуществляется при помощи 
утилиты \verb'SSH', если у Вас не настроена авторизация по открытому ключу то пароль придется вводить трижды.
Директория \verb'PATH' создается при необходимости, включая родительские каталоги.
Корректный перенос файлов библиотеки \verb'aiwlib' возможен только при условии ее установки в поддиректорию проекта.
\end{itemize}
Таким образом, опция \verb'with=swig' необходима, если на целевой машине отсутствует утилита \verb'SWIG'. Опция \verb'with=aiwlib'
необходима, если библиотека \verb'aiwlib' была установлена локально в поддиректорию проекта.

В питоне, при импорте модулей библиотеки \verb'aiwlib' будут импортироваться модули из локальной копии библиотеки,
при условии что до этого был импортирован собранный пользовательский модуль. 
