\section{Построение линий постоянного уровня для двумерной сетки --- модуль {\tt isolines}}
Для построения изолиний (линий постоянного уровня) на сеточных функциях $f(x,y)$ \verb'C++' модуль
\verb'isolines' предоставляет класс
\begin{verbatim}
    class IsoLines{
    public:
        init(const aiw::Mesh<float, 2> &arr, double z0, double dz, bool logscale=false);
		
        int count() const; // число изолиний
        size_t size(int l) const; // число точек в изолинии l
        float level(int l) const; // значений уровня на изолинии l
        aiw::Vecf<2> point(int l, int i) const; // координаты узла изолинии l

#ifndef SWIG
        template <typename T> void out2dat(T &&S) const;
#endif //SWIG
        void out2dat(aiw::IOstream &S) const;        
        void out2dat(std::ostream &S) const;
        void out2dat(const char *path) const;
	};
\end{verbatim}
Реализация методов класса \verb'IsoLines' находится в файле \verb'src/isolines.cpp'

Метод \verb'init' принимает сетку \verb'arr', опорное значение функции \verb'z0' (которому точно соответствует изолиния),
шаг изолиний \verb'dz' и флаг \verb'logscale' задающий логарифмический масштаб по $z$ (в этом лучаем параметр \verb'dz' трактуется как
отношение значения на двух соседних изолиниях, \verb'dz>0'), и строит изолинии.

При построении считается, что функция \verb'arr' билинейно интерполируется внутри ячеек. Каждая изолиния представляет ломанную линию,
узлы которой попадают на границы ячеек сеточной функции.

Изолинии могут быть прочитаны из экземпляра класса \verb'IsoLines' при помощи методов \verb'count', \verb'size', \verb'level' и \verb'point',
либо выведены в файловый поток вывода (или файла) при помощи методов \verb'out2dat'.
