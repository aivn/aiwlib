\section{Описание пользовательских типов~--- модуль {\tt typeinfo}}
Модуль \verb'typeinfo' предоставляет механизм для описания пользовательских типов,
обеспечивающий некоторые возможности для интроспекции. В частности,
описание пользовательского типа может быть сохранено (например как описание типа ячейки сетки),
и использовано в дальнейшем при анализе данных, построении графиков и т.д.

На уровне конечного пользователя работа с модулем \verb'typeinfo' выглядит как вызов макросов
\begin{verbatim}
    #define TYPEINFO(ARGS...)
    #define TYPEINFOX(T, ARGS...)
\end{verbatim}
перегружающих для пользовательского типа \verb'T' операции
\begin{verbatim}
    aiw::TypeInfo operator ^ (aiw::TypeInfoObj& tio) const;
    aiw::TypeInfo operator ^ (const T& X, aiw::TypeInfoObj& tio);
\end{verbatim}
Макрос \verb'TYPEINFO' перегружает операцию внутри пользовательского типа, макрос \verb'TYPEINFOX' перегружает операцию снаружи. Например
\begin{verbatim}
    struct A{
        int x;
        double y[10];
        bool f;
        TYPEINFO(x, y, f);
    }; 
    struct B{
        A a[2][3];
        Vec<3> v;
    }; 
    TYPEINFOX(B, X.a, X.v);
\end{verbatim}

Предполагается, что использование этого механизма позволит эффективно обрабатывать результаты расчетов, сохраненных в виде сеток из польщовательских типов.
Находится на стадии реализации.

