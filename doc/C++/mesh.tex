% -*- mode: LaTeX; coding: utf-8 -*-
\section{Равномерные многомерные прямоугольные (картезианские) сетки --- модуль {\tt mesh}}\label{mesh:sec}
\subsection{Общие замечания}
Равномерные прямоугольные сетки реализованы в заголовочном файле
{\tt <aiwlib/mesh>}  в виде параметризованного класса {\tt Mesh<$T$,$D$>}, где $T$~---~тип ячейки массива, $D$~---
размерность массива. 

Многомерные сетки поддерживают настройку осей координат~--- для каждой оси могут быть заданы пределы, шаг
и опционально логарифмический масштаб. Поддерживается обращение к ячейкам сетки как по индексу (номеру по всем осям)
так и по координате, которая пересчитывается в индекс на основе настроек осей.
Кроме того, возможно использование интерполяции различных порядков, задание периодических граничных условий, продолжение сетки за область ее определения
на основе граничных значений.

Многомерная сетка (массив) эмулируется при помощи
одномерного массива, смещение в котором пересчитывается с учетом размеров
многомерной области. Предоставляются средства для организации эффективного обхода содержимого сетки  с учетом локальности данных.

Многомерные сетки {\tt aiw::Mesh} обеспечивают упорядоченный доступ к некоторому участку памяти.
Возможно создание сеток с другими размерами, обеспечивающих доступ
к тому же участку. В частности, конструкторы копирования сеток не
выделяют новых участков памяти под данные~---~копии ссылаются на тот же
участок. Сборка мусора призводится на основе подсчета ссылок.  С одной
стороны это существенно ускоряет копирование объектов, с другой стороны копии не являются
независимыми, т.е. изменение данных в одной копии влечет за собой изменение всех
остальных копий. Для полноценного копирования с выделением нового участка
памяти под данные используется метод {\tt copy()}.

Многомерные сетки допускают проведение ряда преобразований~--- разворот и изменение порядка нумерации осей координат, 
вырезание подобластей, построение срезов и т.д. При этом не происходит копирование данных исходной сетки, а лишь предоставляется 
альтернативный способ доступа к исходным данным, что открывает широкие возможности для манипуляций с данными.

Многомерные сетки обеспечивают запись и чтение данных на диск в бинарном формате (сейчас используется старый 
формат библиотеки {\tt aivlib}) и форматированный вывод данных в текстовом виде для {\tt gnuplot}. 

\subsection{Поля и методы для получения информации о сетке}
Класс {\tt Mesh<$T$,$D$>} содежит следующие открытые поля:
\begin{verbatim}
    std::string head;  // произвольный текстовый заголовок
    T out_value;       // значение (ячейка) за пределами сетки
    aiw::Vec<D> bmin;  // координаты левого нижнего угла области
    aiw::Vec<D> bmax;  // координаты правого верхнего угла области
    aiw::Vec<D> step;  // размер ячейки сетки
    aiw::Vec<D> rstep; // обратный размер ячейки
    int logscale;      // битовая маска отмечающая логарифмические масштабы осей
\end{verbatim}

Класс {\tt Mesh<$T$,$D$>} предоставляет следующие методы для получения информации о состоянии сетки
\begin{verbatim}
    size_t size() const;       // общее число элементов сетки
    aiw::Ind<D> bbox() const;  // размеры сетки по всем осям
    size_t mem_size() const;   // размер области памяти в ячейках
    size_t mem_sizeof() const; // размер ячейки в байтах
\end{verbatim}
Методы \verb'mem_size()' и \verb'mem_sizeof()' выдают информацию об области памяти сетки без учета
проведенных преобразований. 

\subsection{Инициализация сетки и настройка осей}
Для настройки осей сетки предназначен метод
\begin{verbatim}
    void set_axes(const aiw::Vec<D> &bmin, const aiw::Vec<D> &bmax, int logscale=0);
\end{verbatim}
Метод настраивает оси на основе текущих размеров сетки в ячейках.

Для инициализации сетки (выделения памяти) служат методы 
\begin{verbatim}
    void init(const aiw::Ind<D> &box);
    void init(const aiw::Ind<D> &box, 
              const aiw::Vec<D> &bmin, const aiw::Vec<D> &bmax, int logscale=0);
\end{verbatim}
Вторая версия метода \verb'init' производит настройку осей после выделения памяти.
Первая версия метода \verb'init' настраивает оси по умолчанию~--- размеры области от нуля до \verb'box' 
(шаг равен еденице), логарифмического масштаба нет.

Каждый вызов метода \verb'init' приводит к выделению новой области памяти под данные, однако старая область памяти
может оказаться используемой другой сеткой и не обязательно будет освобождена, см. раздел~\ref{mesh:change:sec}

Все настройки осей хранятся в открытых полях \verb'bmin', \verb'bmax', \verb'step',
\verb'rstep' и \verb'logscale'. 
Несогласованное изменение этих полей может привести к некорректному преобразованию координат точки
в индекс ячейки сетки.

Кроме того, для настройки граничных условий и интерполяции используются поля
\begin{verbatim}
    int periodic; // битовая маска, задающая периодические граничные условия для осей
    Ind<2> bound_min, bound_max;  // битовые маски, задающие обработку границ слева/справа (если нет периодики)
		                          // 0   - ничего не делать (при выходе за границу выкидывает исключение)
		                          // 1,0 - возвращает out_value
		                          // 1,1 - возвращает крайнее значение
    Ind<3> Itype; // битовые маски, задающие типы интерполяции по осям:
                  // 0 - без интерполяции, 1,0 - линейная, 1,1,0 - кубическая, 1,1,1 - B-сплайнами.
\end{verbatim}


\subsection{Доступ к ячейкам}
Сетки обеспечивают доступ к ячейкам по координате (вектору) или индексу (набору номеров ячейки по всем осям).

Базовыми являются методы
\begin{verbatim}
    inline aiw::Ind<D> coord2pos(const aiw::Vec<D> &r) const;
    inline double pos2coord(int pos, int axe) const;
    inline aiw::Vec<D> pos2coord(const aiw::Ind<D> &p) const;
    inline aiw::Vec<D> cell_angle(const aiw::Ind<D> &p, bool up) const;
\end{verbatim}
пересчитывающие координаты в индексы и обратно согласно настройкам осей. Метод \verb'pos2coord' возвращает координаты 
центра ячейки. Метод \verb'cell_angle' в зависимости от аргумента \verb'up' возвращает координаты левого нижнего или правого верхнего угла ячейки. 

Для доступа к ячейкам служат методы
\begin{verbatim}
    inline const T& get(const aiw::Ind<D> &pos) const;
    inline const T& get(const aiw::Vec<D> &r) const;
\end{verbatim}
%При включенном режиме отладки (опция \verb'debug=on' утилиты \verb'make' либо опция \verb'-DEBUG' компилятора)
%метод доступа по индексу проверяет попадание индекса внутрь сетки, при промахе возбуждается исключение.
%При отключенной отладке проверка не происходит, что может приводить при промахе к доступу в неверные ячейки 
%либо ошибке сегментирования. При доступе по индексу вычисление адреса ячейки требует $D$ целочисленных умножений и сложений.

В зависимости от значений полей \verb'periodic', \verb'bound_min' и \verb'bound_max' при промахе (выходе за границы сетки)
реализуются периодические граничные условия (если поднят соответствующий бит маски \verb'periodic'), выкидывается исключение (если не поднят соответствующий  бит
в маске \verb'bound_min/max[0]'),
обеспечивается доступ к полю \verb'out_value' (если поднят соответствующий  бит
в маске \verb'bound_min/max[0]') либо к граничной ячейке (если поднят соответствующий  бит
в маске \verb'bound_min/max[1]').
В настоящий момент функция \verb'get' обеспечивает довольно гибкое управление поведением сетки при промахах, но при этом проводиться довольно много проверок.

Метод доступа по вектору требует дополнительных вычислений для перевода вектора в индекс ячейки. 
При промахе (если вектор попадает за пределы обоасти сетки) обеспечивается доступ к открытому полю сетки \verb'out_value'.

Для традиционного доступа в \verb'С++' перегружены операции 
\begin{verbatim}
    inline const T& operator [] (const aiw::Ind<D> &p) const;
    inline       T& operator [] (const aiw::Ind<D> &p);
    inline const T& operator [] (const aiw::Vec<D> &r) const;
    inline       T& operator [] (const aiw::Vec<D> &r);
\end{verbatim}
вызывающие функции \verb'get', те же операторы перегружены в \verb'Python' как
\begin{verbatim}
    inline const T& __getitem__(const aiw::Ind<D> &p) const;
    inline const T& __getitem__(const aiw::Vec<D> &r) const;
    inline void __setitem__(const aiw::Ind<D> &p, const T& v);
    inline void __setitem__(const aiw::Vec<D> &r, const T& v);
\end{verbatim}

Для реализации доступа с периодическими граничными условиями предназначены методы\footnote{Пока оставлено для обратной совместимости}
\begin{verbatim}
    template<int P> inline const T& periodic_bc(Ind<D> pos) const;
    template<int P> inline T& periodic_bc(Ind<D> pos);
\end{verbatim}
где $P$ --- битовая маска, указывающая по каким осям необходимо создать периодичность.
Например $P=5$ задаст периодические граничные условия по осям $x$ и $z$.
Методы \verb'periodic_bc' корректируют компоненты индекса по тем осям, для которых указаны периодические граничные условия,
и затем вызывают метод \verb'get'.

Для интерполяции перегружена операция 
\begin{verbatim}
    inline T operator () (const aiw::Vec<D> &r) const;
\end{verbatim}
тип интерполяции задается при помощи битовых масок \verb'Ind<3> Itype', при этом используются функции из модуля \verb'interpolations' (см. \ref{interpolations:sec}).
Соответствующие некоторой оси \verb'axe' биты в \verb'Itype' означают: \verb'Itype[0]&(1<<axe)==0'~--- интерполяция нулевого порядка (кусочно--постоянная
в рамках ячейки), \verb'Itype[0]&(1<<axe)==1', \verb'Itype[1]&(1<<axe)==0'~--- линейная интерполяция между центрами ячеек,
\verb'Itype[0]&(1<<axe)==1', \verb'Itype[1]&(1<<axe)==1', \verb'Itype[2]&(1<<axe)==0'~--- локальный кубический сплайн,
\verb'Itype[0]&(1<<axe)==1', \verb'Itype[1]&(1<<axe)==1', \verb'Itype[2]&(1<<axe)==1'~--- кубический $B$--сплайн.

При интерполяции на границе сетки важны настройки \verb'periodic' и \verb'bound_min/max'. Функции из модуля \verb'interpolations' не пороверяют
выход за границы, поэтому при \verb'bound_min/max[0]&(1<<axe)==0' возможна генерация исключения.

\subsection{Обход сетки}
Для оптимального обхода сетки предназначены методы
\begin{verbatim}
    inline aiw::Ind<D> inbox(size_t offset) const;
    inline aiw::Ind<D> first() const;
    inline bool next(aiw::Ind<D> &pos) const;
\end{verbatim}
Метод \verb'inbox' преобразует номер элемента сетки (от начала области памяти) в его индекс. Для непреобразованной сетки
его результат будет совпадать с результатами операции \verb'offset%bbox()', 
однако для преобразованной сетки это может быть неверно. Метод \verb'inbox' является относительно дорогостоящим,
но обеспечивает оптимальный (с точки зрения локальности данных) порядок обхода сетки и позволяет
легко распараллеливать циклы обходы средствами библиотеки \verb'OpenMP'
\begin{verbatim}
    Mesh<T,D> M;
    ...
    size_t sz = M.size();
#pragma omp parallel for
    for(size_t i=0; i<sz; ++i){
        Ind<D> pos = M.inbox(i);
        ...
    }
\end{verbatim}

Аналогичный порядок обхода (с меньшими накладными расходами, но без такого простого распараллеивания) 
можно получить при помощи конструкции
\begin{verbatim}
    Ind<D> pos=M.first(); 
    do{...} while(M.next(pos));
\end{verbatim}
Для непреобразованных сеток этот обход экивалентен конструкции
\begin{verbatim}
    for(Ind<D> pos; pos^=M.bbox(); ++pos){...} 
\end{verbatim}
но после преобразований такой вариант может оказаться неэффективным.

\subsection{Преобразования сеток}\label{mesh:change:sec}
Для преобразования сеток служат методы
\begin{verbatim}
    Mesh flip(int a, bool axe=true) const;
    Mesh transpose(int a, int b) const;
    Mesh crop(aiw::Ind<D> l, aiw::Ind<D> m, aiw::Ind<D> n=Ind<D>(1)) const;
    template <class T2, int D2> 
        Mesh<T2, D2> slice(Ind<D> pos, size_t offset_in_cell) const;
\end{verbatim}
Все эти методы не приводят к выделению новых областей памяти для данных сетки, а лишь 
создают альтернативные способы доступа к уже выделенной памяти в исходной сетке. 
После создания преобразованной сетки исходная сетка может быть удалена/перенициализирована, однако 
освобождение памяти произойдет лишь после уничтожения/перенициализации всех преобразованных сеток. 
Сборка мусора осуществляется при помощи подсчета ссылок на основе указателя \verb'std::shared_ptr<BaseAlloc>',
исходная сетка и построенные на ее основе преобразованные сетки являются равноправными владельцами выделенной под 
данные памяти.

Метод  \verb'flip(int a, bool axe=true)' разворачивает (отражает) ось $a$, параметр \verb'axe' указывает следует ли 
преобразовать так же настройки оси (пределы и шаг).

Метод \verb'transpose(int a, int b)' меняет оси $a$ и $b$ местами, при этом преобразуются так же настройки осей.

Метод \verb'crop(aiw::Ind<D> l, aiw::Ind<D> m, aiw::Ind<D> n=Ind<D>(1))' вырезает фрагмент сетки
с левым нижним углом в ячейке $\l$, правым верхним углом в ячейке $\m$, правая верхняя граница не включается.
При задании $\l$ и $\m$ допускается использовать отрицательные значения, которые отсчистываются 
от верхней границы (размера сетки) по соответствующей оси. Необязательный параметр $\n$ позволяет задать шаг,
т.е. использовать каждую $\n$--ю ячейку внутри указанной области.

Метод \verb'slice<D2, T2>(Ind<D> pos, size_t offset_in_cell)' позволяет строить срезы~---
уменьшать размерность сетки и изменять тип хранимых данных,
например составляя новую сетку из отдельных полей структуры хранящейся в исходной сетке.

Для уменьшения  размерности необходимо указать в аргументе \verb'pos' значения -1 по тем осям, которые
должны войти в срез (ровно $D_2$ штук), и положение среза по остальным осям.

Для измения типа данных необходимо указать новый тип и сдвиг данных внутри исходной структуры в байтах.
Следует с осторожностью использовать этот вариант вызова метода {\tt slice}, поскольку
неверно вычисленное смещение может привести к непредсказуемым результатам. Допустим есть
стуктура 
\begin{verbatim}
     struct Cell{ double a, b; int c[10];}; 
\end{verbatim}
для которой создана сетка \verb'Mesh<Cell,2> A;' Тогда, вызов 
\begin{verbatim}
     Mesh<int,1> B = A.slice<1, int>(Indx(10,-1), 2*8+2*4);
\end{verbatim}
создаст срез {\tt B} в виде одномерного массива развернутого по оси $y$, проходящего
через десятые ячейки по оси $x$, и обеспечивающий доступ к полям {\tt c[2]}
соответствующих ячеек. 

В настоящий момент метод \verb'slice' доступен из \verb'Python'-а как семейство методов
\begin{verbatim}
    sliceX(self, pos, offset=0)
\end{verbatim}
где \verb'X' пробегает значения от \verb'1' до \verb'D-1'. Таким образом в \verb'Python' возможно построение срезов, но невозможно
изменение типа ячейки\footnote{Поскольку изменение типа ячейки невозможно, изначальный смысл аргумента {\tt offset} в {\tt Python} теряется,
однако возможны экзотические ситуации когда этот аргумент все же будет востребован}. 


\subsection{Сохранение и загрузка сеток}
Для сохранения и загрузки содержимого сеток в бинарном формате предназначены методы
\begin{verbatim}
    void dump(aiw::IOstream &&S) const;
    void load(aiw::IOstream &&S, int use_mmap=0);
    void dump(aiw::IOstream &S) const;
    void load(aiw::IOstream &S, int use_mmap=0);
\end{verbatim}
В настоящий момент используется старый формат библиотеки \verb'aivlib'. Необязательны параметр \verb'use_mmap' 
указывает на использование мапирования файла, 0~--- не использовать мапирование, 1~--- мапировать файл только на чтение,
2~--- мапировать файл на чтение и запись.

Кроме того перегружены операции \verb'<>' для бинарного ввода/вывода
\begin{verbatim}		
    IOstream& operator < (IOstream &S, const Mesh<T, D> &M);
    IOstream& operator > (IOstream &S,       Mesh<T, D> &M);
\end{verbatim}

Для форматированного вывода (\verb'.dat'--файлы для \verb'gnuplot') предназначен метод
\begin{verbatim}		
    template <typename S> 
    void out2dat(S &&str, Ind<D> coords=Ind<D>(2), const char* prefix=nullptr) const;
\end{verbatim}
и его оболочки для конкретных видов потоков для инстацирования в \verb'Python'
\begin{verbatim}		
    void out2dat(std::ostream &str=std::cout, aiw::Ind<D> coords=aiw::Ind<D>(2), 
                 const char* prefix=nullptr) const;
    void out2dat(aiw::IOstream &str, aiw::Ind<D> coords=aiw::Ind<D>(2), 
                 const char* prefix=nullptr) const;
\end{verbatim}
Аргумент \verb'coords' содержит режимы вывода для каждой из координатных осей, возможны следующие режимы:\\ 
\begin{tabular}{rcl}
0 & --- & выводить значения из сетки вдоль оси в одну строку через пробел;\\
1 & --- & выводить номер ячейки;\\
2 & --- & выводить координату центра ячейки;\\
3 & --- & не выводить ни номер ячейки ни координату центра ячейки;\\
+4 & --- & не выводить пустую строку при изменении номера ячейки, не влияет на режим 0.
\end{tabular}\\
Аргумент \verb'prefix' задает префикс перед каждой (не пустой) строкой при выводе.
Для загрузки сеток из \verb'.dat'--файлов используется \verb'C++' модуль \verb'dat2mesh' (см. раздел \ref{dat2mesh:sec}).

Для сериализации сеток при помощи модуля \verb'pickle' в \verb'Python' реализованы методы
\begin{verbatim}		
    std::string __getstate__() const; 
    void __C_setstate__(const std::string &state);
\end{verbatim}

		
\subsection{Другие методы}
Для получения копии сетки (с отдельной областью памяти) предназначен метод
\begin{verbatim}		
    Mesh copy() const;
\end{verbatim}
Новая сетка является упорядоченной, в ее память переносятся лишь те данные, к которым обеспечивала доступ исходная сетка.

Для заполнения сетки предназначены методы		
\begin{verbatim}		
    void fill(const T &x);
    template <typename T2> void fill(const Mesh<T2, D> &M);
    void fill(const Mesh &M);
    void fill(aiw::IOstream &&S);
    void fill(aiw::IOstream &S);
\end{verbatim}
Метод \verb'fill(const T &x)' заполняет все яячейки значением $x$.

Метод \verb'fill(const Mesh<T2, D> &M)' копирует в сетку содержимое сетки $M$, при этом должен существовать оператор
приведения типа $T_2$ к $T$. Если размеры (в ячейках) заполняемой сетки и сетки $M$ не совпадают, копируются данные лишь из области пересечения 
сеток.

Метод \verb'fill(aiw::IOstream)' загружает сетку из потока (при этом предполагается что тип данных 
совпадает с заполняемой сеткой), и вызывает метод \verb'fill(const Mesh &M)'.

Методы
\begin{verbatim}		
  bool min_max(T &a, T &b, aiw::Ind<D> &pos_a, aiw::Ind<D> &pos_b) const;
  aiw::Vec<2, T> min_max() const;
\end{verbatim}
находит минимальное $a$ и максимальное $b$
значения в ячейках сетки, а так же их индексы (либо возвращают пару
$a, b$), при этом значения \verb'nan' и \verb'inf' игнорируются.
Для проверки используется функция \verb'is_bad_value' из модуля
\verb'debug' (см. раздел \ref{debug:sec}), которая может быть дополнительно перегружена для любого
пользовательского типа. Первый вариант метода возвращает \verb'true'
если в сетке есть хотя бы одно хорошее значение (не \verb'nan' и не \verb'inf').

\subsection{Инстацирование в {\tt Python}}
Для каждого набора параметров шаблон сетки должен быть инстацирован в питон при помомщи утилиты \verb'make'.
Для этого в директории библиотеки \verb'aiwlib' надо набрать команду
\begin{verbatim}		
    make MeshXXX-T-D
\end{verbatim}
где \verb'MeshXXX'~--- имя инстацируемого шаблона в \verb'Python' (оно же имя модуля содержащего инстацированный шаблон),
\verb'T'~--- тип данных ячейки в \verb'C++', \verb'D'~--- размерность.
 Например команда 
\begin{verbatim}		
    make MeshF3-float-3
\end{verbatim}
создаст модуль \verb'MeshF3' содержащий класс \verb'MeshF3' отвечающий шаблону \verb'Mesh<float,3>'.

