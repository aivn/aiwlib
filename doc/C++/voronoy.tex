\section{Дискретизация разбиения Вороного на равномерной сетке в $D$--мерном пространстве~--- модуль {\tt voronoy}}
Модуль \verb'voronoy' предоставляет функцию
\begin{verbatim}
    template <int D, typename T1, typename T2>
    void voronoy(const std::vector<Vec<D, T1> > &src, aiw::Mesh<T2, D> dst);
\end{verbatim}
дискретизирующую разбиение Вороного в $D$--мерном пространстве.
Вектор \verb'src' содержит множество исходных точек,
после вызова функции в ячейках сетки \verb'dst' будут прописаны номера ближайших
к центрам ячеек сетки \verb'dst' из множества \verb'src' (считая с нуля до \verb'src.size()-1' включительно).

При дискретизации сетка \verb'dst' разбивается на $2^D$ одинаковых $D$--мерных параллелепипедов,
затем для каждрого угла перебором находится ближайшая точка из множества \verb'src'.
Если все углы параллелепипеда относятся к одной и той же точке, весь параллелепипед относится к этой точке.
В противном случае параллелепипед снова разбивается на $2^D$ частей, и т.д.~--- вплоть до того момента как элементом
разбиения будет одна ячейка сетки \verb'dst'. 
