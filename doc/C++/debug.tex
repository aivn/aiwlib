\section{Средства отладки --- модуль {\tt debug}}\label{debug:sec}
\subsection{Общие замечания}
Модуль \verb'debug' предоставляет ряд макросов для вывода отладочной информации в процессе исполнения
(фактически удобную альтернативу традиционным отладочным \verb'printf') и генерации исключений:
\begin{itemize}
  \item\verb'WOUT(expressions...)' --- вывод информации в \verb'std::cout';
  \item\verb'WERR(expressions...)' --- вывод информации в \verb'std::cerr';
  \item\verb'WSTR(S, expressions...)' --- вывод информации в поток \verb'S', являющийся наследником \verb'std::ostream';
  \item\verb'WCHK(expressions...)' --- проверка реузльтатов вычисления выражений при помощи \verb'isnan' и \verb'isinf',
    генерация исключения если хотя бы одно из выражений не прошло проверку;
  \item\verb'WEXC(expressions...)' --- вывод информации со стека в \verb'std::cerr' при обработке исключения (потокобезопасный вариант); 
  \item\verb'WEXT(expressions...)' --- вывод информации со стека в \verb'std::cerr' при обработке исключения или ошибке сегментирования
    ({\bf не} потокобезопасный вариант); 
  \item\verb'WASSERT(condition, message, expressions...)' --- вывод инфомации в \verb'std::cerr'
    при нарушении условия \verb'condition';
  \item\verb'WARNING(message, expressions...)' --- вывод информации в \verb'std::cerr';
  \item\verb'WRAISE(message, expressions...)' --- вывод информации в \verb'std::cerr'
    и генерация исключения типа \verb'const char *' содержащего выведенную информацию.
\end{itemize}
Например
\begin{verbatim}
  int a; double b[3];
  ...
  WOUT(a, a*b[1], b[0]+b[2]);
\end{verbatim}

Все макросы выводят информацию в виде
\begin{verbatim}
    #filename function() LLL: expr1=value1, expr2=value2 ...
\end{verbatim}
или
\begin{verbatim}
    #filename function() LLL: message [SYSERR] expr1=value1, expr2=value2 ...
\end{verbatim}
где \verb'LLL' --- номер строки в файле \verb'filename' в которой был сгенерирован вывод сообщения,
\verb'function()' --- имя функции в которой был сгенерирован вывод сообщения, \verb'expr'~--- выражение,
\verb'value'~--- значение выражения.

Макрос \verb'WRAISE' допонительно включает в сообщение информацию о системной ошибке \verb'[SYSERR]' (результат работы
фунцкии \verb'strerror(errno)'), если состояние ошибки установлено.

Кроме того, модуль \verb'debug' предоставляет функции
\begin{verbatim}
    void init_signal_hook(int signal);
    void init_segfault_hook();
\end{verbatim}
устанавливающий перехватчик сигнала \verb'SIGSEGV' возбуждающегося при ошибке сегментирования. В этом случае на \verb'stderr' выводится информация со стека исключения
(помещенная туда при помощи макроса \verb'WEXT'), выводится стек вызовов и инструкция (команда для \verb'bash') по его раскрутке. 

\subsection{Синтаксические ограничения}
В качестве выражений (аргументов макросов) могут использоваться любые \verb'rvalue' выражения,
для значений которых реализованы операторы
вывода в поток
\begin{verbatim}
    std::ostream& operator << (std::ostream&, expr_type)
\end{verbatim}

В выражениях могут присутствовать скобки \verb'()[]{}', операции \verb'<>' скобками {\bf не} считаются.
%если в выражении присутствует
%явное задание параметров шаблона, его необходимо брать в скобки, например
%\begin{verbatim}
%     std::complex<double>(a+b)   // неправильно
%     (std::complex<double>(a+b)) // правильно
%\end{verbatim}
Если в выражении есть запятые, части содержащие запятые так же должны быть в скобках, например
\begin{verbatim}
    pow(a, b)   
\end{verbatim}
В противном случае макросы сохраняют работоспособность, но вывод может иметь странный вид, например
\begin{verbatim}
template <int D, typename T> struct V{
    T p[D];
    V(T x){ for(int i=0; i<D; ++i) p[i] = x; }
};

template <int D, typename T> 
std::ostream& operator << (std::ostream& out, const V<D,T> &v){
    out<<"{"<<v.p[0];
    for(int i=1; i<D; ++i) out<<" "<<v.p[i];
    return out<<"}";
}
...
    int a=2;
    WOUT(V<3,int>(a), a*2);
\end{verbatim}
даст вывод
\begin{verbatim}
# ... : V<3={2 2 2}, int>(a), a*2=4
\end{verbatim}
вместо ожидаемого
\begin{verbatim}
# ... : V<3, int>(a)={2 2 2}, a*2=4
\end{verbatim}
Для корректного вывода необходимо использовать дополнительные скобки
\begin{verbatim}
    WOUT((V<3,int>(a)), a*2);
\end{verbatim}

В одной строке может использоваться только один макрос \verb'WEXC' (или \verb'WEXT').

\subsection{Режимы работы}
Макросы \verb'WOUT', \verb'WERR', \verb'WSTR', \verb'WCHK', \verb'WEXC', \verb'WEXT' и \verb'WASSERT' работают только если определен макрос \verb'EBUG'
(например при помощи опции компилятора \verb'-DEBUG'). При сборке на основе шаблонного \verb'aiwlib/Makefile'
макрос \verb'EBUG' по умолчанию отключен, для его подключения необходимо использовать команду
\begin{verbatim}
    make ... debug=on ...
\end{verbatim}

Отличие между вызовом 
\begin{verbatim}
    WASSERT(condition, ...)
\end{verbatim}
и
\begin{verbatim}
    if(!(condition)) WRAISE(...)
\end{verbatim}
заключается в том, что при отключенном режиме отладке макрос \verb'WASSERT' игнорируется полностью (включая проверку условия).

Макросы \verb'WARNIG' и \verb'WRAISE' работают всегда, вне зависимости от макроса \verb'EBUG'.

\subsection{Вывод информации со стека при обработке исключения}
Макросы \verb'WEXC' и \verb'WEXT' выводят свои аргументы на стандартный поток ошибок \verb'std::cerr' при обработке исключения.
Типовой ситуацией является возбуждение исключения в \verb'C++' функции, вызываемой из \verb'Python', если
модуль был собран при помощи шаблонного \verb'aiwlib/Makefile' --- в этом случае в \verb'Python' происходит
вызов стандартного обработчика исключений.

Выводятся только аргументы макросов \verb'WEXC' и \verb'WEXT', размещенных на стеке {\bf до} возбуждения исключения.
Для каждого аргумента выводится значение, которое принимал аргумент в момент вызова макросов \verb'WEXC' и \verb'WEXT'.

В одной строке может использоваться только один макрос \verb'WEXC/WEXT'~--- это связано с размещением в строке
экземпляра класса \verb'aiw::DebugStackTupleFrame', с именем формируемым на основе номера строки.
При этом копии аргументов макроса хранятся внутри экземпляра класса в виде \verb'std::tuple', сообщение формируется и выводится в \verb'std::cerr'
дестуктором экземпляра класса только при необходимости, если возникла исключительная ситуация или сигнал \verb'SIGSEGV'.
Наличие исключительной ситуации проверяется при помощи фунцкии \verb'std::uncaught_exception()'.

Если режим отладки выключен (макрос \verb'EBUG' не определен), макросы \verb'WEXC' и \verb'WEXT' игнорируются.

При работе макросов \verb'WEXC' и \verb'WEXT' экземпляры классов  \verb'aiw::DebugStackTupleFrame' размещаются на стеке, но
макрос \verb'WEXT' дополнительно размещается указатели на создаваемые экземпляры классов в глобальной таблице, что не является потокобезопасным. 
В случае возниковноения сигнала \verb'SIGSEGV' (и если до этого была вызвана функция \verb'init_segfault_hook()'),
все зарегистристрированные в глобальной таблице экземпляры классов \verb'aiw::DebugStackTupleFrame' выводят сообщения в поток \verb'stderr'.

\subsection{Детали реализации}
Модуль \verb'debug' это легкий (звголовчный файл около 100 строк и файл \verb'src/debug.cpp' около 50-ти строк),
независимый от остальных частей библиотеки \verb'aiwlib' модуль.

Вывод выражений построен на рекурсивной функции
\begin{verbatim}
    template <typename ... Args> 
    void aiw::debug_out(std::ostream& out, const char* str, Args ... args);
\end{verbatim}
вызываемой из макросов, в качестве \verb'str' подставляются аргументы макроса в виде строки и затем еще раз
в виде аргументов (уже значений соответствующих выражений). 

Функция \verb'aiw::debug_out' разбирает \verb'str' по запятым, учитывая при этом скобки \verb'()[]{}'.

%Про детали работы макроса \verb'WEXC' было сказано выше.

Работа макроса \verb'WCHK' основана на функции \verb'is_bad_value':
\begin{verbatim}
  template <typename T> bool is_bad_value(T){ return false; }
  inline bool is_bad_value(float x){ return (isnan(x)) || (isinf(x)); }
  inline bool is_bad_value(double x){ return (isnan(x)) || (isinf(x)); }
  template <typename T, typename ... Args> bool is_bad_value(T x, Args ... args){
    return is_bad_value(x) || is_bad_value(args...);
  }
\end{verbatim}
которая так же перегружена для векторов из раздела \ref{vec:sec}. Аргументами макроса могут являться выражения любых типов
(для которых перегружен оператор вывода в \verb'std::ostream'), но если значения этих выражений необходимо проверять~--- необходимо перегузить
функцию \verb'is_bad_value' для соответствующих типов.

При выводе сообщений от всех макросов метод потока вывода \verb'flush' {\bf не} вызывается. 

%\endinput
%%%%%%%%%%%%%%%%%%%%%%%%%%%%%%%%%%%%%%%%%%%%%%
Модуль \verb'debug' подключает и использует следующие стандартные библиотеки:
\begin{itemize}
\item \verb'<cerrno>', \verb'<cstring>' --- доступ к глобальной переменной \verb'errno' и функциии \verb'strerror(errno)';  
\item \verb'<iostream>' --- работа со стандартными потоками вывода;
\item \verb'<sstream>' --- работа с потоком \verb'std::stringstream' в макросах \verb'WEXC';
\item \verb'<exception>' --- определение наличия исключительной ситуации в деструкторе объекта \verb'aiw::DebugStackFrame'
  при помощи фунцкии \verb'std::uncaught_exception()'.
\item \verb'<tuple>' --- контейнер для сбора информации со стека для вывода исключения.
\item \verb'<cmath>' --- макросы \verb'isnan' и \verb'isinf'.
\end{itemize}


 
