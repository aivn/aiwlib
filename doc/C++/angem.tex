\section{Некоторые элементы аналитической геометрии~--- модуль {\tt angem}}
\def\r{{\bf r}}
\def\n{{\bf n}}
\def\p{{\bf p}}
\def\g{{\bf g}}
\def\a{{\bf a}}
\def\b{{\bf b}}

Модуль \verb'angem' предоставляет несколько функицй из аналитической геометрии.

\begin{verbatim}
  template <int D, typename T> Vec<D, T>
  point2plane(const Vec<D, T> &r, const Vec<D, T> &p, const Vec<D, T> &n);
\end{verbatim}
проекция точки $\r$ на плоскость проходящую через точку $\p$ с нормалью $\n$, $n=1$, вычисляется как
$\r-(\r-\p)\cdot \n \cdot \n$.

\begin{verbatim}
    template <int D, typename T> 
    bool cross_plane(Vec<D, T> &res, const Vec<D, T> &r, const Vec<D, T> &g,
                     const Vec<D, T> &p, const Vec<D, T> &n, bool ray=false);
\end{verbatim}
расчет точки пересечения прямой (или луча) проходящей через точку $\r$ в направлении $\g$, $g=1$ и плоскости проходящей через точку $\p$ с нормалью $\n$, $n=1$
вычисляется как
$$
{\tt res} = \r - \r\frac{(\r-\p)\cdot\n}{\n\cdot\g}.
$$
Функция возвращает \verb'true' если проекция существует~--- если луч перескает плоскость, должно выполняться условие
$$
(\r-\p)\cdot\n\cdot\n\cdot\g<0,
$$
кроме того для существования проекции необходимо что бы $|\n\cdot\g|>10^{-8}$.
В случае если проекция сущсетвует, функция возвращает \verb'true', результат записывается в \verb'res'.

Упрощенным (оптимизированным) вариантом этой функции является
\begin{verbatim}
    template <int D, typename T> 
    bool cross_plane(Vec<D, T> &res, const Vec<D, T> &r, const Vec<D, T> &g,
				     const Vec<D, T> &p, int axe, bool ray=false);
\end{verbatim}
где ориентация плоскости задается ортогональной ей осью координат \verb'axe'.

\begin{verbatim}
    template <int D, typename T> Vec<D, T> 
    shoot_box_out(const Vec<D, T> &r, const Vec<D, T> &g, 
                  const Vec<D, T> &a, const Vec<D, T> &b);
\end{verbatim}
расчитывает точку пересечения луча выходящего из точки $\r$ в направлении $\g$, $g=1$ и параллепипеда заданного точками $\a$, $\b$, $\a\le\r$, $\r\le\b$.

