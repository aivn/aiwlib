\section{Сериализация данных в формате {\tt pickle} языка {\tt Python}~--- модуль {\tt pickle}}\label{pickle:sec}
Модуль \verb'pickle' позволяет выгружать сериализовыванные данные из \verb'C++' в формате \verb'pickle',
что бывает полезно при подготовке данных  для последующей обработки в \verb'Python'.

Для сериализации предоставляются следующие структуры и функции
\begin{verbatim}
    struct NoneType{};
    const NoneType None;

    struct Pickle{
        // основные операторы для вывода объектов
        Pickle& operator << (const Pickle& other);
        Pickle& operator << (NoneType);
        Pickle& operator << (bool x);
        Pickle& operator << (int x);
        Pickle& operator << (long x);
        Pickle& operator << (double x);
        Pickle& operator << (const std::string& x); 		
        Pickle& operator << (const char* x);

        template <typename T> Pickle& operator << (const std::complex<T> &x);
		
        // для сериализации записи в словаре вида k:v
        template <typename T1, typename T2> 
        Pickle& operator ()(const T1& k, const T2& v);

        // для модуля objconf 
        template <typename T> void get(const std::string& k, const T& v);
        template <typename T> void set(const std::string& k, T& v);
    };

    template <typename ... Args> Pickle pickle_tuple(const Args& ... args);
    template <typename ... Args> Pickle pickle_list(const Args& ... args);
    template <typename ... Args> Pickle pickle_set(const Args& ... args);
    inline Pickle pickle_dict();
    inline Pickle pickle_class(const char* module, const char *name, bool dict=true);
    
    inline std::ostream& operator << (std::ostream& S, const Pickle &P);
\end{verbatim}
Для сериализации рекомендуется использовать функции вида \verb'pickle_'$XXX$.

Сериализованы могут быть любые данные типа \verb'T', для которых перегружена операция
\begin{verbatim}
    Pickle& operator << (Pickle &P, const T &x);
\end{verbatim}
допускается перегрузка операций для своих типов данных, например код
\begin{verbatim}
    struct A{ int x; double y; bool z; };
    Pickle& operator << (Pickle &P, const A &a){ P<<pickle_tuple(a.x, a.y, a.z); }
\end{verbatim}
будет сериализовывать экземпляры структуры \verb'A' как кортежи. 

Сформированный в итоге экземпляр структуры \verb'Pickle' может быть выведен в поток \verb'std::ostream' (при этом обязательно нужно вывести финальный символ <<точка>>)
и далее загружен в \verb'Python', например код
\begin{verbatim}
    std::cout<<pickle_tuple(
        pickle_list(vec(1.,2,3), ind(2,3), vecf(3,4,5,6)), 
        std::complex<double>(1,2), 
        pickle_dict()(ind(1), "qwe"), 
        None)
     <<'.';
\end{verbatim}
выдаст в итоге сериализованную структуру данных \verb'Python'
\begin{verbatim}
    ([Vec(1.0,2.0,3.0), Ind(2,3), Vecf(3.0,4.0,5.0,6.0)], 
     (1+2j), 
     {Ind(1): 'qwe'}, 
     None)
\end{verbatim}


