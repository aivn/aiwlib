\section{Контрольные точки для остановки и последующего восстановления расчета --- модуль {\tt checkpoint}}
Модуль \verb'checkpoint' состоит из файлов \verb'aiwlib/checkpoint' и \verb'src/checkpoint.cpp' и
предоставляет макрос \verb'CHECKPOINT(ARGS...)', обеспечивающий сохранение в контрольных точках и последующее восстановление расчета.
Макрос работает через глобальный экземпляр класса \verb'CheckPoint'
\begin{verbatim}
    class CheckPoint{
    public:
        void init(const char *path, bool wmode);
    #ifndef SWIG
        template <typename ... Args> 
        bool operator()(const char* fname, int line,
                        const char *argnames, Args& ... args);
    #endif //SWIG
    };
    extern CheckPoint checkpoint;
    #define CHECKPOINT(ARGS...) aiw::checkpoint(__FILE__, __LINE__, #ARGS, ARGS)
\end{verbatim}
Метод \verb'init' задает путь к файлу, содержащему состояние расчета, и режим работы~--- сохранение состояния расчета (аргумент \verb'wmode=true')
или предварительное восстановление состояния расчета (аргумент \verb'wmode=false').
Без вызова метода
\begin{verbatim}
    checkpoint.init(...)
\end{verbatim}
все макросы \verb'CHECKPOINT' всегда возвращают \verb'true'.

В режиме записи состояния расчета, вызовы макросов \verb'CHECKPOINT' всегда возвращают \verb'true',
при этом каждый вызов приводит к записи аргументов макроса в файл, указанный при вызове метода \verb'checkpoint.init'.
Запись производится в бинарном формате, для аргументов макроса должны быть перегружены операции
\begin{verbatim}
    aiw::IOstream& operator < (aiw::IOstream&, const T&);
    aiw::IOstream& operator > (aiw::IOstream&, T&);
\end{verbatim}
см. раздел \ref{binaryio:sec} (этому требованию удовлевотряют практически все контейнеры \verb'aiwlib', встроенные типы, типы \verb'std::string',
\verb'std::vector', \verb'std::list' и \verb'std::map').

В режиме восстановления состояния расчета вызовы макроса \verb'CHECKPOINT' возвращают \verb'false' до тех пор, пока очередь не дойдет до последнего вызова
в режиме записи при предыдущем запуске расчета\footnote{Определение соответствующего вызова производится на основе  имени файла с исходным кодом,
  номера строки в файле и набора имен аргументов}~--- при этом производится восстановление значения аргументов макроса, макрос возвращает \verb'true',
а глобальный экземпляр класса \verb'checkpoint' переводится в режим записи и может быть снова использован для сохранения состояния расчета с последующим восстановлением.

Расстановка вызовов макроса \verb'CHECKPOINT' вдоль трассы выполнения программы (в условиях, циклах и т.д.) позволяет при восстановлении расчета пропускать
уже пройденные ранее участки трассы выполнения. Недостатком такого подхода является некоторое замедление работы~--- при кжадом вызове макроса в режиме записи происходит
сохранение данных на диск.

Пример использования
\begin{verbatim}
    ...
    int main(int argc, const char **argv){
        ...
        checkpoint.init(argv[1], !(argc>2 && std::string(argv[2])=="-load"));
        aiw::Mesh<float, 3> data; // сетка определяющая состояние расчета 
        ...
        for(int i=0; CHECKPOINT(i, data) && i<100; i++){
            // основной расчетный цикл
            ...
        }
        ...
    }
\end{verbatim}



