\section{Настройка пользовательских классов~--- модуль {\tt objconf}}\label{objconf:sec}
Модуль \verb'objconf' предоставляет макрос
\begin{verbatim}
    #define CONFIGURATE(ARGS...)
\end{verbatim}
который при вызове внутри пользовательского класса
\begin{verbatim}
    struct A{
        int x; double y; bool z;
        CONFIGURATE(x, y, z);
    };
\end{verbatim}
создает в классе метод
\begin{verbatim}
    template<typename ConfT>					
    void configurate(ConfT &conf, bool wmode, const char *prefix="");
\end{verbatim}
который в зависимости от значения аргумента \verb'wmode' вызывает методы
\begin{verbatim}
    conf.set(prefix+name, parametr); // wmode==true
    conf.get(prefix+name, parametr); // wmode==false
\end{verbatim}
для каждого из полей класса, перечисленных в аргументах макроса \verb'CONFIGURATE'.
Аргумент \verb'prefix' позволяет добавить префикс к именам всех полей при чтении/записи.

Созданный метод \verb'configurate(...)' может использоваться при чтении/записи конфигурационных файлов (раздел~\ref{configfile:sec}),
записи состояния объекта в формате \verb'pickle' (раздел~\ref{pickle:sec}) и т.д.
