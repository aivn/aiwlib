\section{Различные варинаты интерполяции --- модуль {\tt inteporlations}}\label{interpolations:sec}
\subsection{Локальный кубический сплайн}
Локальный кубический сплайн строится по четырем отсчетам интерполируемой функции $f_{-1,0,1,2}$
как
$$
\widetilde f(x) = \sum\limits_{i=-1}^2 W_i(x) f_i = \sum_{j=0}^3 S_j x^j, \qquad x\in[0,1].
$$
При этом должны выполняться следующие условия
$$
\widetilde f(0) = f_0, \qquad
\widetilde f(1) = f_1, \qquad
\widetilde f'(0)  = \frac{f_1-f_{-1}}2, \qquad
\widetilde f'(1)  = \frac{f_2-f_{0}}2, 
$$
что дает в итоге СЛАУ на коэффициенты $S_i$:
$$
\left(
\begin{matrix}
  1 & 0 & 0 & 0 \\
  1 & 1 & 1 & 1 \\
  0 & 1 & 0 & 0 \\
  0 & 1 & 2 & 3 \\
\end{matrix}
\right)\cdot {\bf S} = \left(
\begin{matrix}
f_0 \\ f_1 \\ \displaystyle\frac{f_1-f_{-1}}2 \\  \displaystyle\frac{f_2-f_{0}}2
\end{matrix}
\right),
\qquad
{\bf S} = \left(
\begin{matrix}
  1 & 0 & 0 & 0 \\
  0 & 0 & 1 & 0 \\
  -3 & 3 & -2 & -1 \\
  2 & -2 & 1 & 1 \\
\end{matrix}
\right)\cdot \left(
\begin{matrix}
f_0 \\ f_1 \\ \displaystyle\frac{f_1-f_{-1}}2 \\  \displaystyle\frac{f_2-f_{0}}2
\end{matrix}
\right),
$$
откуда
\begin{multline}
\widetilde f = f_0 + \frac{f_1-f_{-1}}2 x + \left[-3 f_0 + 3 f_1 + f_{-1} - f_1 + \frac{f_0-f_2}2 \right] x^2
+ \left[2f_0 -2f_1 + \frac{f_1 - f_{-1} + f_2-f_0}2 \right] x^3
=\\=
\left(-\frac x2 + x^2 -\frac{x^3}2\right)f_{-1}
+\left(1-\frac52 x^2+ \frac 32x^3\right)f_0
+\left(\frac x2 + 2x^2 - \frac32 x^3 \right)f_1
+\left(-\frac{x^2}2 + \frac{x^3}2 \right)f_2.
\notag
\end{multline}
Веса при $f_{-1...2}$ рассчитываются при помощи функции \verb'C++' модуля \verb'interpolations'
\begin{verbatim}
    inline Vec<4> interpolate_cube_weights(double x);
\end{verbatim}

\subsection{Кубический $B$--сплайн}
Фрагмент кубического $B$-сплайна на участке $[x_j,x_{j+1}]$ определяется как
$$
\widetilde f(x) = \sum_{i=j-1}^{j+2} f_i N_{i,4}(x),\qquad x\in[0,1],
$$
где\footnote{почему то во всех источниках выражения для $N_{i,k}$ сдвинуты вперед на единицу, что приводит к ассиметрии. На самом деле центр функции $N_{i,k}$
  должен совпадать с точкой $x_i$?} на равномерной сетке по $x$, при $x_{i+1}-x_i=1\forall i$
\begin{multline}
  N_{i,k}(x) = \frac{\left(x-x_{i-\frac k2}\right)N_{i-\frac12,k-1}(x) + \left(x_{i+\frac k2}-x\right)N_{i+\frac12,k-1}(x)}{k-1}
  =\\=
\frac{\chi_{-k}N_{i-\frac12,k-1} - \chi_{k}N_{i+\frac12,k-1}(x)}{k-1},
\qquad N_{i,1}(x) \equiv \Pi_i = \left\{\begin{array}{ll} 1,& x_{i-\frac12}\le x<x_{i+\frac12},\\ 0, & \rm else, \end{array}\right.
\notag
\end{multline}
где $\chi_k=x-x_{i+\frac k2}$. Тогда, с учетом того что $x-\left(x_{i+\frac k2}+\Delta\right) = \chi_{k+2\Delta}$, получаем 
$$
N_{i,2}(x) = \chi_{-2}\Pi_{i-\frac12} - \chi_2\Pi_{i+\frac12},
$$
\begin{multline}
  N_{i,3}(x) = \frac{\chi_{-3} N_{i-\frac12,2} -\chi_3 N_{i+\frac12,2}}2
  =\frac12\bigg\{\chi_{-3}\Big[\chi_{-3}\Pi_{i-1} -\chi_1\Pi_{i}\Big] -\chi_3\Big[\chi_{-1}\Pi_{i} -\chi_3\Pi_{i+1}\Big]\bigg\}
  =\\=\frac12\bigg\{\chi_{-3}^2 \Pi_{i-1} - \Big[\chi_{-3}\chi_1 + \chi_{-1}\chi_3\Big]\Pi_{i} +\chi_3^2\Pi_{i+1}\bigg\},
  \notag
\end{multline}
\begin{multline}
  N_{i,4}(x) = \frac{\chi_{-4} N_{i-\frac12,3} -\chi_4 N_{i+\frac12,3}}3 =
  \frac16\bigg\{\chi_{-4}^3 \Pi_{i-\frac32} - \chi_{-4}\Big[\chi_{-4}\chi_0 + \chi_{-2}\chi_2\Big]\Pi_{i-\frac12} +\chi_{-4}\chi_2^2\Pi_{i+\frac12}
  -\\-\chi_{-2}^2\chi_4 \Pi_{i-\frac12} + \Big[\chi_{-2}\chi_2 + \chi_{0}\chi_4\Big]\chi_4\Pi_{i+\frac12} -\chi_4^3\Pi_{i+\frac32}
  \bigg\}=\\=
\frac16\bigg\{\chi_{-4}^3 \Pi_{i-\frac32} - \Big[ \chi_{-4}^2\chi_{0} + \chi_{-4}\chi_{-2}\chi_2 + \chi_{-2}^2\chi_4\Big]\Pi_{i-\frac12} 
  +\Big[ \chi_{-4}\chi_2^2 + \chi_{-2}\chi_2\chi_4 + \chi_0\chi_4^2 \Big]\Pi_{i+\frac12} -\chi_4^3\Pi_{i+\frac32}
  \bigg\}
  .\notag
\end{multline}
В итоге получаем
$$
\widetilde f = \frac16\bigg\{ - \chi_2^3 f_{-1} +\Big[ \chi_{-4}\chi_2^2 + \chi_{-2}\chi_2\chi_4 + \chi_0\chi_4^2 \Big]f_0
-\Big[ \chi_{-2}^2\chi_{2} + \chi_{-2}\chi_{0}\chi_4 + \chi_{0}^2\chi_6 \Big] f_1 +\chi_0^3 f_2 \bigg\},
$$
или после преобразований в \verb'maxima':

$\chi_{-4}\chi_2^2 + \chi_{-2}\chi_2\chi_4 + \chi_0\chi_4^2\to$
\begin{verbatim}
(%i1) expand((x+2)*(x-1)^2 + (x+1)*(x-1)*(x-2)+x*(x-2)^2);
                                   3      2
(%o1)                           3 x  - 6 x  + 4
\end{verbatim}

$ \chi_{-2}^2\chi_{2} + \chi_{-2}\chi_{0}\chi_4 + \chi_{0}^2\chi_6\to$
\begin{verbatim}
(%i2) expand((x+1)^2*(x-1)+(x+1)*x*(x-2)+x^2*(x-3));
                                3      2
(%o2)                        3 x  - 3 x  - 3 x - 1
\end{verbatim}
$$
\widetilde f = \frac16\left\{ \Big[-x^3 +3x^2-3x+1 \Big]f_{-1} +  \Big[3x^3-6x^2+4\Big] f_0 + \Big[-3x^3+3x^2+3x+1\Big] f_1 + x^3 f_2 \right\}.
$$
Веса при $f_{-1...2}$ рассчитываются при помощи функции \verb'C++' модуля \verb'interpolations'
\begin{verbatim}
    inline Vec<4> interpolate_Bspline_weights(double x);
\end{verbatim}


\subsection{Функции модуля {\sf interpolations}}
Для проведения интерполяции используется параметризованная функция (frontend)
\begin{verbatim}
    template<typename C>
    typename C::cell_type interpolate(const C& arr,            // контейнер
                                      const Ind<C::dim> &pos,  // позиция ячейки
                                      const Vec<C::dim> &x,    // координаты в ячейке
                                      const Ind<3> &Itype);    // тип интерполяции
\end{verbatim}
где \verb'Itype' --- битовые маски. Соответствующие некоторой оси \verb'axe' биты в \verb'Itype' означают:
\verb'Itype[0]&(1<<axe)==0'~--- интерполяция нулевого порядка (кусочно--постоянная
в рамках ячейки), \verb'Itype[0]&(1<<axe)==1', \verb'Itype[1]&(1<<axe)==0'~--- линейная интерполяция между центрами ячеек,
\verb'Itype[0]&(1<<axe)==1', \verb'Itype[1]&(1<<axe)==1', \verb'Itype[2]&(1<<axe)==0'~--- локальный кубический сплайн,
\verb'Itype[0]&(1<<axe)==1', \verb'Itype[1]&(1<<axe)==1', \verb'Itype[2]&(1<<axe)==1'~--- кубический $B$--сплайн.



%Че то я туплю. Вот готовый ответ из \verb'http://sernam.ru/book_mm3d.php?id=93', над строчкой <<{\it Из уравнения (5-83) получаем параметрический В-сплайн}>>
%$$
%\widetilde f = (1-x)^3f_{-1} + 3x(1-x)^2 f_0 + 3x^2(1-x) f_1 + x^3 f_2
%$$
