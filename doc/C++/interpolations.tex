\section{Различные варинаты интерполяции --- модуль {\tt inteporlations}}
Локальный кубический сплайн строится по четырем отсчетам интерполируемой функции $f_{-1,0,1,2}$
как
$$
\widetilde f(x) = \sum\limits_{i=-1}^2 W_i(x) f_i = \sum_{j=0}^3 S_j x^j, \qquad x\in[0,1].
$$
При этом должны выполняться следующие условия
$$
\widetilde f(0) = f_0, \qquad
\widetilde f(1) = f_1, \qquad
\widetilde f'(0)  = \frac{f_1-f_{-1}}2, \qquad
\widetilde f'(1)  = \frac{f_2-f_{0}}2, 
$$
что дает в итоге СЛАУ на коэффициенты $S_i$:
$$
\left(
\begin{matrix}
  1 & 0 & 0 & 0 \\
  1 & 1 & 1 & 1 \\
  0 & 1 & 0 & 0 \\
  0 & 1 & 2 & 3 \\
\end{matrix}
\right)\cdot {\bf S} = \left(
\begin{matrix}
f_0 \\ f_1 \\ \displaystyle\frac{f_1-f_{-1}}2 \\  \displaystyle\frac{f_2-f_{0}}2
\end{matrix}
\right),
\qquad
{\bf S} = \left(
\begin{matrix}
  1 & 0 & 0 & 0 \\
  0 & 0 & 1 & 0 \\
  -3 & 3 & -2 & -1 \\
  2 & -2 & 1 & 1 \\
\end{matrix}
\right)\cdot \left(
\begin{matrix}
f_0 \\ f_1 \\ \displaystyle\frac{f_1-f_{-1}}2 \\  \displaystyle\frac{f_2-f_{0}}2
\end{matrix}
\right),
$$
откуда
\begin{multline}
\widetilde f = f_0 + \frac{f_1-f_{-1}}2 x + \left[-3 f_0 + 3 f_1 + f_{-1} - f_1 + \frac{f_0-f_2}2 \right] x^2
+ \left[2f_0 -2f_1 + \frac{f_1 - f_{-1} + f_2-f_0}2 \right] x^3
=\\=
\left(-\frac x2 + x^2 -\frac{x^3}2\right)f_{-1}
+\left(1-\frac52 x^2+ \frac 32x^3\right)f_0
+\left(\frac x2 + 2x^2 - \frac32 x^3 \right)f_1
+\left(-\frac{x^2}2 + \frac{x^3}2 \right)f_2.
\notag
\end{multline}

Фрагмент кубического $B$-сплайна определяется как
$$
\widetilde f(x) = \sum_{i=-1}^3 f_i N_{i,4}(x),\qquad x\in[0,1],
$$
где
$$
\quad N_{i,k}(x) = \frac{(x-x_i)N_{i,k-1}(x)}{x_{i+k-1}-x_i} + \frac{(x_{i+k}-x)N_{i+1,k-1}(x)}{x_{i+k}-x_{i+1}},
\quad N_{i,1}(x) \equiv \Pi_i = \left\{{1,\,x_i\le x<x_{i+1},\atop 0,\,\rm else. }\right.
$$
Обозначим $\chi_k=x-x_{i+k}$, тогда с учетом того что $x_i-x_j=i-j$
$$
N_{i,2}(x) = \chi_0\Pi_i - \chi_2\Pi_{i+1},
$$
\begin{multline}
  N_{i,3}(x) = \frac{\chi_0 N_{i,2} -\chi_3 N_{i+1,2}}2
  =\frac12\bigg\{\chi_0\Big[\chi_0\Pi_i -\chi_2\Pi_{i+1}\Big] -\chi_3\Big[\chi_1\Pi_{i+1} -\chi_3\Pi_{i+2}\Big]\bigg\}
  =\\=\frac12\bigg\{\chi_0^2 \Pi_i - \Big[\chi_0\chi_2 + \chi_1\chi_3\Big]\Pi_{i+1} +\chi_3^2\Pi_{i+2}\bigg\},
  \notag
\end{multline}
\begin{multline}
  N_{i,4}(x) = \frac{\chi_0 N_{i,3} -\chi_4 N_{i+1,3}}3 =\frac16\bigg\{\chi_0^3 \Pi_i - \Big[\chi_0^2\chi_2 + \chi_0\chi_1\chi_3\Big]\Pi_{i+1} +\chi_0\chi_3^2\Pi_{i+2}
  -\\-\chi_1^2\chi_4 \Pi_{i+1} + \Big[\chi_1\chi_3\chi_4 + \chi_2\chi_4^2\Big]\Pi_{i+2} -\chi_4^3\Pi_{i+3}
  \bigg\}=\\=
\frac16\bigg\{\chi_0^3 \Pi_i - \Big[\chi_0^2\chi_2 + \chi_0\chi_1\chi_3 + \chi_1^2\chi_4 \Big]\Pi_{i+1} 
  +\Big[\chi_1\chi_3\chi_4 + \chi_0\chi_3^2 + \chi_2\chi_4^2\Big]\Pi_{i+2} -\chi_4^3\Pi_{i+3}
  \bigg\}.  \notag
\end{multline}
%В итоге получаем
%$$
%\widetilde f = -\frac16\Big[\chi_0^2\chi_2 + \chi_0\chi_1\chi_3 + \chi_1^2\chi_4 \Big] f_{-1}
%$$

Че то я туплю. Вот готовый ответ из \verb'http://sernam.ru/book_mm3d.php?id=93', над строчкой <<{\it Из уравнения (5-83) получаем параметрический В-сплайн}>>
$$
\widetilde f = (1-x)^3f_{-1} + 3x(1-x)^2 f_0 + 3x^2(1-x) f_1 + x^3 f_2
$$
