\section{Решение СЛАУ с трехдиагональными матрицами (методы прогонки) --- модуль {\tt 3diag}}
Модуль \verb'3diag' предоставляет методы прогонки (обычной и циклической) для решения СЛАУ с трехдиагональными матрицами
$$
M_{i-1,i}X_{i-1}+M_{i,i}X_{i}+M_{i+1,i}X_{i+1} = R_i,
$$
\begin{verbatim}
    template <typename T>
    void shuttle_alg(const Mesh<T, 2> &M, const Mesh<T, 1> &R, Mesh<T, 1> &X);
    template <typename T>
    void cyclic_shuttle_alg(const Mesh<T, 2> &M, const Mesh<T, 1> &R, Mesh<T, 1> &X);
\end{verbatim}
где \verb'M' --- тридиагональная матрица, сетка размерами \verb'3хN', главная диагональ расположена в ячейках \verb'(1,0...N-1)',
для обычной прогонки ячейки \verb'(0,0)' и \verb'(2,N-1)' игнорируются;
\verb'R'~--- правая часть, \verb'X'~--- решение. Сетки \verb'R' и \verb'N' должны иметь размер \verb'N' ячеек.

Алгоритм для обычной прогонки взят из английской Википедии\footnote{{\tt https://en.wikipedia.org/wiki/Tridiagonal\_matrix\_algorithm}},
алгоритм для циклической прогонки взят у Самарского\footnote{Самарский А.А. <<Введение в теорию разностных схем>>. М.: Наука, 1971.~--- 553~с.\\
{\tt http://info.alnam.ru/book\_sub.php?id=91}} с изменением знаков у главной диагонали и правой части.

Проверка на диагональное преобладание {\bf не} проводится.
