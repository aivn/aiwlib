\section{Модуль {\tt timedate} --- удобное представление времени
  и даты}
Модуль {\tt timedate} предоставляет классы {\tt Time} и {\tt Date} для удобной
работы со временем и датой. Под временем здесь понимается не текущеее время, а
время, затраченное на какие либо действия (например работу программы), либо
время как разность между двумя датами (результат может быть отрицательным).
Под датой понимается дата и текущее время (число секунд с начала эпохи {\tt Unix}).

Кроме того, модуль предоставляет исключения
\begin{itemize}
\item \verb'class BaseTimeException(Exception)' --- базовое исключение для
  всех исключений модуля;
\item \verb'class IllegalInitTimeValue(BaseTimeException)' --- некорректное
  значение для инициализации экземпляра класса {\tt Time};
\item \verb'class IllegalInitDateValue(BaseTimeException)' --- некорректное
  значение для инициализации экземпляра класса {\tt Date};
\item \verb'class IllegaMuleTimeValue(BaseTimeException)' ---
  умножение/деление экземпляра класса {\tt Time} на некорректное
  значение.
\end{itemize}

\subsection{Класс {\tt Time}}
Конструктор класса принимает  число секунд (целое
либо с плавающей точкой) или строку вида \verb|'[-]hh:mm[:ss]'| или \verb|'[-]ss'|  в качестве неименованного аргумента, либо
именованные аргументы {\tt h, m, s} для задания часов, минут и секунд. Во
втором случае достаточно указать хотя бы один именованный аргумент. Если время
отрицательное, во втором случае все значения именованных аргументов должны
быть отрицательными. 

Экземпляр класса {\tt Time} имеет следующие поля:
\begin{itemize}
\item \verb'val' --- длина временного промежутка в секундах со знаком;
\item \verb'h' --- целое число часов без знака;
\item \verb'm' --- целое число минут без знака;
\item \verb's' --- число секунд без знака с плавающей точкой.
\end{itemize}
Возможна установка (изменение) полей {\tt h, m, s} и согласование значения
поля {\tt val} при вызове метода {\tt setval()}.

Статическое поле {\tt precision = 3} задает количество знаков после запятой
при отображении секунд. 

Метод {\tt \_\_call\_\_( **d )} принимает именованные аргументы {\tt h, m, s}
и возвращает экземпляр класса {\tt Time} с соответствующими измененными полями
и согласованным полем {\tt val}.

Экземпляр класса {\tt Time} может быть приведен к целому числу или числу с
плавающей точкой (возвращается значение поля {\tt val}), или к строке
вида \verb|'[-]hh:mm[:ss]'|. Метод {\tt \_\_repr\_\_()} возвращает строку вида
\verb|Time('[-]hh:mm[:ss]')|.

Определены методы {\tt \_\_getstate\_\_/\_\_setstate\_\_} для сериализации при
помощи модуля {\tt pickle}. 

Арифметические операторы приведены в таблице \ref{racs:mytime:op:table}.

\begin{table}
\begin{center}
\begin{tabular}{|rcl|rcl|}
\hline
\verb't1 CMP x' &$\to$& \verb't1.val CMP Time(x).val' &
\verb'd1 CMP x' &$\to$& \verb'd1.val CMP Date(x).val' \\
\verb'x CMP t2' &$\to$& \verb'Time(x).val CMP t2.val' &
\verb'x CMP d2' &$\to$& \verb'Date(x).val CMP d2.val' \\
\hline
\verb'-t2' &$\to$& \verb'Time( -t2.val )' &  
\verb'abs(t2)' &$\to$& \verb'Time( abs(t2.val) )' \\
\verb't1+x' &$\to$& \verb'Time( t1.val+Time(x).val )' &
\verb'x+t2' &$\to$& \verb'Time( Time(x).val+t2.val )' \\
\verb't1-x' &$\to$& \verb'Time( t1.val-Time(x).val )' &
\verb'x-t2' &$\to$& \verb'Time( Time(x).val-t2.val )' \\
\verb't1+d2' &$\to$& \verb'Date( t1.val+d2.val )' &
\verb'd1+t2' &$\to$& \verb'Date( d1.val+t2.val )' \\
\verb'd1+x' &$\to$& \verb'Date( d1.val+Time(x) )' &
\verb'x+d2' &$\to$& \verb'Date( Time(x)+d2.val )' \\
\verb'd1-x' &$\to$& \verb'Date( d1.val-Time(x) )' &
\verb'd1-d2' &$\to$& \verb'Time( d1.val-d2.val )' \\
\hline
\verb't1*x' &$\to$& \verb'Time( t1.val*float(x) )' &
\verb'x*t2' &$\to$& \verb'Time( t2.val*float(x) )' \\
\verb't1/x' &$\to$& \verb'Time( t1.val/float(x) )' &
\verb't1/t2' &$\to$& \verb't1.val/t2.val' \\
\hline
\end{tabular}
\end{center}
\caption{Арифметические операторы для классов {\tt Time} и {\tt Date}. Здесь
  {\tt t1, t2} экземпляры класса {\tt Time}; {\tt d1, d2} экземпляры класса
  {\tt Date}; {\tt x} некоторое значение (число или строка); {\tt CMP} один из
операторов сравнения $<,\,>,\,\le,\,\ge,\, =,\,\neq$.}\label{racs:mytime:op:table}
\end{table}

\if{xxx}
Определен оператор сравнения с любым значением, которое может рассматриваться
в качестве неименованного аргумента конструктора {\tt Time}~--- значение сначала
приводиться к экземпляру класса {\tt Time}, затем сравниваются поля {\tt
  val}. 

Оператор {\tt \_\_neg\_\_()} (унарный минус) возвращает экземпляр класса {\tt
  Time} с измененным знаком поля {\tt val}.

Оператор {\tt \_\_abs\_\_()} (взятие модуля) возвращает экземпляр класса {\tt
  Time} положительным значением поля {\tt val}.

Определен оператор сложения для любого значения, которое может рассматриваться
в качестве неименованного аргумента конструктора {\tt Time} (возвращает
экземпляр класса {\tt Time} с суммарным значением поля {\tt val}) и оператор
сложения с экземпляром класса {\tt Date}~--- возвращает экземпляр класса {\tt
  Date}, сдвинутый на величину поля {\tt val}. 

Определен оператор вычитания для любого значения, которое может рассматриваться
в качестве неименованного аргумента конструктора {\tt Time}~--- возвращает
экземпляр класса {\tt Time} с разностью значений полей {\tt val}. 

Определен оператор умножения на любое значение {\tt other}, которое может быть приведено к
числу с плавающей точкой~--- возвращает
экземпляр класса {\tt Time} с полем {\tt val}, умноженным на значение {\tt
  other}.

Определен оператор деления на любое значение {\tt other}, которое может быть приведено к
числу с плавающей точкой~--- возвращает
экземпляр класса {\tt Time} с полем {\tt val}, деленное на значение {\tt
  other}.

Определен оператор деления на любое значение {\tt other}, 
которое может рассматриваться
в качестве неименованного аргумента конструктора {\tt Time}~---
возвращает число с плавающей точкой  равное {\tt val/Time(other)}.
\fi

\subsection{Класс {\tt Date}}
Конструктор класса без параметров создает экземпляр класса, соответствующий
текущей дате. Конструктор может принимать один неименованный параметр~--- число секунд (целое
либо с плавающей точкой либо строковое представления числа) с начала эпохи {\tt Unix}, другой экземпляр класса
{\tt Date} или строку вида
\verb|'year.month.day-hour:minute:secs'|. Год представляется четырьмя
цифрами; месяц, день, час и минуты двумя цифрами; секунды числом с плавающей точкой. 

Конструктор может принимать один
именованные аргументы {\tt Y} (год), {\tt M} (месяц), {\tt D} (день), {\tt h}
(час), {\tt m} (минуты) и {\tt s} (секунды). 
Во втором случае достаточно указать хотя бы один именованный аргумент. 

Экземпляр класса {\tt Date} имеет следующие поля:
\begin{itemize}
\item \verb'val' --- число секунд с начала эпохи {\tt Unix};
\item \verb'Y' --- год (четыре цифры);
\item \verb'M' --- номер месяца;
\item \verb'D' --- день месяца;
\item \verb'h' --- час в 24-х часовом представлении;
\item \verb'm' --- минуты;
\item \verb's' --- секунды.
\end{itemize}
Возможна установка (изменение) полей {\tt Y, M, D, h, m, s} и согласование значения
поля {\tt val} при вызове метода {\tt setval()}.

Метод {\tt \_\_call\_\_( **d )} принимает именованные аргументы {\tt Y, M, D, h, m, s}
и возвращает экземпляр класса {\tt Date} с соответствующими измененными полями
и согласованным полем {\tt val}.

Экземпляр класса {\tt Time} может быть приведен к целому числу или числу с
плавающей точкой (возвращается значение поля {\tt val}), или к строке
вида \verb|'YYYY.MM.DD-hh:mm:ss'|. Метод {\tt \_\_repr\_\_()} возвращает строку вида
\verb|Time('YYYY.MM.DD-hh:mm:ss')|.

Определены методы {\tt \_\_getstate\_\_/\_\_setstate\_\_} для сериализации при
помощи модуля {\tt pickle}. 

Арифметические операторы приведены в таблице \ref{racs:mytime:op:table}.

\if{xxx}
Определен оператор сравнения с любым значением, которое может рассматриваться
в качестве неименованного аргумента конструктора {\tt Date}~--- значение сначала
приводиться к экземпляру класса {\tt Date}, затем сравниваются поля {\tt
  val}. 

Определен оператор сложения для любого значения, которое может рассматриваться
в качестве неименованного аргумента конструктора {\tt Time}~--- возвращает
экземпляр класса {\tt Date} с суммарным значением поля {\tt val}. 

Определен оператор вычитания для любого значения, которое может рассматриваться
в качестве неименованного аргумента конструктора {\tt Time}~--- возвращает
экземпляр класса {\tt Date} с разностью значений полей {\tt val}. 

Определен оператор вычитания для любого значения, которое может рассматриваться
в качестве неименованного аргумента конструктора {\tt Date}~--- возвращает
экземпляр класса {\tt Time} с разностью значений полей {\tt val}. 
\fi
%\subsection{}
%( '@weekday', 'timemodule.localtime(timemodule.time())[6]', 'номер дня недели' ),
%( '@today', 'reg(Date(h=0,m=0,s=0),Date())', 'сегодняшний день (интервал времени )' ),
%( '@yesterday', 'reg(Date(h=0,m=0,s=0)-86400,Date(h=0,m=0,s=0))', 'вчерашний день (интервал времени)' ),
%( '@week', 'reg(Date(h=0,m=0,s=0)-86400*weekday,Date())', 'текущая неделя (интервал времени)' ),
%( '@lastweek', 'reg(Date(h=0,m=0,s=0)-86400*(weekday+7),Date(h=0,m=0,s=0)-86400*weekday)', 'прошлая неделя (интервал времени)' ),
