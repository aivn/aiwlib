\section{Модуль {\sf mixt.py} --- набор служебных функций и классов}
Модуль {\sf mixt.py} содержит ряд функций, использующихся остальными модулями
библиотеки {\sf raclib}.

\subsection{Различные системные функции}
Функция \verb'except_report( stderr=sys.stderr )' выводит отчет о последней
ошибке (возбужденном исключении) включая стек в файловый объект {\sf stderr} и возвращает
список строк отчета об ошибке виде результата. 

Функция \verb'get_checksums( Lf )' вовращает список контрольных сумм для
файлов из списка {\sf Lf}, контрольные суммы считаются при помощи утилиты {\sf
md5sum}.

Функции \verb'time2string( t, precision=3 )' и \verb'string2time( S )'
преобразуют число секунд {\sf t} в строку {\sf S} вида \verb|'hours:mm:sec'| и
обратно. Число знаков после десятичной точки при отображении секунд задается
аргументом {\sf precision}. Функции устаревшие, рекомендуется использовать класс {\sf mytime.Time}.

Функции \verb'date2string( d )' и \verb'string2date( S )'
преобразуют дату {\sf d} (число секунд  с начала эпохи {\sf Unix}) в строку {\sf S} вида \verb|'YYYY:MM:DD-hh:mm:sec'| и
обратно. Функции устаревшие, рекомендуется использовать класс {\sf mytime.Date}.

Функция \verb'size2string( sz )' преобразует размер {\sf sz} в байтах в строку
вида '$XXX${\sf K}' либо '$XXX.X${\sf M}' либо '$XXX.X${\sf G}' либо '$XXX.X${\sf T}'.

Функция \verb'GetLogin()' возвращает имя пользователя, пытаясь сначала вызвать
функцию \verb'os.getlogin()' а в случае ошибки возвращает занчение переменной
окружения {\sf USER}


Функция \verb'compare( name, patterns )' проверяет  при помощи функции {\sf
  fnmatch.fnmatch(...)} \cite{GVR} соответствует ли строка {\sf name} одному из
шаблонов в списке {\sf patterns}. Шаблоны могут иметь стандартный вид {\sf
  shell}, т.е. включать символы \verb|'*'|, \verb|'?'| и т.д. Если найдено соотвествие
хотя бы одному из шаблонов  возвращается {\sf True}, иначе возвращается {\sf False}.

Функция \verb'str_len( S )' возвращает длину строки \verb'S' в символах (при печати),
учитывая что символы {\sf utf-8} занимают два байта в памяти и один символ на
печати.

Функция \verb'get_tty_width()' возвращает ширину текущего терминала, запуская
в {\sf shell} команду \verb'stty size' и анализируя ее вывод.

Функция
\verb'table2strlist( LL, pattern=None, s_line=1, s_empty=2, s_bound=1, max_len=None )'
возвращает список форматированных строк (без символа конца строки
\verb|'\n'|). Аргумент \verb'LL' это таблица (список списков), для включения разделителя
(горизонтальной линии) необходимо вставить в \verb'LL' значение {\sf
  None}, все остальные элементы \verb'LL' должны быть списками или кортежами
одинаковой длины и содержать величины, которые при выводе будут
преобразовываться к строке при помощи функции {\sf str()}. Аргумент
\verb'pattern' задает паттерн аналогичный заголовку таблиц \LaTeX, т.е. может
содержать символы \verb|'r'|, \verb|'c'|, \verb|'l'| для обозначения
выравнивания содержимого колонки или символ \verb-'|'- для задания
вертикальной линии между колонками. Аргументы \verb's_line', \verb's_empty',
\verb's_bound' задают число пробелоов между текстом и вертикальными
разделительными линиями, между колонками без вертикальных разделительных линия
линий и между колонками и краями (если по краям нет линий). Аргумент
\verb'max_len' задает максимальную длину строк в выводимом списке (по
умолчанию без ограничений), лишние символы отбрасываются.



\subsection{Создание уникальных директорий расчета}
Функция  \verb'make_unique_path( base_path, num=3 )' генерирует уникальное имя
директории, добавляя к \verb'base_path' необходимое минимальное число
дополненное с начала нулями до длины {\sf num}. Например, если в директории
\verb'mypath/' есть поддиректория (или файл) с именем \verb'a023' то вызов 
\verb|make_unique_path( 'mypath/a' )| вернет строку \verb|'mypath/a024'|.

Функция \verb'close_path( p )' добавляет в конец пути {\sf p} символ
\verb|'/'| если {\sf p} не заканчивается этим символом.

Функция \verb'make_path( repository )' создает уникальную директорию в
репозитории \verb"repository" (в том числе и сам репозиторий при необходимости) и возвращает путь к ней.
Название директории формируется из
года, номера недели, дня недели и некоторого трехзначного порядкового номера
уникального для данной даты~--- таким образом расчеты внутри
репозитория автоматически
упорядочиваются по дате (например {\sf
  c07\_00\_1005}~--- пятый расчет проведенный первого января 2007
года). 
В рабочем каталоге (из которого была вызвана функция \verb'make_path') на директорию создается
символическая ссылка \verb|'_'| (одиночный символ подчеркивания).


\subsection{Класс {\sf progressbar} для интерактивного отображения степени
  выполнения вычислений}
Класс {\sf progressbar} обеспечивает интерактивное отображение степени
выполнения вычислений в стандатрном выводе или в стандартном потоке ошибок,
автоматически экстраполируя время завершения вычислений. Конструктор класса
принимает единственный аргумент~--- стандартный поток вывода {\sf sys.stdout} (по умолчанию)
или стандартный поток ошибок {\sf sys.stderr}. 

Класс имеет следующие методы:
\begin{itemize}
\item \verb'clean()' --- сбрасывает состояние объекта для отображения степени
  выполнения нового процесса;
\item \verb|out( progress, prompt='' )| --- отображает степень выполнения {\sf
progress} (число от нуля до единицы) и выводит некоторую дополнительную
  информацию {\sf prompt};
\item \verb|close( prompt='', result='OK' )| --- завершает отображение степени
  выполнения,  выводит  некоторую дополнительную
  информацию {\sf prompt} и результат выполнения {\sf result}. 
\end{itemize}

Отображение производится в виде строки 
\begin{verbatim}
<PROMPT> <RUNTIME> from <TOTALTIME> [###          ]
\end{verbatim}
где \verb'<RUNTIME>' --- время прошедшее с начала выполнения отображаемого
процесса, \verb'<TOTALTIME>'~--- оценка общего времени выполнения процесса
(может быть весьма неточной), строка \verb|'[###          ]'| отображает степень
выполнения. Общая длина строки всегда равняется ширине терминала (определяется
при помощи функции \verb'get_tty_width()'), поэтому не следует использовать слишком
длинные варианты {\sf prompt}.

Если поток, в который экземпляр класса {\tt progressbar} производит вывод, был перенаправлен в файл 
(или изначально являлся обычным файлом и не был связан с терминалом), вывод производится только методом {\tt close()}.

\subsection{Класс {\sf reg} --- задание интервалов значений для сравнения}

Модуль содержит определение глобальных величин
\begin{verbatim}
    inf, nan = float('inf'), float('nan')
\end{verbatim}
для операций сравнения.

Класс {\sf reg} предназначен для задания областей (интервалов) значений
произвольного типа для сравнения. Для класса определены следующие операторы:
\begin{center}
\begin{tabular}{rclcrcl}
\verb'reg(a,b)' &$\to$& $[a,b]$ &\rule{1cm}{0pt}&
\verb'x in reg(a,b)' &$\to$& $x\in [a,b]$ \\
\verb'reg(a,b) < x' &$\to$& $x>b$ && 
\verb'reg(a,b) <= x' &$\to$& $x\ge b$ \\
\verb'x < reg(a,b)' &$\to$& $x<a$ && 
\verb'x <= reg(a,b)' &$\to$& $x\le a$ \\
\verb'reg(a,b) > x' &$\to$& $x<a$ &&
\verb'reg(a,b) >= x' &$\to$& $x\le a$ \\
\verb'x > reg(a,b)' &$\to$& $x>b$ &&
\verb'x >= reg(a,b)' &$\to$& $x\ge b$ \\
\verb'reg(a,b) == x' &$\to$& $x \in [a,b]$ &&
\verb'reg(a,b) != x' &$\to$& $x \notin [a,b] $ \\
\verb'x == reg(a,b)' &$\to$& $x \in [a,b]$ &&
\verb'x != reg(a,b)' &$\to$& $x \notin [a,b] $ \\
\verb'reg(a,b) + X' &$\to$& $[a,b] \cup X$ &&
\verb'X + reg(a,b)' &$\to$& $X \cup [a,b]$ \\
\verb'reg(a,b)*x' &$\to$& $[ax,bx]$ &&
\verb'x*reg(a,b)' &$\to$& $[xa,xb]$ \\
\verb'reg(a,b)/x' &$\to$& $[a/x,b/x]$ &&
\verb'-reg(a,b)' &$\to$& $[b,a]$ \\
\end{tabular}
\end{center}
где $a, b, x$~--- некоторые значения допускающие сравнения, $X$~--- экземпляр
класса {\sf reg} или некоторое значение допускающее сравнение. Оператор суммы
возвращает кортеж из региона и добавленного значения, поэтому операторы
сравнения для него не работают, но работает оператор {\sf 'in'}.
