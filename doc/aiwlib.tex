% -*- mode: LaTeX; coding: utf-8 -*-
\documentclass[12pt]{book}
\usepackage[T2A]{fontenc}
\usepackage[utf8]{inputenc}
\usepackage[russian]{babel}
\usepackage{amsmath}
\usepackage{amssymb}
\usepackage{eufrak}
\usepackage{epsfig}
\usepackage{psfrag}
\usepackage{tabularx}
\usepackage{wrapfig}
\usepackage{subfig}
\setlength{\topmargin}{-0.5in}
\setlength{\oddsidemargin}{-5.mm}
\setlength{\evensidemargin}{-5.mm}
\setlength{\textwidth}{7.in}
\setlength{\textheight}{9.in}

%\usepackage{listings}
%\usepackage{graphicx}
%\lstloadlanguages{ [ LaTeX ] TeX, bash, C++, make, python}
%\lstset{language=python, escapechar=|, extendedchars=true} 

\usepackage[unicode,colorlinks]{hyperref}
%\hypersetup{colorlinks=true, linkcolor=blue, filecolor=blue, pagecolor=blue, urlcolor=blue}
\hypersetup{colorlinks=true, linkcolor=blue, filecolor=blue, urlcolor=blue,
pdftitle={Библиотека aiwlib}, pdfauthor={Иванов А.В, Хилков С.А и Жданов С.А}}

\def\nil{}
\begin{document}
\title{Библиотека {\tt aiwlib}}
\author{
Иванов А.В., Хилков С.А. и Жданов С.А
}

%\institution{Институт прикладной математики им.~М.В.~Келдыша РАН}
%\libcatnum{519.63}
%\city{\rule{0pt}{1.5cm}Москва}
%\date{\number\year\month\day}
\maketitle

%\def\baselinestretch{1.5}
%\setcounter{page}{1}
\tableofcontents


\chapter{Начало работы} % вводное руководство
%\section{Введение}

Интерфейс приложения численного моделирования должен позволять
легко изменять параметры задачи (число которых иногда доходит до сотен или даже тысяч),
выбирать тот или иной алгоритм (в том числе разлиные варианты начальных и граничных условий),
обеспечивать анализ и визуализацию
результатов. Практика показала, что для сложных задач оптимальным
являеться не оконный интерфейс, а интерфейс командной
строки. Фактически речь идет о использовании собственного (или уже
существующего) высокоуровневого интерпретируемого языка,
адаптированного к специфике задачи.

При проведении массовых расчетов (например при анализе зависимости поведения устройства от 
нескольких параметров и построении фазовых диаграмм) требуется механизм, обеспечивающий
многократный автоматический запуск приложения с меняющимися заданным образом параметрами, 
желательно с контролем распределения ресурсов в рамках локальной сети или на кластере.

Для каждого расчета полученные зависимости должны сопровождаться
информацией о использованных параметрах расчета и алгоритмах. Если для
сохранения параметров существует большое количество методик и
библиотек, то сохранение алгоритмов является проблемой, и единственным
приемлемым решением на сегодняшний день является сохранение исходного
кода приложения.

Для анализа результатов необходим многопараметрический поиск по
проведенным расчетам, для чего результаты расчетов должны храниться
специальным, упорядоченным образом. Необходимо обеспечить возможность
поиска по версиям исходного кода. Эту проблему можно решать в ручную,
например размещая результаты расчетов на хорошо структурированном дереве
каталогов~--- однако такой подход требует строгой внутренней культуры пользователя, и
усложняется тем, что в процессе расчетов критерии упорядоченности могут
расширятся и изменяться кардинальным образом.   

При массовых расчетах аккуратное решение вышеописанных проблем может отнимать значительное время и силы. 
В разных рабочих группах
разработаны собственные библиотеки, позволяющие упростить процесс
написания окружения, но единый подход до сих пор не выработан.

Описанная в данной главе система {\tt RACS} ({\tt Results \& Algorithms Control System}~--- система контроля
результатов и алгоритмов) обеспечивает:
\begin{itemize}
\item задание параметров расчетов при запуске для приложений на языках \verb'Python' и \verb'C++';
\item автоматическое сохранение параметров и исходных кодов расчетов;
\item пакетный запуск расчетов (циклы по значениям параметров) и балансировка загрузки, как на локальных машинах так и на кластерах под \verb'MPI';
\item работа с контрольными точками для приложений \verb'C++', в том числе кластерах под \verb'MPI' ({\it в разработке});
\item развитые средства для многопараметрического поиска, анализа и обработки результатов.
\end{itemize}

При разработке \verb'RACS' делались следующие акценты:
\begin{itemize}
\item простота подключения (требуется минимальная модификация отлаженного кода);
\item лаконичный и интуитивно понятный синтаксис при запуске расчетов;
\item возможность обработки результатов средствами операционной системы и сторонними утилитами без потери целостности данных;
\item интеграция с другими утилитами~--- вывод данных в формате \verb'gnuplot' с заголовками \verb'gplt',
  чтение метаинформации о расчетах другими утилитами.
\end{itemize}


Даже для низкоквалифицированного
пользователя  {\tt RACS} автоматически обеспечивает необходимый минимум
<<культуры>> проведения расчетов (сохранение исходных кодов  и
параметров).
В результате пользователь имеет
возможность полностью сконцентрироваться на работе непосредственно  над задачей.

\verb'RACS' написан на языке \verb'Python' и ориентирован в первую очередь
на приложения написанные на
языках \verb'C++' (высокопроизводительное вычислительное ядро) и \verb'Python' (верхний управляющий слой приложения и
интерфейсные части), связанные при помощи утилиты \verb'SWIG'~\cite{SWIG}.

К настоящему моменту (первые версии появились в 2003 году, первая публикация \cite{racs:2007} в 2007 году) 
{\tt RACS} хорошо зарекомендовал себя при организации массовых расчетов в различных областях~--- сейсмике,
моделировании разработки керогеносодержащих месторождений с учетом внутрипластового горения,
моделировании магнитных систем и разработке устройств спинтроники, %физике плазмы,
газодинамике горения, изучении резонансных свойств нелинейных систем и т.д.
%Тем не менее, в процессе эксплуатации был обнаружен ряд недостатков, требующих существенной доработки системы.

\endinput

Целый ряд задач численного моделирования требует проведения больших объемов
однотипных серий расчетов~--- расчеты в серии независимы, и отличаются
лишь значением одного или нескольких параметров, и именно в этом в этом случае
{\tt RACS} оказывается наиболее эффективен. 
Кроме поиска в результатах расчетов, запущенный в клиент--серверном режиме {\tt RACS} обеспечивает автоматический
запуск расчетов на нескольких компьютерах в рамках кластера или локальной сети с разнородными версиями 
операционной системы.
Инструментальные средства {\tt Python} и {\tt RACS} позволяют реализовывать
консервацию и восстановление расчета для продолжения.


Изначально {\tt RACS} был построен по асинхронной схеме, без центрального сервера (такая архитектура
представлялась более надежной). Появившийся со временем сервер 
занимался лишь даигностикой и сбором статистики загруженности ресурсов.
Практика показала, что при интенсивных разнородных расчетах в рамках локальной сети или кластера 
такая архитектура не позволяет 
должным образом распределять ресурсы, что приводит к эпизодическим конфликтам. 

Интерфейс подключения {\tt RACS} к приложениям численного моделирования так же может быть существенно улучшен.
В настоящий момент подключение {\tt RACS} к уже готовому коду требует рутинной переработки кода, что неизбежно приводит
к ошибкам. Представляется возможным организовать подключение с минимальными изменениями отлаженного ранее кода.

С другой стороны, за время эксплуатации был накоплен большой опыт по организации массовых расчетов и 
постобработке результатов моделирования,  сформулированна соответствующая идеология. Разработанные подходы должны быть 
особенно эффективны при решении инженерных задач, требующих проведения больших объемов однотипных расчетов и комплексного 
анализа их результатов для выбора
оптимальной конфигурации устройства.
 % разжигая ваш аппетит
% быстрый старт (примеры использования)
% установка
% \input{structure}
\section{Сборка и установка библиотеки}
\subsection{Получение исходных кодов}
Для получения исходных кодов библиотеки \verb'aiwlib' удобнее всего
использовать утилиту \verb'git':
\begin{verbatim}
    git clone https://github.com/aivn/aiwlib
\end{verbatim}
при этом будет создана директория \verb'aiwlib' содержащая последнюю
версию исходных кодов. Если Вы хотите извлечь исходные коды в директорию с
другим именем \verb'my-aiwlib-specific-path' используйте команду
\begin{verbatim}
    git clone https://github.com/aivn/aiwlib my-aiwlib-specific-path
\end{verbatim}

В дальнейшем, для получения обновлений,
достаточно будет перейти в эту директорию и набрать
\begin{verbatim}
    git pull
\end{verbatim}

Если по каким то причинам использование \verb'git' невозможно,
перейдите по ссылке
\begin{verbatim}
    https://github.com/aivn/aiwlib
\end{verbatim}
нажмите на открывшей странице расположенную справа в центре зеленую
кнопку <<Clone or download>> и выберите в открывшемся окне пункт
<<Download ZIP>>. Однако при этом получение обновленных версий
библиотеки будет довольно неудобным~--- Вам придется каждый раз
получать новый архив и фактически собирать и устанавливать библиотеку заново. 

\subsection{Сборка библиотеки}
Для сборки библиотеки используется утилита \verb'GNU Make'.
Желательно использование компилятора \verb'g++' с версией не старше 4.8. 
Все пакеты и утилиты на языке \verb'Python' рассчитаны на
использование версии \verb'Python2.7' (хотя некоторые из них могут
работать и на более старших версиях), язык \verb'Python3' {\bf не} поддерживается. 

Процесс сборки организован нетрадиционным образом~---
отсутствует скрипт \verb'./configure'. Для настройки сборки необходимо
вручную отредактировать файл
\verb'include/aiwlib/config.mk',\footnote{
Предполагается, что пользователь занимающийся численным моделированием
обладает достаточной квалификацией для этого действия}
либо задать необходимые параметры сборки через аргументы командной
строки \verb'make'~--- во втором случае это придется делать при
каждой новой сборке.

Если Вы получили исходные коды \verb'aiwlib' через \verb'git', после
внесения изменений в файл \verb'include/aiwlib/config.mk'
рекомендуется создать новую ветку и переключится на нее, для этого выполните
команды
\begin{verbatim}
  git commit include/aiwlib/config.mk
  git branch my-custom-config
  git checkout my-custom-config
\end{verbatim}
это позволит не потерять настройки конфигурации после получения обновлений.

Рассмотрим подробно часть файла \verb'include/aiwlib/config.mk'
предназначенную для редактирования (приводятся значения по
умолчанию).

Фрагмент
\begin{verbatim}
PYTHONDIR=/usr/lib/python2.7
LIBDIR=/usr/lib
INCLUDEDIR=/usr/include
BINDIR=/usr/bin
BIN_LIST=racs approx isolines gplt uplt splt mplt fplt
\end{verbatim}
задает целевые пути и список утилит \verb'aiwlib' и играет роль только
при установке уже собранной библиотеки.

Фрагмент 
\begin{verbatim}
zlib=on
swig=on
png=on
pil=on
bin=on
ezz=on
# mpi=on
# mpi=off
\end{verbatim}
задает флаги определяющие список модулей (частей библиотеки) которые будут собираться
и использоваться в проектах. Если в системе отсутствуют необходимые для
сборки модуля зависимости\footnote{Например пакет {\tt zlib1g-dev}
  необходимый для сборки кода с поддержкой сжатых файлов} или
функциональность предоставляемая модулем избыточна, модуль может быть
исключен из сборки по умолчанию при комментировании
(раскомментировании для \verb'mpi') соответствующей
строки фрагмента, либо при помощи аргумента командной строки при
вызове \verb'make'.


\newpage
Фрагмент 
\begin{verbatim}
CXX:=g++
MPICXX:=mpiCC
SWIG:=swig

PYTHON_H_PATH:=/usr/include/python2.7
override CXXOPT:=$(CXXOPT) -std=c++11 -Wall -fopenmp -O3 -fPIC -g 
override MPICXXOPT:=$(MPICXXOPT) $(CXXOPT)
override LINKOPT:=$(LINKOPT) -lgomp 
override SWIGOPT:=$(SWIGOPT) -Wall -python -c++ 
\end{verbatim}
задает компиляторы, опции компиляции и линковки, и путь к файлу \verb'Python.h'
(актуален в случае \verb'swig=on'). Если Вы не знаете где находится
\verb'Python.h'
Вы можете воспользоваться командой
\begin{verbatim}
    locate Python.h
\end{verbatim}
Опции компиляции и линковки могут быть дополнены через аргументы
командной строки \verb'make'.

Содержимое библиотеки с точки зрения особенностей сборки можно условно разбить на следующие части:
\begin{itemize}
  \item ядро~--- модули на языке \verb'C++' в результате сборки дающие файл
    \verb'libaiw.a' и компонуемые с этим файлом утилиты на языке \verb'C++';
  \item модули для работы со структурами данных \verb'C++' из
    \verb'Python' требующие сборки и зависящие от утилиты \verb'swig' и пакета \verb'python-dev';
  \item модули для визуализации на языках \verb'C++', \verb'OpenGL' и
    \verb'Python' требующие сборки и зависящие от довольно
    значительного числа пакетов;
  \item модули и утилиты на языке \verb'Python' не требующие сборки
    (но возможно требующие установки).
\end{itemize}

\begin{table}
\begin{center}
\begin{tabular}{lll}
  \hline
  флаг & зависимости & функциональность \\
  \hline
  \verb'zlib=on' & \verb'zlib...-dev' & работа со сжатыми
  файлами \\[3mm]
  \verb'png=on' & \verb'libpng...-dev' & построение изображений в
  формате \verb'.png' \\[3mm]
  \verb'pil=on' & \verb'python-pil' & построение изображений для {\tt Python Imaging Library}, \\  
                & \verb'python-dev' & работает только при \verb'swig=on' \\[3mm]  
  \verb'mpi=off' &  & сборка с поддержкой \verb'MPI' всегда отключена \\[3mm]  
  \verb'mpi=on' & \verb'mpi...-bin' & для сборки всегда используется \verb'$(MPICXX)', как для  \\  
                & \verb'mpi...-dev' & {\tt aiwlib.a} так и для пользовательских проектов \\[3mm]  
  \verb'mpi='   & опционально       & для сборки некоторых частей {\tt aiwlib.a} используется  \\  
                & \verb'mpi...-bin' & \verb'$(MPICXX)' (при условии что такой компилятор есть, \\
                & \verb'mpi...-dev' & иначе поддержка \verb'MPI' отключается и используется \verb'$(CXX)'), \\
                &                   & при необходимости можно указывать \verb'mpi=on' для проектов \\
                &                   & пользователя  \\[3mm]
  \verb'bin=on' &                   & собирать по умолчанию бинарные утилиты \verb'aiwlib' \\
  \hline
\end{tabular}
\end{center}
\caption{Зависимости ядра, функциональность и влияние флагов сборки}\label{make:core:flags:table}
\end{table}

Зависимости ядра \verb'aiwlib', функциональность и влияние флагов сборки показаны  таблице~\ref{make:core:flags:table}.

Модули для работы со структурами данных \verb'C++' из \verb'Python'
требуют наличия утилиты \verb'swig' (версии не ниже 2.0) и пакета
\verb'python-dev'. Для сборки модулей необходимо установить флаг \verb'swig=on'.
%Более подробно сборка таких модулей описано в разделе~\ref{make:swig:sec}.!!!

Библиотека \verb'aiwlib' предоставляет целое семейство вьюверов:
\begin{itemize}
  \item \verb'gplt' --- построение графиков типографского качества на
    основе {\tt gnuplot};
  \item \verb'uplt' ---  построение срезов многомерных и
    сферических сеток;
  \item\verb'splt' --- визуализация поверхностей, заданных треугольными неструктурированными сетками;
  \item\verb'mplt' --- визуализация распределений магнитных моментов
    и векторных полей;
  \item\verb'fplt' --- воксельная визуализация трехмерных равномерных
    сеток.
\end{itemize}

Вьювер \verb'gplt' не требует сборки, но требует установки библиотеки (настройки
путей к другим питоновским модулям) и требует как минимум пакета
\verb'gnuplot'.
Для рисования графиков на экране требуются пакеты \verb'gnuplot-x11'
или \verb'gnuplot-qt'. Для построения графиков типографского качества
дополнительно необходим \LaTeX, в том числе приложения \verb'pdflatex'
и \verb'pdfcrop'.

Вьювер \verb'uplt' требует сборки \verb'aiwlib' с флагами
\verb'swig=on', \verb'pil=on', кроме того необходим пакет
\verb'python-tk'.

Вьюверы \verb'splt', \verb'mplt' и \verb'fplt' собираются при
установке флага \verb'ezz=on' и требуют пакеты
\verb'glm', \verb'glu', \verb'glew', \verb'python2-pillow', \verb'glut', \verb'python2-glut', \verb'python2-opengl', \verb'readline'.


\subsection{Установка библиотеки}
Для установки библиотеки используйте команды (из под \verb'root')
\begin{verbatim}
    make install
\end{verbatim}
для копирования собранных модулей и заголовочных файлов в системные директории, либо
\begin{verbatim}
    make install-links|links-install
\end{verbatim}
для создания мягких ссылок на собранные модули и заголовочные файлы в системных директориях.
Второй вариант предпочтительней с точки зрения простоты обновления библиотеки, 
но может иметь некоторые уязвимости с точки зрения безопасности
при работе в многопользовательском режиме~--- 
гипотетически пользователь установивший у себя библиотеку может подсунуть другим пользователям вредоносный код.

Предыдущая версия, библиотека {\tt aivlib}, имела два существенных недостатка~--- сложную установку и проблемы при
работе с несколькими версиями библиотеки на одной машине. Текущая версия допускает локальную установку
произвольного числа версий/копий библиотеки, более того это рекомендуется для  
упрощения переноса проектов на другие машины, где библиотека \verb'aiwlib' не установлена~---
в этом случае вместо копирования всей библиотоеки достаточно создать мягкую ссылку на библиотеку в директории проекта.

Таким образом Вы можете не устанавливать \verb'aiwlib' глобально в
систему~--- достаточно собрать библиотеку и использовать для своих
проектов систему сборки \verb'aiwlib' описанную ниже. При этом доступ
к утилитам командной строки \verb'aiwlib' и вьюверам (расположенным в директории
\verb'aiwlib/bin') Вам придется обеспечивать самостоятельно.  

\section{Сборка проектов с использованием aiwlib}
Для упрощения сборки проекта рекомендуется использовать файл \verb'include/aiwlib/user.mk'.
При этом пользовательский \verb'Makefile' должен иметь вид
\begin{verbatim}
name=NAME #имя проекта
headers=... #список хидеров обрабатываемых SWIG-ом
modules=... #список .cpp модулей проекта
sources=... #другие исходные файлы проекта

include aiwlib/user.mk
\end{verbatim}
или
\begin{verbatim}
include local-path-to-aiwlib/include/aiwlib/user.mk
\end{verbatim}
При этом заголовочные файлы библиотеки всегда включаются как \verb'<aiwlib/...>',
путь к локально установленной библиотеке при необходимости определяется автоматически на основе пути к файлу \verb'user.mk'.

По умолчанию, такой \verb'Makefile' собирает модуль для питона, и предоставляет еще ряд целей
\begin{itemize}
\item \verb'sources' --- выводит список исходных файлов проекта включая \verb'make'--файл, хидеры определяются автоматически 
при помощи вызова \verb'g++ -M' на основе переменной \verb'modules', заголовочные файлы библиотеки \verb'aiwlib' 
{\bf НЕ} включаются в список. Если при вызове \verb'make' указать опцию \verb'with=swig', в список будут включены файлы 
\verb'NAME.i', \verb'NAME.py' и \verb'NAME_wrap.cxx'
\item \verb'all_sources' --- выводит список всех исходных файлов проекта включая \verb'make'--файл, 
хидеры определяются автоматически 
при помощи вызова \verb'g++ -M' на основе переменной \verb'modules', заголовочные файлы библиотеки \verb'aiwlib' 
включаются в список {\bf ПРИ УСЛОВИИ ЕЕ ЛОКАЛЬНОЙ УСТАНОВКИ}. 
Если библиотека \verb'aiwlib' установлена локально и при вызове \verb'make' указана опция \verb'with=aiwlib', в список включаются  
необходимые файлы \verb'.py' и \verb'.i' библиотеки \verb'aiwlib'.
Если библиотека \verb'aiwlib' установлена локально и при вызове \verb'make' указана опция \verb'with=aiwlib,swig',
(последовательность не имеет значения) в список включаются
все файлы \verb'.py', \verb'.i' и \verb'_wrap.cxx' библиотеки \verb'aiwlib', а так же файлы 
\verb'NAME.i', \verb'NAME.py' и \verb'NAME_wrap.cxx'.
\item \verb'clean' --- удаляет все созданные объектные файлы и файл \verb'_NAME.so'.
\item \verb'cleanall' --- удаляет все созданные объектные и \verb'.so' модули, а так же файлы 
\verb'NAME.i', \verb'NAME.py' и \verb'NAME_wrap.cxx'.
\item \verb'NAME.tgz' \verb'NAME.md5' --- создает сжатый архив и файл с контрольной суммой {\bf несжатого} архива
со списком файлов  проекта \verb'sources', опция \verb'with' влияет на список файлов.
\item \verb'tar NAME-all.tgz' --- создает сжатый архив \verb'NAME-all.tgz'
со списком файлов  проекта \verb'all_sources', опция \verb'with' влияет на список файлов.
\item \verb'export to=...' --- переносит проект в другую директорию (если опция \verb'to' имеет вид \verb'to=PATH')
либо на другую машину (если опция \verb'to' имеет вид \verb'to=HOST:PATH'). Для переноса используется список 
файлов \verb'all_sources', опция \verb'with' влияет на список файлов. Перенос на другую машину осуществляется при помощи 
утилиты \verb'SSH', если у Вас не настроена авторизация по открытому ключу то пароль придется вводить трижды.
Директория \verb'PATH' создается при необходимости, включая родительские каталоги.
Корректный перенос файлов библиотеки \verb'aiwlib' возможен только при условии ее установки в поддиректорию проекта.
\end{itemize}
Таким образом, опция \verb'with=swig' необходима, если на целевой машине отсутствует утилита \verb'SWIG'. Опция \verb'with=aiwlib'
необходима, если библиотека \verb'aiwlib' была установлена локально в поддиректорию проекта.

В питоне, при импорте модулей библиотеки \verb'aiwlib' будут импортироваться модули из локальной копии библиотеки,
при условии что до этого был импортирован собранный пользовательский модуль. 

Если модуль для питона собрать не требуется, не указывайте переменную
\verb'name'. Если требуется собирать некоторое количество исполняемых
файлов из \verb'C++' кода, их можно перечислить в переменной
\verb'cxxmain' через пробел, указываются \verb'C++'--файлы содержащие
\verb'main()'--функции.
Для каждого такого файла будет собран исполняемый файл, при этом
производится линковка со всеми объектными файлами из переменных
\verb'modules' и  \verb'sources'.



% изменения по сравнению с aivlib ?

\chapter{Ядро библиотеки}
\section{Средства отладки --- модуль {\tt debug.hpp}}
\subsection{Общие замечания}
Модуль \verb'debug.hpp' предоставляет ряд макросов для вывода отладочной информации в процессе исполнения
(фактически удобную альтернативу традиционным отладочным \verb'printf') и генерации исключений:
\begin{itemize}
  \item\verb'WOUT(expressions...)' --- вывод информации в \verb'std::cout';
  \item\verb'WERR(expressions...)' --- вывод информации в \verb'std::cerr';
  \item\verb'WSTR(S, expressions...)' --- вывод информации в поток \verb'S', являющийся наследником \verb'std::ostream';
  \item\verb'WEXC(expressions...)' --- вывод информации со стека в \verb'std::cerr' при обработке исключения; 
  \item\verb'WASSERT(condition, message, expressions...)' --- вывод инфомации в \verb'std::cerr'
    при нарушении условия \verb'condition';
  \item\verb'AIW_WARNING(message, expressions...)' --- вывод информации в \verb'std::cerr';
  \item\verb'AIW_RAISE(message, expressions...)' --- вывод информации в \verb'std::cerr'
    и генерация исключения типа \verb'const char *' содержащего выведенную информацию.
\end{itemize}
Например
\begin{verbatim}
  int a; double b[3];
  ...
  WOUT(a, a*b[1], b[0]+b[2]);
\end{verbatim}

Все макросы выводят информацию в виде
\begin{verbatim}
    #filename function() LLL: expr1=value1, expr2=value2 ...
\end{verbatim}
или
\begin{verbatim}
    #filename function() LLL: message expr1=value1, expr2=value2 ...
\end{verbatim}
где \verb'LLL' --- номер строки в файле \verb'filename' в которой был сгенерирован вывод сообщения,
\verb'function()' --- имя функции в которой был сгенерирован вывод сообщения, \verb'expr'~--- выражение,
\verb'value'~--- значение выражения.

\subsection{Синтаксические ограничения}
В качестве выражений (аргументов макросов) могут использоваться любые \verb'rvalue' выражения,
для значений которых реализованы операторы
вывода в поток
\begin{verbatim}
    std::ostream& operator << (std::ostream&, expr_type)
\end{verbatim}

В выражениях могут присутствовать скобки \verb'()[]{}', операции \verb'<>' скобками {\bf не} считаются.
%если в выражении присутствует
%явное задание параметров шаблона, его необходимо брать в скобки, например
%\begin{verbatim}
%     std::complex<double>(a+b)   // неправильно
%     (std::complex<double>(a+b)) // правильно
%\end{verbatim}
Если в выражении есть запятые, части содержащие запятые так же должны быть в скобках, например
\begin{verbatim}
     pow(a, b)   
\end{verbatim}
В противном случае макросы сохраняют работоспособность, но вывод может иметь странный вид, например
\begin{verbatim}
template <int D, typename T> struct V{
	T p[D];
	V(T x){ for(int i=0; i<D; ++i) p[i] = x; }
};

template <int D, typename T> 
std::ostream& operator << (std::ostream& out, const V<D,T> &v){
	out<<"{"<<v.p[0];
	for(int i=1; i<D; ++i) out<<" "<<v.p[i];
	return out<<"}";
}
...
	int a=2;
	WOUT(V<3,int>(a), a*2);
\end{verbatim}
даст вывод
\begin{verbatim}
# ... : V<3={2 2 2}, int>(a), a*2=4
\end{verbatim}
вместо ожидаемого
\begin{verbatim}
# ... : V<3, int>(a)={2 2 2}, a*2=4
\end{verbatim}
Для корректного вывода необходимо использовать дополнительные скобки
\begin{verbatim}
	WOUT((V<3,int>(a)), a*2);
\end{verbatim}

В одной строке может использоваться только один макрос \verb'WEXC'.

\subsection{Режимы работы}
Макросы \verb'WOUT', \verb'WERR', \verb'WSTR', \verb'WEXC' и \verb'WASSERT' работают только если определен макрос \verb'EBUG'
(например при помощи опции компилятора \verb'-DEBUG'). При сборке на основе шаблонного \verb'aiwlib/Makefile'
макрос \verb'EBUG' по умолчанию отключен, для его подключения необходимо использовать команду
\begin{verbatim}
    make -DEBUG ...
\end{verbatim}

Отличие между вызовом 
\begin{verbatim}
    WASSERT(condition, ...)
\end{verbatim}
и
\begin{verbatim}
    if(!(condition)) AIW_RAISE(...)
\end{verbatim}
заключается в том, что при отключенном режиме отладке макрос \verb'WASSERT' игнорируется полностью (включая проверку условия).

Макросы \verb'AIW_WARNIG' и \verb'AIW_ASSERT' работают всегда, вне зависимости от макроса \verb'EBUG'.

\subsection{Вывод информации со стека при обработке исключения}
Макрос \verb'WEXC' выводит свои аргументы на стандартный поток ошибок \verb'std::cerr' при обработке исключения.
Типовой ситуацией является возбуждение исключения в \veb'C++' функции, вызываемой из \verb'Python', если
модуль был собран при помощи шаблонного \verb'aiwlib/Makefile' --- в этом случае в \verb'Python' происходит
вызов стандартного обработчика исключений.

Выводятся только аргументы макросов \verb'WEXC', размещенных на стеке {\bf до} возбуждения исключения.
Для каждого аргумента выводится значение, которое принимал аргумент в момент вызова макроса \verb'WEXC'.

В одной строке может использоваться только один макрос \verb'WEXC'~--- это связано с размещением в строке
экземпляра класса \verb'aiw::DebugStackFrame', с именем формируемым на основе номера строки.
При этом выводимое сообщение формируется при вызове макроса \verb'WEXC', хранится внутри экземпляра класса в виде
буфера \verb'std::stringstream', и выводится дестуктором экземпляра класса если возникла исключительная ситуация.
Наличие исключительной ситуации проверяется при помомщи фунцкии \verb'std::uncaught_exception()'.

Если включен режим отладки (определен макрос \verb'EBUG'), сообщения {\bf всех} макросов \verb'WEXC' формируются в виде строк,
но не выводятся пока не возникнет исключительная ситуация.
Это может отрицательно сказываться на производительности, поскольку создание каждого сообщения
требует форматированного вывода, выделения памяти в куче и т.д.

Если режим отладки выключен (макрос \verb'EBUG' не определен), макросы \verb'WEXC' игнорируются.


\subsection{Детали реализации}
Модуль \verb'debug.hpp' это легкий (около 50-ти строк), независимый от остальных частей библиотеки \verb'aiwlib' модуль.

Вывод выражений построен на рекурсивной функции
\begin{verbatim}
    template <typename ... Args> 
    void aiw::debug_out(std::ostream& out, const char* str, Args ... args);
\end{verbatim}
вызываемой из макросов, в качестве \verb'str' подставляются аргументы макроса в виде строки и затем еще раз
в виде аргументов (уже значений соответствующих выражений). 

Функция \verb'aiw::debug_out' разбирает \verb'str' по запятым, учитывая при этом скобки \verb'()[]{}'.

Про детали работы макроса \verb'WEXC' было сказано выше.

При выводе сообщений от всех макросов метод потока вывода \verb'flush' {\bf не} вызывается. 

Модуль \verb'debug.hpp' подключает и использует следующие стандартные библиотеки:
\begin{itemize}
  \item \verb'<iostream>' --- работа со стандартными потоками вывода;
  \item \verb'<sstream>' --- работа с потоком \verb'std::stringstream' в макросах \verb'WEXC';
  \item \verb'<exception>' --- определение наличия исключительной ситуации в деструкторе объекта \verb'aiw::DebugStackFrame'
    при помощи фунцкии \verb'std::uncaught_exception()'.
\end{itemize}


 

\section{Выделение и освобождение ресурсов  в контейнерах и потоках ввода/вывода --- модуль {\tt alloc}}
Для безопасного копирования экземпляров потоков ввода/вывода \verb'aiw::IOstream', контейнеров библиотеки
\verb'aiwlib' и освобождения ресурсов используются <<умные>> указатели \verb'std::shared_ptr'.

В заголовчном файле \verb'alloc' объявлен абстрактный класс \verb'aiw::BaseAlloc',
предоставляющий интерфейс для работы с выделенными ресурсом (областью памяти или мапированным файлом).
Класс имеет следующие методы
\begin{itemize}
\item \verb'void* get_addr()' --- возвращает адрес контролируемой области памяти;
\item \verb'size_t get_size() const'  --- возвращает размер контролируемой области памяти в байтах;
\item \verb'virtual ~BaseAlloc() = 0' --- полностью виртуальный деструктор;
\item \verb'virtual size_t get_sizeof() const = 0' --- возвращает размер элемента (ячейки массива) в байтах.
\end{itemize}

Класс \verb'template<T> aiw::MemAlloc' является наследником класса \verb'aiw::BaseAlloc',
и кроме перегрузки соответвующих методов предоставляет конструктор
\begin{verbatim}
    template<typename ... Args> MemAlloc(size_t sz, Args ... args)
\end{verbatim}
создающий в памяти массив размера \verb'sz' из элементов типа \verb'T' с аргументами конструктора \verb'args'.

Класс \verb'aiw::MMapAlloc' является наследником класса \verb'aiw::BaseAlloc',
и кроме перегрузки соответвующих методов предоставляет конструктор
\begin{verbatim}
    MMapAlloc(const std::shared_ptr<FILE> &pf, size_t size, int flags)
\end{verbatim}
мапирующий в память (с флагами \verb'flags') область размера \verb'size' байт из файла \verb'pf' от текущей позиции в файле.

Модуль \verb'alloc' является легким (около 60-ти строк) файлом, зависящими только от модуля \verb'debug'.
Модуль \verb'alloc' подключает и использует следующие библиотеки:
\begin{itemize}
\item \verb'aiwlib/debug' --- генерация исключений;
\item \verb'<memory>' --- доступ к классу \verb'std::shared_ptr';
\item \verb'<sys/mman.h>', \verb'<cunistd>' --- мапирование файлов.  
\end{itemize}

При кросс--компиляции под ОС \verb'Windows' компилятором \verb'minGW' необходимо указывать опцию \verb'-DMINGW',
при этом библиотеки \verb'<sys/mman.h>', \verb'<cunistd>' не подключаются и
класс \verb'MMapAlloc' является недоступным.


\section{Потоки ввода/вывода --- модули {\tt iosream}, {\tt gzstream} и {\tt binaryio}}
\subsection{Общие замечания}
При создании приложений численного моделирования потоки ввода/вывода \verb'std::iostream'
из стандартной библиотеки оказываются не всегда удобны. В частности желательно:
\begin{enumerate}
\item обеспечить максимально возможную производительность, особенно при бинарном вводе/выводе~---
  в этом смысле потоки \verb'std::iostream' сделаны не вполне оптимально;
\item иметь абстрактный базовый класс потока и его наследников для работы с обычными файлами и с файлами сжатыми
  библиотекой \verb'zlib.h'~--- такую возможность предоставляет например библиотека
  \verb'boost', но использование \verb'boost' только ради потоков предcтавляется черезмерным;
\item иметь возможность мапировать файл (стандартная функция \verb'mmap') при помощи метода потока,
  с текущей позиции, указав лишь размер области и режим, и обеспечивать при этом автоматическую сборку мусора;
\item иметь возможность применять для форматированного вывода типобезопасный аналог функций \verb'fprintf';
\item иметь возможность формировать имя файла в аргументах конструктора при помощи типобезопасного аналога функций \verb'fprintf';
\item использовать операторы для бинарного ввода/вывода --- впрочем эта возможность может быть реализована и для
  \verb'std::iostream'.
\end{enumerate}
Библиотека \verb'aiwlib' предоставляет свои потоки ввода/вывода~--- абстрактный класс \verb'aiw::IOstream'
и его наследников \verb'aiw::File' (модуль \verb'iostream') и \verb'aiw::GzFile' (модуль \verb'gsztream'). В модуле \verb'binaryio'
перегружены операции \verb'<' и \verb'>' для бинарного ввода/вывода для большинства актуальных типов.

\subsection{Типобезопесный форматированный вывод}
Модуль \verb'iostream' предоставляет функцию
\begin{verbatim}
    template <typename S, typename ... Args>
    void aiw::format2stream(S &&str, const char *format, Args ... args);
\end{verbatim}
обеспечивающую типобезопасный форматированный вывод в поток \verb'str' согласно строке \verb'format'.
Аргументы \verb'args' подставляются вместо символов \verb'%'.
Для вывода символа \verb'%' необходимо использовать строку \verb'%%'.
Может выводится любой аргумент \verb'x' для которого определен оператор
форматированного вывода \verb'str<<x'.

\subsection{Абстрактный класс {\tt aiw::IOstream}}
Абстрактный класс \verb'aiw::IOstream' определен в заголовочном файле \verb'iostream'.

Класс \verb'aiw::IOstream' имеет поле \verb'name', содержащее имя открытого файла.

Класс предоставляет следующие методы:
\begin{itemize}
\item \verb'virtual ~IOstream(){}' --- виртуальный деструктор;
\item \verb'virtual void close() = 0' --- закрывает поток;
\item \verb'virtual size_t tell() const = 0' --- возвращает текущую позицию в потоке;
\item \verb'virtual void seek(size_t offset, int whence=0) = 0' --- устанавливает позицию в потоке относительно
  точки укзаываемой параметром \verb'whence', допустимые значения: 0 (\verb'SEEK_SET')~--- начало файла,
  1 (\verb'SEEK_CUR')~--- текущая позиция, 2 (\verb'SEEK_END')~--- конец файла;
\item \verb'virtual size_t read(void* buf, size_t size) = 0' --- читает \verb'size' байт в буфер \verb'buf' из файла,
  возвращает число прочитанных байт;
\item \verb'virtual size_t write(const void* buf, size_t size) = 0' --- записывает в файл \verb'size' байт из буфера \verb'buf',
  возвращает число записанных байт;
\item \verb'virtual void flush() = 0' --- принудительно сбрасывает содержимое буфера на диск;
\item \verb'virtual std::shared_ptr<BaseAlloc> mmap(size_t size, bool write_mode=false)' --- мапирует из файла область
  размерами \verb'size' (начиная с текущей позиции), возвращает \verb'proxy'--объект (см. описание модуля \verb'alloc'),
  если мапирование невозможно (например при работе со сжатым файлом) происходит копирование
  соответствующей области в память, при этом мапирование с доступом на запись невозможно;
\item \verb'virtual int printf(const char * format, ...) = 0' --- обеспечивает
  {\bf не}типобезобасный форматированный вывод при помощи фунцкии \verb'::fprintf()';
\item \verb'template <typename ... Args> IOstream& operator ()(const char *format, Args ... args)' --- обеспечивает
  типобезопасный форматированный вывод на основе функции  \verb'aiw::format2stream()';
\item операторы форматированного вывода \verb'<' для встроенных типов.
\end{itemize}


\subsection{Класс {\tt aiw::File}}
Абстрактный класс \verb'aiw::File' определен в заголовочном файле \verb'iostream'.

Класс \verb'aiw::File' является наследником класса \verb'aiw::IOstream'. Кроме перегрузки необходимых
виртуальных методов класса \verb'aiw::IOstream', класс \verb'aiw::File' предоставляет следующие методы:
\begin{itemize}
\item \verb'File(){}' --- конструктор по умолчанию, создает неактивный поток;
\item \verb'template <typename ... Args> open(const char *format, const char *mode, Args ... args)'~---
  открывает файлу в режиме \verb'mode' с именем, формируемым на основе строки \verb'format' и аргументов \verb'args'
  при помощи функции  \verb'aiw::format2stream()';
\item \verb'template <typename ... Args> File(const char *format, const char *mode, Args ... args)'~---
  конструктор, открывает файл при помощи описанного выше метода \verb'open'.
\end{itemize}

\subsection{Класс {\tt aiw::GzFile}}
Класс \verb'aiw::GzFile' определен в заголовочном файле \verb'gzstream'.

Класс \verb'aiw::GzFile' является наследником класса \verb'aiw::IOstream'. Кроме перегрузки необходимых
виртуальных методов класса \verb'aiw::IOstream', класс \verb'aiw::GzFile' предоставляет следующие методы:
\begin{itemize}
\item \verb'GzFile(){}' --- конструктор по умолчанию, создает неактивный поток;
\item \verb'template <typename ... Args> open(const char *format, const char *mode, Args ... args)'~---
  открывает файлу в режиме \verb'mode' с именем, формируемым на основе строки \verb'format' и аргументов \verb'args'
  при помощи функции  \verb'aiw::format2stream()';
\item \verb'template <typename ... Args> GzFile(const char *format, const char *mode, Args ... args)'~---
  конструктор, открывает файл при помощи описанного выше метода \verb'open'.
\end{itemize}

\subsection{Операторы бинарного ввода/вывода~--- модуль {\tt binaryio}}
Модуль \verb'binaryio' предоставляет перегруженные операции \verb'<' (для бинарного вывода) и \verb'>' (для бинарного ввода)
в потоки \verb'aiw::IOstream'.

Встроенные типы, \verb'std::complex<T>' и вектора \verb'aiw::Vec' обрабатываются обычным копированием.

Для типов \verb'std::vector', \verb'std::string', \verb'std::list', \verb'std::map' сначала записывается размер контейнера
(тип \verb'uint32_t' для строк и \verb'uint64_t' для остальных), затем содержимое контейнера.

Во избежании подключения лишних модулей, операции \verb'<' и \verb'>' в модуле \verb'binaryio'
для {\bf не}встроенных типов перегружаются только если в единице трансляции был подключен
заголовочный файл с определением соотвествующего типа {\bf до} заголовочного файла \verb'binaryio'.

Например, для перегрузки операций \verb'<' и \verb'>' для комплексных чисел \verb'std::complex<T>',
заголовочный файл \verb'binaryio' должен быть включен {\bf после} заголовочного файла \verb'complex'.

Допускается многократное включение заголовочного файла \verb'binaryio', при этом можно считать что с точки
зрения перегрузки операций \verb'<' и \verb'>' актуальным является
последнее включение.

\subsection{Детали реализации}
Модули \verb'iostream' (порядка 100 строк), \verb'gzstream' (40 строк) и \verb'binaryio' (порядка 100 строк)
являются довольно легкими модулями, зависящими только от модулей \verb'debug' и \verb'alloc'.

Модуль \verb'iostream' подключает и использует следующие библиотеки:
\begin{itemize}
\item \verb'aiwlib/debug' --- генерация исключений;
\item \verb'aiwlib/alloc' --- доcтуп к объекту \verb'MMapAlloc' при мапировании файлов;
\item стандартная библиотека \verb'<cstdio>' --- работа с файлами \verb'FILE*';
\item стандартная библиотека \verb'<string>' --- доступ к классу \verb'std::string'.
\end{itemize}

Модуль \verb'gzstream' подключает и использует следующие библиотеки:
\begin{itemize}
\item \verb'aiwlib/iostream' --- доступ к абстрактному классу \verb'aiw::IOstream';
\item стандартная библиотека \verb'<zlib.h>' --- работа со сжатыми файлами \verb'gzFile'.
\end{itemize}

Модуль \verb'binaryio' подключает и использует следующие библиотеки:
\begin{itemize}
\item \verb'aiwlib/iostream' --- доступ к абстрактному классу \verb'aiw::IOstream'.
\end{itemize}

% вектора
% сетки
% сфера
% инстацирование в питон - размазать по предыдыщуим модулям?

\chapter{Средства визуализации}
%3.1 gplt
%3.2 arr2D
%3.3 arr3D
%3.4 сфера
%3.5 вьювер для поверхностей
%3.6 вьювер для магнетиков

\chapter{RACS}

\chapter{Система кодогенерации SYMBALG}

\chapter{Дополнительные модули для питона}

\chapter{Алгоритмы}
\subsection{Построение изолиний}
\subsection{Интерполяция}

\end{document}
%\input{aivlib/installation}

\chapter{Ядро библиотеки}
\section{Введение}
\section{Введение}

Интерфейс приложения численного моделирования должен позволять
легко изменять параметры задачи (число которых иногда доходит до сотен или даже тысяч),
выбирать тот или иной алгоритм (в том числе разлиные варианты начальных и граничных условий),
обеспечивать анализ и визуализацию
результатов. Практика показала, что для сложных задач оптимальным
являеться не оконный интерфейс, а интерфейс командной
строки. Фактически речь идет о использовании собственного (или уже
существующего) высокоуровневого интерпретируемого языка,
адаптированного к специфике задачи.

При проведении массовых расчетов (например при анализе зависимости поведения устройства от 
нескольких параметров и построении фазовых диаграмм) требуется механизм, обеспечивающий
многократный автоматический запуск приложения с меняющимися заданным образом параметрами, 
желательно с контролем распределения ресурсов в рамках локальной сети или на кластере.

Для каждого расчета полученные зависимости должны сопровождаться
информацией о использованных параметрах расчета и алгоритмах. Если для
сохранения параметров существует большое количество методик и
библиотек, то сохранение алгоритмов является проблемой, и единственным
приемлемым решением на сегодняшний день является сохранение исходного
кода приложения.

Для анализа результатов необходим многопараметрический поиск по
проведенным расчетам, для чего результаты расчетов должны храниться
специальным, упорядоченным образом. Необходимо обеспечить возможность
поиска по версиям исходного кода. Эту проблему можно решать в ручную,
например размещая результаты расчетов на хорошо структурированном дереве
каталогов~--- однако такой подход требует строгой внутренней культуры пользователя, и
усложняется тем, что в процессе расчетов критерии упорядоченности могут
расширятся и изменяться кардинальным образом.   

При массовых расчетах аккуратное решение вышеописанных проблем может отнимать значительное время и силы. 
В разных рабочих группах
разработаны собственные библиотеки, позволяющие упростить процесс
написания окружения, но единый подход до сих пор не выработан.

Описанная в данной главе система {\tt RACS} ({\tt Results \& Algorithms Control System}~--- система контроля
результатов и алгоритмов) обеспечивает:
\begin{itemize}
\item задание параметров расчетов при запуске для приложений на языках \verb'Python' и \verb'C++';
\item автоматическое сохранение параметров и исходных кодов расчетов;
\item пакетный запуск расчетов (циклы по значениям параметров) и балансировка загрузки, как на локальных машинах так и на кластерах под \verb'MPI';
\item работа с контрольными точками для приложений \verb'C++', в том числе кластерах под \verb'MPI' ({\it в разработке});
\item развитые средства для многопараметрического поиска, анализа и обработки результатов.
\end{itemize}

При разработке \verb'RACS' делались следующие акценты:
\begin{itemize}
\item простота подключения (требуется минимальная модификация отлаженного кода);
\item лаконичный и интуитивно понятный синтаксис при запуске расчетов;
\item возможность обработки результатов средствами операционной системы и сторонними утилитами без потери целостности данных;
\item интеграция с другими утилитами~--- вывод данных в формате \verb'gnuplot' с заголовками \verb'gplt',
  чтение метаинформации о расчетах другими утилитами.
\end{itemize}


Даже для низкоквалифицированного
пользователя  {\tt RACS} автоматически обеспечивает необходимый минимум
<<культуры>> проведения расчетов (сохранение исходных кодов  и
параметров).
В результате пользователь имеет
возможность полностью сконцентрироваться на работе непосредственно  над задачей.

\verb'RACS' написан на языке \verb'Python' и ориентирован в первую очередь
на приложения написанные на
языках \verb'C++' (высокопроизводительное вычислительное ядро) и \verb'Python' (верхний управляющий слой приложения и
интерфейсные части), связанные при помощи утилиты \verb'SWIG'~\cite{SWIG}.

К настоящему моменту (первые версии появились в 2003 году, первая публикация \cite{racs:2007} в 2007 году) 
{\tt RACS} хорошо зарекомендовал себя при организации массовых расчетов в различных областях~--- сейсмике,
моделировании разработки керогеносодержащих месторождений с учетом внутрипластового горения,
моделировании магнитных систем и разработке устройств спинтроники, %физике плазмы,
газодинамике горения, изучении резонансных свойств нелинейных систем и т.д.
%Тем не менее, в процессе эксплуатации был обнаружен ряд недостатков, требующих существенной доработки системы.

\endinput

Целый ряд задач численного моделирования требует проведения больших объемов
однотипных серий расчетов~--- расчеты в серии независимы, и отличаются
лишь значением одного или нескольких параметров, и именно в этом в этом случае
{\tt RACS} оказывается наиболее эффективен. 
Кроме поиска в результатах расчетов, запущенный в клиент--серверном режиме {\tt RACS} обеспечивает автоматический
запуск расчетов на нескольких компьютерах в рамках кластера или локальной сети с разнородными версиями 
операционной системы.
Инструментальные средства {\tt Python} и {\tt RACS} позволяют реализовывать
консервацию и восстановление расчета для продолжения.


Изначально {\tt RACS} был построен по асинхронной схеме, без центрального сервера (такая архитектура
представлялась более надежной). Появившийся со временем сервер 
занимался лишь даигностикой и сбором статистики загруженности ресурсов.
Практика показала, что при интенсивных разнородных расчетах в рамках локальной сети или кластера 
такая архитектура не позволяет 
должным образом распределять ресурсы, что приводит к эпизодическим конфликтам. 

Интерфейс подключения {\tt RACS} к приложениям численного моделирования так же может быть существенно улучшен.
В настоящий момент подключение {\tt RACS} к уже готовому коду требует рутинной переработки кода, что неизбежно приводит
к ошибкам. Представляется возможным организовать подключение с минимальными изменениями отлаженного ранее кода.

С другой стороны, за время эксплуатации был накоплен большой опыт по организации массовых расчетов и 
постобработке результатов моделирования,  сформулированна соответствующая идеология. Разработанные подходы должны быть 
особенно эффективны при решении инженерных задач, требующих проведения больших объемов однотипных расчетов и комплексного 
анализа их результатов для выбора
оптимальной конфигурации устройства.

\section{Общие модули}
Для всех классов и функций библиотеки {\tt aivlib} необходимы потоки ввода/вывода, индексы и вектора~--- это
то <<основание>>, на котором строятся все остальные библиотечные классы и функции.
\input{aivlib/mystream}
\input{../tests/mystream}

\input{aivlib/IndxVctr}
\input{../tests/IndxVctr}

\input{aivlib/dekart}

\section{Контейнеры}
\input{aivlib/array}
\small
\input{../tests/Arr2D}
\input{../tests/Arr3D}
\normalsize
\input{aivlib/sphere}
\input{../tests/sphereT}


\section{Классы и функции для построения изображений}
%\input{aivlib/images}
%\input{tests/images}
\begin{center}
\framebox{************* раздел находится в разработке ****************}
\end{center}

\section{Утилиты командной строки}
\input{aivlib/shellutils}
\input{aivlib/viewers}


\chapter{Система контроля результатов и алгоритмов RACS}
\section{Введение}

Интерфейс приложения численного моделирования должен позволять
легко изменять параметры задачи (число которых иногда доходит до сотен или даже тысяч),
выбирать тот или иной алгоритм (в том числе разлиные варианты начальных и граничных условий),
обеспечивать анализ и визуализацию
результатов. Практика показала, что для сложных задач оптимальным
являеться не оконный интерфейс, а интерфейс командной
строки. Фактически речь идет о использовании собственного (или уже
существующего) высокоуровневого интерпретируемого языка,
адаптированного к специфике задачи.

При проведении массовых расчетов (например при анализе зависимости поведения устройства от 
нескольких параметров и построении фазовых диаграмм) требуется механизм, обеспечивающий
многократный автоматический запуск приложения с меняющимися заданным образом параметрами, 
желательно с контролем распределения ресурсов в рамках локальной сети или на кластере.

Для каждого расчета полученные зависимости должны сопровождаться
информацией о использованных параметрах расчета и алгоритмах. Если для
сохранения параметров существует большое количество методик и
библиотек, то сохранение алгоритмов является проблемой, и единственным
приемлемым решением на сегодняшний день является сохранение исходного
кода приложения.

Для анализа результатов необходим многопараметрический поиск по
проведенным расчетам, для чего результаты расчетов должны храниться
специальным, упорядоченным образом. Необходимо обеспечить возможность
поиска по версиям исходного кода. Эту проблему можно решать в ручную,
например размещая результаты расчетов на хорошо структурированном дереве
каталогов~--- однако такой подход требует строгой внутренней культуры пользователя, и
усложняется тем, что в процессе расчетов критерии упорядоченности могут
расширятся и изменяться кардинальным образом.   

При массовых расчетах аккуратное решение вышеописанных проблем может отнимать значительное время и силы. 
В разных рабочих группах
разработаны собственные библиотеки, позволяющие упростить процесс
написания окружения, но единый подход до сих пор не выработан.

Описанная в данной главе система {\tt RACS} ({\tt Results \& Algorithms Control System}~--- система контроля
результатов и алгоритмов) обеспечивает:
\begin{itemize}
\item задание параметров расчетов при запуске для приложений на языках \verb'Python' и \verb'C++';
\item автоматическое сохранение параметров и исходных кодов расчетов;
\item пакетный запуск расчетов (циклы по значениям параметров) и балансировка загрузки, как на локальных машинах так и на кластерах под \verb'MPI';
\item работа с контрольными точками для приложений \verb'C++', в том числе кластерах под \verb'MPI' ({\it в разработке});
\item развитые средства для многопараметрического поиска, анализа и обработки результатов.
\end{itemize}

При разработке \verb'RACS' делались следующие акценты:
\begin{itemize}
\item простота подключения (требуется минимальная модификация отлаженного кода);
\item лаконичный и интуитивно понятный синтаксис при запуске расчетов;
\item возможность обработки результатов средствами операционной системы и сторонними утилитами без потери целостности данных;
\item интеграция с другими утилитами~--- вывод данных в формате \verb'gnuplot' с заголовками \verb'gplt',
  чтение метаинформации о расчетах другими утилитами.
\end{itemize}


Даже для низкоквалифицированного
пользователя  {\tt RACS} автоматически обеспечивает необходимый минимум
<<культуры>> проведения расчетов (сохранение исходных кодов  и
параметров).
В результате пользователь имеет
возможность полностью сконцентрироваться на работе непосредственно  над задачей.

\verb'RACS' написан на языке \verb'Python' и ориентирован в первую очередь
на приложения написанные на
языках \verb'C++' (высокопроизводительное вычислительное ядро) и \verb'Python' (верхний управляющий слой приложения и
интерфейсные части), связанные при помощи утилиты \verb'SWIG'~\cite{SWIG}.

К настоящему моменту (первые версии появились в 2003 году, первая публикация \cite{racs:2007} в 2007 году) 
{\tt RACS} хорошо зарекомендовал себя при организации массовых расчетов в различных областях~--- сейсмике,
моделировании разработки керогеносодержащих месторождений с учетом внутрипластового горения,
моделировании магнитных систем и разработке устройств спинтроники, %физике плазмы,
газодинамике горения, изучении резонансных свойств нелинейных систем и т.д.
%Тем не менее, в процессе эксплуатации был обнаружен ряд недостатков, требующих существенной доработки системы.

\endinput

Целый ряд задач численного моделирования требует проведения больших объемов
однотипных серий расчетов~--- расчеты в серии независимы, и отличаются
лишь значением одного или нескольких параметров, и именно в этом в этом случае
{\tt RACS} оказывается наиболее эффективен. 
Кроме поиска в результатах расчетов, запущенный в клиент--серверном режиме {\tt RACS} обеспечивает автоматический
запуск расчетов на нескольких компьютерах в рамках кластера или локальной сети с разнородными версиями 
операционной системы.
Инструментальные средства {\tt Python} и {\tt RACS} позволяют реализовывать
консервацию и восстановление расчета для продолжения.


Изначально {\tt RACS} был построен по асинхронной схеме, без центрального сервера (такая архитектура
представлялась более надежной). Появившийся со временем сервер 
занимался лишь даигностикой и сбором статистики загруженности ресурсов.
Практика показала, что при интенсивных разнородных расчетах в рамках локальной сети или кластера 
такая архитектура не позволяет 
должным образом распределять ресурсы, что приводит к эпизодическим конфликтам. 

Интерфейс подключения {\tt RACS} к приложениям численного моделирования так же может быть существенно улучшен.
В настоящий момент подключение {\tt RACS} к уже готовому коду требует рутинной переработки кода, что неизбежно приводит
к ошибкам. Представляется возможным организовать подключение с минимальными изменениями отлаженного ранее кода.

С другой стороны, за время эксплуатации был накоплен большой опыт по организации массовых расчетов и 
постобработке результатов моделирования,  сформулированна соответствующая идеология. Разработанные подходы должны быть 
особенно эффективны при решении инженерных задач, требующих проведения больших объемов однотипных расчетов и комплексного 
анализа их результатов для выбора
оптимальной конфигурации устройства.

\input{racs/formalism}
\input{racs/general}
\input{racs/pickle_db}
\section{Анализ и обработка результатов~--- утилита командной строки {\tt racs}}



\input{racs/sh_utils}
\input{racs/grid}

\section{Модуль {\sf mixt.py} --- набор служебных функций и классов}
Модуль {\sf mixt.py} содержит ряд функций, использующихся остальными модулями
библиотеки {\sf raclib}.

\subsection{Различные системные функции}
Функция \verb'except_report( stderr=sys.stderr )' выводит отчет о последней
ошибке (возбужденном исключении) включая стек в файловый объект {\sf stderr} и возвращает
список строк отчета об ошибке виде результата. 

Функция \verb'get_checksums( Lf )' вовращает список контрольных сумм для
файлов из списка {\sf Lf}, контрольные суммы считаются при помощи утилиты {\sf
md5sum}.

Функции \verb'time2string( t, precision=3 )' и \verb'string2time( S )'
преобразуют число секунд {\sf t} в строку {\sf S} вида \verb|'hours:mm:sec'| и
обратно. Число знаков после десятичной точки при отображении секунд задается
аргументом {\sf precision}. Функции устаревшие, рекомендуется использовать класс {\sf mytime.Time}.

Функции \verb'date2string( d )' и \verb'string2date( S )'
преобразуют дату {\sf d} (число секунд  с начала эпохи {\sf Unix}) в строку {\sf S} вида \verb|'YYYY:MM:DD-hh:mm:sec'| и
обратно. Функции устаревшие, рекомендуется использовать класс {\sf mytime.Date}.

Функция \verb'size2string( sz )' преобразует размер {\sf sz} в байтах в строку
вида '$XXX${\sf K}' либо '$XXX.X${\sf M}' либо '$XXX.X${\sf G}' либо '$XXX.X${\sf T}'.

Функция \verb'GetLogin()' возвращает имя пользователя, пытаясь сначала вызвать
функцию \verb'os.getlogin()' а в случае ошибки возвращает занчение переменной
окружения {\sf USER}


Функция \verb'compare( name, patterns )' проверяет  при помощи функции {\sf
  fnmatch.fnmatch(...)} \cite{GVR} соответствует ли строка {\sf name} одному из
шаблонов в списке {\sf patterns}. Шаблоны могут иметь стандартный вид {\sf
  shell}, т.е. включать символы \verb|'*'|, \verb|'?'| и т.д. Если найдено соотвествие
хотя бы одному из шаблонов  возвращается {\sf True}, иначе возвращается {\sf False}.

Функция \verb'str_len( S )' возвращает длину строки \verb'S' в символах (при печати),
учитывая что символы {\sf utf-8} занимают два байта в памяти и один символ на
печати.

Функция \verb'get_tty_width()' возвращает ширину текущего терминала, запуская
в {\sf shell} команду \verb'stty size' и анализируя ее вывод.

Функция
\verb'table2strlist( LL, pattern=None, s_line=1, s_empty=2, s_bound=1, max_len=None )'
возвращает список форматированных строк (без символа конца строки
\verb|'\n'|). Аргумент \verb'LL' это таблица (список списков), для включения разделителя
(горизонтальной линии) необходимо вставить в \verb'LL' значение {\sf
  None}, все остальные элементы \verb'LL' должны быть списками или кортежами
одинаковой длины и содержать величины, которые при выводе будут
преобразовываться к строке при помощи функции {\sf str()}. Аргумент
\verb'pattern' задает паттерн аналогичный заголовку таблиц \LaTeX, т.е. может
содержать символы \verb|'r'|, \verb|'c'|, \verb|'l'| для обозначения
выравнивания содержимого колонки или символ \verb-'|'- для задания
вертикальной линии между колонками. Аргументы \verb's_line', \verb's_empty',
\verb's_bound' задают число пробелоов между текстом и вертикальными
разделительными линиями, между колонками без вертикальных разделительных линия
линий и между колонками и краями (если по краям нет линий). Аргумент
\verb'max_len' задает максимальную длину строк в выводимом списке (по
умолчанию без ограничений), лишние символы отбрасываются.



\subsection{Создание уникальных директорий расчета}
Функция  \verb'make_unique_path( base_path, num=3 )' генерирует уникальное имя
директории, добавляя к \verb'base_path' необходимое минимальное число
дополненное с начала нулями до длины {\sf num}. Например, если в директории
\verb'mypath/' есть поддиректория (или файл) с именем \verb'a023' то вызов 
\verb|make_unique_path( 'mypath/a' )| вернет строку \verb|'mypath/a024'|.

Функция \verb'close_path( p )' добавляет в конец пути {\sf p} символ
\verb|'/'| если {\sf p} не заканчивается этим символом.

Функция \verb'make_path( repository )' создает уникальную директорию в
репозитории \verb"repository" (в том числе и сам репозиторий при необходимости) и возвращает путь к ней.
Название директории формируется из
года, номера недели, дня недели и некоторого трехзначного порядкового номера
уникального для данной даты~--- таким образом расчеты внутри
репозитория автоматически
упорядочиваются по дате (например {\sf
  c07\_00\_1005}~--- пятый расчет проведенный первого января 2007
года). 
В рабочем каталоге (из которого была вызвана функция \verb'make_path') на директорию создается
символическая ссылка \verb|'_'| (одиночный символ подчеркивания).


\subsection{Класс {\sf progressbar} для интерактивного отображения степени
  выполнения вычислений}
Класс {\sf progressbar} обеспечивает интерактивное отображение степени
выполнения вычислений в стандатрном выводе или в стандартном потоке ошибок,
автоматически экстраполируя время завершения вычислений. Конструктор класса
принимает единственный аргумент~--- стандартный поток вывода {\sf sys.stdout} (по умолчанию)
или стандартный поток ошибок {\sf sys.stderr}. 

Класс имеет следующие методы:
\begin{itemize}
\item \verb'clean()' --- сбрасывает состояние объекта для отображения степени
  выполнения нового процесса;
\item \verb|out( progress, prompt='' )| --- отображает степень выполнения {\sf
progress} (число от нуля до единицы) и выводит некоторую дополнительную
  информацию {\sf prompt};
\item \verb|close( prompt='', result='OK' )| --- завершает отображение степени
  выполнения,  выводит  некоторую дополнительную
  информацию {\sf prompt} и результат выполнения {\sf result}. 
\end{itemize}

Отображение производится в виде строки 
\begin{verbatim}
<PROMPT> <RUNTIME> from <TOTALTIME> [###          ]
\end{verbatim}
где \verb'<RUNTIME>' --- время прошедшее с начала выполнения отображаемого
процесса, \verb'<TOTALTIME>'~--- оценка общего времени выполнения процесса
(может быть весьма неточной), строка \verb|'[###          ]'| отображает степень
выполнения. Общая длина строки всегда равняется ширине терминала (определяется
при помощи функции \verb'get_tty_width()'), поэтому не следует использовать слишком
длинные варианты {\sf prompt}.

Если поток, в который экземпляр класса {\tt progressbar} производит вывод, был перенаправлен в файл 
(или изначально являлся обычным файлом и не был связан с терминалом), вывод производится только методом {\tt close()}.

\subsection{Класс {\sf reg} --- задание интервалов значений для сравнения}

Модуль содержит определение глобальных величин
\begin{verbatim}
    inf, nan = float('inf'), float('nan')
\end{verbatim}
для операций сравнения.

Класс {\sf reg} предназначен для задания областей (интервалов) значений
произвольного типа для сравнения. Для класса определены следующие операторы:
\begin{center}
\begin{tabular}{rclcrcl}
\verb'reg(a,b)' &$\to$& $[a,b]$ &\rule{1cm}{0pt}&
\verb'x in reg(a,b)' &$\to$& $x\in [a,b]$ \\
\verb'reg(a,b) < x' &$\to$& $x>b$ && 
\verb'reg(a,b) <= x' &$\to$& $x\ge b$ \\
\verb'x < reg(a,b)' &$\to$& $x<a$ && 
\verb'x <= reg(a,b)' &$\to$& $x\le a$ \\
\verb'reg(a,b) > x' &$\to$& $x<a$ &&
\verb'reg(a,b) >= x' &$\to$& $x\le a$ \\
\verb'x > reg(a,b)' &$\to$& $x>b$ &&
\verb'x >= reg(a,b)' &$\to$& $x\ge b$ \\
\verb'reg(a,b) == x' &$\to$& $x \in [a,b]$ &&
\verb'reg(a,b) != x' &$\to$& $x \notin [a,b] $ \\
\verb'x == reg(a,b)' &$\to$& $x \in [a,b]$ &&
\verb'x != reg(a,b)' &$\to$& $x \notin [a,b] $ \\
\verb'reg(a,b) + X' &$\to$& $[a,b] \cup X$ &&
\verb'X + reg(a,b)' &$\to$& $X \cup [a,b]$ \\
\verb'reg(a,b)*x' &$\to$& $[ax,bx]$ &&
\verb'x*reg(a,b)' &$\to$& $[xa,xb]$ \\
\verb'reg(a,b)/x' &$\to$& $[a/x,b/x]$ &&
\verb'-reg(a,b)' &$\to$& $[b,a]$ \\
\end{tabular}
\end{center}
где $a, b, x$~--- некоторые значения допускающие сравнения, $X$~--- экземпляр
класса {\sf reg} или некоторое значение допускающее сравнение. Оператор суммы
возвращает кортеж из региона и добавленного значения, поэтому операторы
сравнения для него не работают, но работает оператор {\sf 'in'}.

\input{racs/mytime}
\input{racs/examples}


\chapter{Различные пакеты и утилиты}
\section{Модуль {\tt bindopt}~--- разбор аргументов командной строки в приложениях
{\tt Python}}\label{igl:sec}
\input{other/bindopt}

\section{Модуль {\tt myTkinter} --- упрощенное создание оконных интерфейсов}
\input{other/myTkinter}

\section{Утилита {\tt convolve}}\label{convolve:sec}
\input{other/convolve}
\clearpage

\section{Утилита {\tt gplt} --- упрощенное создание графиков типографского качества}\label{gplt:sec}
\input{other/gplt}

\chapter{Система кодогенерации SYMBALG}
%\input{symbalg/symbalg}
\begin{center}
\framebox{************* раздел находится в разработке ****************}
\end{center}

\begin{thebibliography}{99}
\bibitem{GVR}  Г. Россум, Ф.Л.Дж. Дрейк, Д.С. Откидач. <<Язык программирования
  {\tt Python}>>. 2001~--- 454C.
\bibitem{ML} Марк Лутц. Программирование на {\tt Python}. С-Пб.:
  <<Символ>>. 2002~--- 1135C. 
\bibitem{MAKE} Ричард Столлман, Роланд МакГрат. <<{\tt GNU Make}. Программа
  управления компиляцией>> 1995.
\bibitem{aiv:cpp2py} \href{http://a-iv.ry/pyart/cpp2py.pdf}
{А.В. Иванов <<Импорт {\tt С++} кода  в {\tt Python} при помощи пакета {\bfseries\tt SWIG}>>}
\bibitem{bash:conspect} 
\href{http://www.linux.org.ru/books/bash-conspect.html}{Григорий Строкин <<BASH конспект>> 1997.}
\bibitem{bash:scripting} Mendel Cooper <<Advanced Bash-Scripting Guide>>, перевод: Андрей
  Киселев
  \href{http://www.opennet.ru/docs/RUS/bash\_scripting\_guide/}{<<Искусство программирования на языке сценариев командной оболочки>>}
\end{thebibliography}

%\input{../tests/indexD}

\end{document}
%%%%%%%%%%%%%%%%%%%%%%%%%%%%%%%%%%%%%%%%%%%%%%%%%%%%%
