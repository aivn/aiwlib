% -*- mode: LaTeX; coding: utf-8 -*-
\documentclass[12pt]{article}
\usepackage[unicode,colorlinks]{hyperref}
\usepackage[T2A]{fontenc}
\usepackage[utf8]{inputenc}
\usepackage[russian]{babel}
\usepackage{amsmath}
\usepackage{amssymb}
\usepackage{eufrak}
\usepackage{epsfig}
%\usepackage[mathscr]{eucal}
\usepackage{psfrag}
\usepackage{tabularx}
\usepackage{wrapfig}
%\usepackage{eucal}
\usepackage{euscript}

\usepackage[usenames]{color}
\usepackage{colortbl} 

\definecolor{codegreen}{rgb}{0,0.6,0}
\definecolor{codegray}{rgb}{0.5,0.5,0.5}
\definecolor{codeblack}{rgb}{0.1,0.,0.3}
\definecolor{codeemph}{rgb}{0.5,0.1,0.5}
\definecolor{codepurple}{rgb}{0.58,0,0.82}
\definecolor{backcolour}{rgb}{0.95,0.95,0.92}

\usepackage{listings}\lstset{
	basicstyle=\ttfamily\fontsize{10pt}{10pt}\selectfont\color{codeblack},
  commentstyle=\color{codegray},
	keywordstyle=\tt\bf\color{codeemph},
	belowskip=0pt
    }

\setlength{\topmargin}{-0.5in}
\setlength{\oddsidemargin}{-5.mm}
\setlength{\evensidemargin}{-5.mm}
\setlength{\textwidth}{7.in}
\setlength{\textheight}{9.in}

\def\dfdx#1#2{\frac{\partial #1}{\partial #2}}
\def\hm#1{#1\nobreak\discretionary{}{\hbox{\m@th$#1$}}{}}
\newcommand{\Frac}[2]{\displaystyle\frac{#1}{#2}}

\def\sr#1{{\left<#1\right>}}
\def\m{\mathbf m{}}

\begin{document}
\begin{center}
  \Large\bf БИНАРНЫЕ ФОРМАТЫ AIWLIB
\end{center}
\tableofcontents

\section{Введение}
Все бинарные форматы \verb'aiwlib' построены по общему принципу. В один файл может быть последовательно записано несколько независимых фреймов,
содержащих данные в т.ч. для разнотипных контейнеров. Исключение составляет только формат для магнетиков, в котором первым фреймом
записываются координаты магнитных моментов, а затем следуют фреймы с ориентациями магнитных моментов.

За исключением магнетиков, форматы ориентированы на хранения данных вида <<array of structure>>. Тип ячейки задается пользователем но не хранится\footnote{Механизм описания структуры ячейки разработан и реализован, но в настоящий момент не используется, в частности потому что нет вьюверов которые могли бы его поддерживать},
сохраняется только размер ячейки в байтах.

Каждый фрейм предваряется заголовком.
В свою очередьЮ заголовок состояит из текстового заголовка вида длина заголовка (4 байта) и текста.
Затем следует четырехбайтовове служебное поле определяющее тип фрейма,
затем идет несколько служебных полей с размером ячейки и размерами сетки. После заголовка следуют сами бинарные данные (ячейки сетки). 

Некоторые форматы являются расширяемыми, дополнительные данные записываются в конец текстового заголовка и являются необязательными. Вьювер \verb'im3D' корректно читает
такой формат игнорируя дополнительные данные.

В большинстве случаев общая длина текстового заголовка и служебных полей в начале фрейма выбираются так, что бы данные фрейма были выравнены на 64 байта.

%%%%%%%%%%%%%%%%%%%%%%%%%%%%%%%%%%%%%%%%%%%%%%%%%%%%%%%%%%%%%%%%%%%%%%%%%%%%%%%%
\section{Традиционные однородные регулярные сетки}
Самая распространенная струкутра данных численного моделирования.
В \verb'aiwlib' реализованы в виде класса \verb'Mesh<typename T, int D>'. 

%\begin{table}
\begin{center}
\begin{tabular}{|p{.1\textwidth}|p{.25\textwidth}|p{.21\textwidth}|p{.35\textwidth}|}
\hline
величина & длина, байт & тип & описание величины \\
\hline
\multicolumn{4}{|c|}{заголовок \rule{0pt}{.6cm}}\\
\hline
{\tt h\_sz} & {\tt 4} & {\tt uin32\_t} & длина текстового заголовка \\
{\tt h} & {\tt h\_sz*4} & {\tt char*} & текстовый заголовок \\
{\tt D} & {\tt 4} & {\tt uint32\_t} & размерность сетки\\
{\tt szT} & {\tt 4} & {\tt uint32\_t} & размер ячейки сетки в байтах\\
{\tt box} & {\tt D*4} & {\tt uint32\_t[D]} & размеры сетки в ячейках\\
\hline
\multicolumn{4}{|c|}{данныe \rule{0pt}{.6cm}}\\
\hline
{\tt data} & {\tt szT*box[0]*box[1]*...} & пользовательский & ось $x$ самая быстрая \\
\hline
\end{tabular}
\end{center}
% \end{table}

Опционально, в {\bf текстовый} заголовок могут быть записаны следующие данные (размещаются после первого нулевого байта {\tt h})
\begin{center}
\begin{tabular}{|p{.1\textwidth}|p{.12\textwidth}|p{.21\textwidth}|p{.48\textwidth}|}
\hline
величина & длина, байт & тип & описание величины \\
\hline
  {\tt axis} & --- & {\tt char*[D]} & имена осей сетки, записываются последовательно, каждая ось состоит из длины (четырехбайтовое целое) и собственно имени\\
  {\tt info} & --- & {\tt char*} &  пользователькая метаинформация, строка которая может содержать нулевые символы, сотсоит из длины (четырехбайтового целого) и массива символов\\
  {\tt typeinfo} & --- & {\tt aiw::TypeInfo} & описание структуры ячейки сетки, в настоящий момент не поддерживается \\
  {\tt out\_value} & {\tt szT} & пользовательский & значение на бесконечности (за пределами сетки) \\
  {\tt align} & --- & --- & некоторое количество нулей, необходимое для выравнивания данных сетки на 64 байта \\
  {\tt bmin} & {\tt D*8} & {\tt double[D]} & координаты левого нижнего угла сетки \\
  {\tt bmax} & {\tt D*8} & {\tt double[D]} & координаты правого верхнего угла сетки \\
  {\tt mask} & {\tt 4} & {\tt uint32\_t} & битовая маска \\  
  \hline
\end{tabular}
\end{center}
Битовая маска содержит:
\begin{itemize}
  \item 31-й бит --- флаг наличия имен осей;
  \item 30-й бит --- флаг наличия структуры {\tt TypeInfo};
  \item 29-й бит --- флаг наличия метаинформации {\tt info};
  \item младшие биты --- флаги логарифмического масштаба по соответствующим осям.
\end{itemize}


%%%%%%%%%%%%%%%%%%%%%%%%%%%%%%%%%%%%%%%%%%%%%%%%%%%%%%%%%%%%%%%%%%%%%%%%%%%%%%%%
\section{Однородные регулярные сетки на основе Z-кривой Мортона}
%%%%%%%%%%%%%%%%%%%%%%%%%%%%%%%%%%%%%%%%%%%%%%%%%%%%%%%%%%%%%%%%%%%%%%%%%%%%%%%%
\section{Сферическая сетка на основе рекурсивного разбиения пентакисдодекаэдра}
%%%%%%%%%%%%%%%%%%%%%%%%%%%%%%%%%%%%%%%%%%%%%%%%%%%%%%%%%%%%%%%%%%%%%%%%%%%%%%%%
\section{Неструктурированная двумерная сетка --- поверхность аппроксимированная треугольниками}
%%%%%%%%%%%%%%%%%%%%%%%%%%%%%%%%%%%%%%%%%%%%%%%%%%%%%%%%%%%%%%%%%%%%%%%%%%%%%%%%
\section{ZAMR}
%%%%%%%%%%%%%%%%%%%%%%%%%%%%%%%%%%%%%%%%%%%%%%%%%%%%%%%%%%%%%%%%%%%%%%%%%%%%%%%%
\section{Магнетики}
\begin{center}
\begin{tabular}{|p{.08\textwidth}|p{.12\textwidth}|p{.21\textwidth}|p{.5\textwidth}|}
\hline
величина & длина, байт & тип & описание величины \\
\hline
\multicolumn{4}{|c|}{заголовок \rule{0pt}{.6cm}}\\
\hline
{\tt F} & {\tt 4} & {\tt int} & флаг формата, равен нулю \\
{\tt N} & {\tt 4} & {\tt int} & число магнитных моментов \\
{\tt r} & {\tt 12*N} & {\tt vctr<3, float>*} & координаты магнитных моментов в конфигурационном прострастве\\
\hline
\multicolumn{4}{|c|}{кадр с данными \rule{0pt}{.6cm}}\\
\hline
{\tt time} & {\tt 8} & {\tt double} & время кадра \\
{\tt data} & {\tt 12*N} & {\tt vctr<3, float>} & массив значений магнитных моментов \\
\hline
\end{tabular}
\end{center}

%%%%%%%%%%%%%%%%%%%%%%%%%%%%%%%%%%%%%%%%%%%%%%%%%%%%%%%%%%%%%%%%%%%%%%%%%%%%%%%% 
\section{Ансамбль сферических частиц}

\end{document}
