\begin{verbatim}
os : module os
timemodule : module time
Date : объект даты
Time : объект времени
inf : float("inf")
nan : float("nan")
reg : класс region
K : 2**10
M : 2**20
G : 2**30
T : 2**40
@size : int, размер расчета в байтах
@hsize : размер расчета в "человеческом" представлении
@lendir : число файлов и папок в директории расчета
time2string : time as float --> time as string "hh:mm:ss.sss"
string2time : time as string "hh:mm:ss.sss" --> time as float
date2string : date as float --> date as string "YYYY.MM.DD-hh:mm:ss.sss"
string2date : date as string "YYYY.MM.DD-hh:mm:ss.sss" --> date as float
size2string : size as string for human
@calc : уникальное имя расчета в репозитории
@user : последний пользователь изменивший состояние 
расчета
@host : последний хост изменивший состояние расчета
@mdate : последняя дата изменения состояние расчета
@cdate : дата создания расчета
@sdate : дата запуска расчета
@adate : дата активации расчета
@weekday : номер дня недели
@today : сегодняшний день (интервал времени )
@yesterday : вчерашний день (интервал времени)
@week : текущая неделя (интервал времени)
@lastweek : прошлая неделя (интервал времени)
@logsize : размер логфайла в байтах
sources : sources(path,pattern="*") --- создает список исходных файлов
			отвечаюших шаблону pattern для расчета c 
путем path
@status : текущий статус расчета 
"waited"|"activated"|"started"|"stopped"|"finished"
@PID : PID расчета (int если запущен, иначе None)
@waited : True для расчета ожидающего запуска
@activated : True для расчета активированного планировщиком
@started : True для запущенного расчета
@stopped : True для остановленного расчета (обычно 
вследствии ошибки)
@finished : True для успешно завершенного расчета
@suspended : True для приостановленного (пользователем) 
расчета
@mutates : строка с последовательностью изменений 
статуса расчета
@rewaite : число сбросов из состояния активации в 
состояние ожидания
@state : состояние расчета --- время счета, прогресс, 
предполагаемое общее время счета
@lefttime : предполагаемое время до завершения расчета
@finishdate : предполагаемая дата завершения расчета
fnmatch : fnmatch( string, pattern ) --- функция проверки соответствия 
шаблону
@logfile : выдает содержимое logfile
priority : приоритет запуска
pow : pow(x, y)  Return x**y (x to the power of y).
fsum : fsum(iterable)  Return an accurate floating point sum of values in the iterable. Assumes 
IEEE-754 floating point arithmetic.
cosh : cosh(x)  Return the hyperbolic cosine of x.
ldexp : ldexp(x, i)  Return x * (2**i).
hypot : hypot(x, y)  Return the Euclidean distance, sqrt(x*x + y*y).
acosh : acosh(x)  Return the hyperbolic arc cosine (measured in radians) of x.
tan : tan(x)  Return the tangent of x (measured in radians).
asin : asin(x)  Return the arc sine (measured in radians) of x.
isnan : isnan(x) -> bool  Check if float x is not a number (NaN).
log : log(x[, base])  Return the logarithm of x to the given base. If the base not specified, 
returns the natural logarithm (base e) of x.
fabs : fabs(x)  Return the absolute value of the float x.
floor : floor(x)  Return the floor of x as a float. This is the largest integral value <= x.
atanh : atanh(x)  Return the hyperbolic arc tangent (measured in radians) of x.
modf : modf(x)  Return the fractional and integer parts of x.  Both results carry the sign of x 
and are floats.
sqrt : sqrt(x)  Return the square root of x.
frexp : frexp(x)  Return the mantissa and exponent of x, as pair (m, e). m is a float and e is an 
int, such that x = m * 2.**e. If x is 0, m and e are both 0.  Else 0.5 <= abs(m) < 1.0.
degrees : degrees(x)  Convert angle x from radians to degrees.
pi : число pi
log10 : log10(x)  Return the base 10 logarithm of x.
asinh : asinh(x)  Return the hyperbolic arc sine (measured in radians) of x.
exp : exp(x)  Return e raised to the power of x.
atan : atan(x)  Return the arc tangent (measured in radians) of x.
factorial : factorial(x) -> Integral  Find x!. Raise a ValueError if x is negative or 
non-integral.
copysign : copysign(x, y)  Return x with the sign of y.
ceil : ceil(x)  Return the ceiling of x as a float. This is the smallest integral value >= x.
isinf : isinf(x) -> bool  Check if float x is infinite (positive or negative).
sinh : sinh(x)  Return the hyperbolic sine of x.
trunc : trunc(x:Real) -> Integral  Truncates x to the nearest Integral toward 0. Uses the 
__trunc__ magic method.
cos : cos(x)  Return the cosine of x (measured in radians).
e : float(x) -> floating point number  Convert a string or number to a floating point number, if 
possible.
tanh : tanh(x)  Return the hyperbolic tangent of x.
radians : radians(x)  Convert angle x from degrees to radians.
sin : sin(x)  Return the sine of x (measured in radians).
atan2 : atan2(y, x)  Return the arc tangent (measured in radians) of y/x. Unlike atan(y/x), the 
signs of both x and y are considered.
fmod : fmod(x, y)  Return fmod(x, y), according to platform C.  x % y may differ.
acos : acos(x)  Return the arc cosine (measured in radians) of x.
log1p : log1p(x)  Return the natural logarithm of 1+x (base e). The result is computed in a way 
which is accurate for x near zero.
os : module os
timemodule : module time
Date : объект даты
Time : объект времени
inf : float("inf")
nan : float("nan")
reg : класс region
K : 2**10
M : 2**20
G : 2**30
T : 2**40
@size : int, размер расчета в байтах
@hsize : размер расчета в "человеческом" представлении
@lendir : число файлов и папок в директории расчета
time2string : time as float --> time as string "hh:mm:ss.sss"
string2time : time as string "hh:mm:ss.sss" --> time as float
date2string : date as float --> date as string "YYYY.MM.DD-hh:mm:ss.sss"
string2date : date as string "YYYY.MM.DD-hh:mm:ss.sss" --> date as float
size2string : size as string for human
@calc : уникальное имя расчета в репозитории
@user : последний пользователь изменивший состояние 
расчета
@host : последний хост изменивший состояние расчета
@mdate : последняя дата изменения состояние расчета
@cdate : дата создания расчета
@sdate : дата запуска расчета
@adate : дата активации расчета
@weekday : номер дня недели
@today : сегодняшний день (интервал времени )
@yesterday : вчерашний день (интервал времени)
@week : текущая неделя (интервал времени)
@lastweek : прошлая неделя (интервал времени)
@logsize : размер логфайла в байтах
sources : sources(path,pattern="*") --- создает список исходных файлов
			отвечаюших шаблону pattern для расчета c 
путем path
@status : текущий статус расчета 
"waited"|"activated"|"started"|"stopped"|"finished"
@PID : PID расчета (int если запущен, иначе None)
@waited : True для расчета ожидающего запуска
@activated : True для расчета активированного планировщиком
@started : True для запущенного расчета
@stopped : True для остановленного расчета (обычно 
вследствии ошибки)
@finished : True для успешно завершенного расчета
@suspended : True для приостановленного (пользователем) 
расчета
@mutates : строка с последовательностью изменений 
статуса расчета
@rewaite : число сбросов из состояния активации в 
состояние ожидания
@state : состояние расчета --- время счета, прогресс, 
предполагаемое общее время счета
@lefttime : предполагаемое время до завершения расчета
@finishdate : предполагаемая дата завершения расчета
fnmatch : fnmatch( string, pattern ) --- функция проверки соответствия 
шаблону
@logfile : выдает содержимое logfile
priority : приоритет запуска
pow : pow(x, y)  Return x**y (x to the power of y).
fsum : fsum(iterable)  Return an accurate floating point sum of values in the iterable. Assumes 
IEEE-754 floating point arithmetic.
cosh : cosh(x)  Return the hyperbolic cosine of x.
ldexp : ldexp(x, i)  Return x * (2**i).
hypot : hypot(x, y)  Return the Euclidean distance, sqrt(x*x + y*y).
acosh : acosh(x)  Return the hyperbolic arc cosine (measured in radians) of x.
tan : tan(x)  Return the tangent of x (measured in radians).
asin : asin(x)  Return the arc sine (measured in radians) of x.
isnan : isnan(x) -> bool  Check if float x is not a number (NaN).
log : log(x[, base])  Return the logarithm of x to the given base. If the base not specified, 
returns the natural logarithm (base e) of x.
fabs : fabs(x)  Return the absolute value of the float x.
floor : floor(x)  Return the floor of x as a float. This is the largest integral value <= x.
atanh : atanh(x)  Return the hyperbolic arc tangent (measured in radians) of x.
modf : modf(x)  Return the fractional and integer parts of x.  Both results carry the sign of x 
and are floats.
sqrt : sqrt(x)  Return the square root of x.
frexp : frexp(x)  Return the mantissa and exponent of x, as pair (m, e). m is a float and e is an 
int, such that x = m * 2.**e. If x is 0, m and e are both 0.  Else 0.5 <= abs(m) < 1.0.
degrees : degrees(x)  Convert angle x from radians to degrees.
pi : число pi
log10 : log10(x)  Return the base 10 logarithm of x.
asinh : asinh(x)  Return the hyperbolic arc sine (measured in radians) of x.
exp : exp(x)  Return e raised to the power of x.
atan : atan(x)  Return the arc tangent (measured in radians) of x.
factorial : factorial(x) -> Integral  Find x!. Raise a ValueError if x is negative or 
non-integral.
copysign : copysign(x, y)  Return x with the sign of y.
ceil : ceil(x)  Return the ceiling of x as a float. This is the smallest integral value >= x.
isinf : isinf(x) -> bool  Check if float x is infinite (positive or negative).
sinh : sinh(x)  Return the hyperbolic sine of x.
trunc : trunc(x:Real) -> Integral  Truncates x to the nearest Integral toward 0. Uses the 
__trunc__ magic method.
cos : cos(x)  Return the cosine of x (measured in radians).
e : float(x) -> floating point number  Convert a string or number to a floating point number, if 
possible.
tanh : tanh(x)  Return the hyperbolic tangent of x.
radians : radians(x)  Convert angle x from degrees to radians.
sin : sin(x)  Return the sine of x (measured in radians).
atan2 : atan2(y, x)  Return the arc tangent (measured in radians) of y/x. Unlike atan(y/x), the 
signs of both x and y are considered.
fmod : fmod(x, y)  Return fmod(x, y), according to platform C.  x % y may differ.
acos : acos(x)  Return the arc cosine (measured in radians) of x.
log1p : log1p(x)  Return the natural logarithm of 1+x (base e). The result is computed in a way 
which is accurate for x near zero.
\end{verbatim}
