\section{Подключение системы {\tt RACS} к готовому приложению численного моделирования на языке {\tt C++11}}

Несмотря на гибкость связки языков \verb'C++' и \verb'Python', в некоторых случаях применение \verb'Python'
может быть существенно затруднено~--- например в связи с отсутствием пакета \verb'python-dev' на целевой машине,
или при необходимости запуска приложения из под \verb'MPI' с некоторыми системами управления на кластерах.

Специально для таких случаев библиотека \verb'aiwlib' содержит ряд модулей на \verb'C++' позволяющих
использовать систему \verb'RACS' для запуска расчетов без языка \verb'Python'. Ниже приведен пример кода для подключения системы \verb'RACS'
к приложению на \verb'C++11':
\begin{verbatim}
#include <aiwlib/racs>     // система RACS
#include <aiwlib/objconf>  // макрос CONFIGURATE
using namespace aiw;
...
struct Model{ // пример класса вычислительного ядра
  int a=1;      // параметры ядра
  double b=2.5;
  bool c=true;
  std::string s="qwe";
  CONFIGURATE(a, b, c, s); // список параметров для обработки

  double t = 0; // не обязательно все параметры перечислять в CONFIGURATE
    ...
};

int main(int argc, const char **argv){
  ...
  RacsCalc calc(argc, argv); // экземпляр класса расчета
  ...
  Model model; 
  // подключаем модель к RACS, при этом все параметры
  // перечисленные в CONFIGURATE становятся доступны для обработки
  calc.control(model);  
  ...
  double t_max = 25; // отдельные параметры
  int Nsteps = 10;
  calc("t_max", t_max)("Ns", Nsteps); // подключаем их к RACS
  ...
  while(model.t<t_max){
    ...
    calc.set_progress(model.t/t_max); // отмечаем степень выполнения расчета
  }
  ...
}
\end{verbatim}

Для получения пути к уникальной директории расчета используется метод \verb'calc.path()',
директория создается автоматически при первом вызове метода. Репозиторий формируется на основе поля объекта \verb'calc'
\begin{verbatim}
  std::string repo = "repo";
\end{verbatim}
форматируемой по параметрам {\bf задаваемым в аргументах командной строки}\footnote{В будущем будут использоваться все параметры поставленные на контроль}.


В отличии от варианта на языке \verb'Python', у  объекта \verb'calc'
есть только служебные поля и атрибуты, то есть невозможно написать например
\begin{verbatim}
  calc.my_parametr = ...;
  ...
  ... = calc.my_parametr + ...;
\end{verbatim}
Все параметры расчета создаются в виде обычных переменных и ставятся в \verb'RACS'
на контроль при помощи перегруженного оператора 
\begin{verbatim}
  template <typename T>
  RacsCalc& RacsCalc::operator ()(const char *key, T& val);
\end{verbatim}
либо при помощи метода
\begin{verbatim}
  template <typename T> 
  void RacsCalc::control(T &data, const char *prefix="");
\end{verbatim}

Оператор \verb'()' запоминает {\bf адрес} параметра, и возвращает объект \verb'calc',
что позволяет использовать цепочки операторов \verb'()'. Метод \verb'control'
так же запоминает адрес объекта, у которого должен быть задан метод \verb'configurate' (см. раздел \label{ibjconf:sec}).

При постановке на контроль значение параметра может быть изменено на основе аргументов командной строки или в случае серийного запуска.
При вызове метода  \verb'calc.commit()' значения  всех контролируемых параметров сохраняются в файл \verb'.RACS'.

Метод \verb'calc.commit()' может быть вызван явно, кроме того он вызывается в методе \verb'calc.set_progress(...)' и при завершении расчета (в деструкторе объекта \verb'calc').
При успешном завершении расчета  автоматически устанавливается \verb'calc.progress=1', фиксируется время выполнения \verb'runtime',
состояние расчета изменяется на \verb'finished'. В случае возникновения исключения
фиксируется время выполнения \verb'runtime',
состояние расчета изменяется на \verb'stopped'. %, при этом в \verb'calc.statelist' дописывается стек исключения и сообщение об ошибке.

Однако, если в течении расчета не было ни одного вызова метода \verb'calc.path()',
уникальная директория расчета не создается и эти действия {\bf не} выполняются~--- если расчет не сохранил никакие данные в
свою уникальную директорию то она ему не нужна, и видимо не требуется метаинформация (файл \verb'.RACS'). 

