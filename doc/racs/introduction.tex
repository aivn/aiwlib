\section{Введение}

Интерфейс приложения численного моделирования должен позволять
легко изменять параметры задачи (число которых иногда доходит до сотен или даже тысяч),
выбирать тот или иной алгоритм (в том числе разлиные варианты начальных и граничных условий),
обеспечивать анализ и визуализацию
результатов. Практика показала, что для сложных задач оптимальным
являеться не оконный интерфейс, а интерфейс командной
строки. Фактически речь идет о использовании собственного (или уже
существующего) высокоуровневого интерпретируемого языка,
адаптированного к специфике задачи.

При проведении массовых расчетов (например при анализе зависимости поведения устройсвта от 
нескольких параметров и построении фазовых диаграмм) требуется механизм, обеспечивающий
многократный автоматический запуск приложения с меняющимися заданным образом параметрами, 
желательно с контролем распределения ресурсов в рамках локальной сети или на кластере.

Для каждого расчета полученные зависимости должны сопровождаться
информацией о использованных параметрах расчета и алгоритмах. Если для
сохранения параметров существует большое количество методик и
библиотек, то сохранение алгоритмов является проблемой, и единственным
приемлемым решением на сегодняшний день является сохранение исходного
кода приложения.

Для анализа результатов необходим многопараметрический поиск по
проведенным расчетам, для чего результаты расчетов должны храниться
специальным, упорядоченным образом. Необходимо обеспечить возможность
поиска по версиям исходного кода. Эту проблему можно решать в ручную,
например размещая результаты расчетов на хорошо структурированном дереве
каталогов~--- однако такой подход требует строгой внутренней культуры пользователя, и
усложняется тем, что в процессе расчетов критерии упорядоченности могут
расширятся и изменяться кардинальным образом.   

При массовых расчетах аккуратное решение вышеописанных проблем может отнимать значительное время и силы. 
В разных рабочих группах
разработаны собственные библиотеки, позволяющие упростить процесс
написания окружения, но единый подход до сих пор не выработан.

Описанная в данной главе система {\tt RACS} ({\tt Results \& Algorithms Control System}~--- система контроля
результатов и алгоритмов) обеспечивает:
\begin{itemize}
\item задание параметров расчетов при запуске для приложений на языках \verb'Python' и \verb'C++';
\item автоматическое сохранение параметров и исходных кодов расчетов;
\item пакетный запуск расчетов (циклы по значениям параметров) и балансировка загрузки, как на локальных машинах так и на кластерах под \verb'MPI';
\item работа с контрольными точками для приложений \verb'C++', в том числе кластерах под \verb'MPI' ({\it в разработке});
\item развитые средства для многопараметрического поиска, анализа и обработки результатов.
\end{itemize}

При разработке \verb'RACS' делались следующие акценты:
\begin{itemize}
\item простота подключения (требуется минимальная модификация отлаженного кода);
\item лаконичный и интуитивно понятный синтаксис при запуске расчетов;
\item возможность обработки результатов средствами операционной системы и сторонними утилитами без потери целостности данных;
\item интеграция с другими утилитами~--- вывод данных в формате \verb'gnuplot' с заголовками \verb'gplt',
  чтение метаинформации о расчетах другими утилитами.
\end{itemize}


Даже для низкоквалифицированного
пользователя  {\tt RACS} автоматически обеспечивает необходимый минимум
<<культуры>> проведения расчетов (сохранение исходных кодов  и
параметров).
В результате пользователь имеет
возможность полностью сконцентрироваться на работе непосредственно  над задачей.

\verb'RACS' написан на языке \verb'Python' и ориентирован в первую очередь
на приложения написанные на
языках \verb'C++' (высокопроизводительное вычислительное ядро) и \verb'Python' (верхний управляющий слой приложения и
интерфейсные части), связанные при помощи утилиты \verb'SWIG'~\cite{SWIG}.

К настоящему моменту (первые версии появились в 2003 году, первая публикация \cite{racs:2007} в 2007 году) 
{\tt RACS} хорошо зарекомендовал себя при организации массовых расчетов в различных областях~--- сейсмике,
моделировании разработки керогеносодержащих месторождений с учетом внутрипластового горения,
моделировании магнитных систем и разработке устройств спинтроники, %физике плазмы,
газодинамике горения, изучении резонансных свойств нелинейных систем и т.д.
%Тем не менее, в процессе эксплуатации был обнаружен ряд недостатков, требующих существенной доработки системы.

\endinput

Целый ряд задач численного моделирования требует проведения больших объемов
однотипных серий расчетов~--- расчеты в серии независимы, и отличаются
лишь значением одного или нескольких параметров, и именно в этом в этом случае
{\tt RACS} оказывается наиболее эффективен. 
Кроме поиска в результатах расчетов, запущенный в клиент--серверном режиме {\tt RACS} обеспечивает автоматический
запуск расчетов на нескольких компьютерах в рамках кластера или локальной сети с разнородными версиями 
операционной системы.
Инструментальные средства {\tt Python} и {\tt RACS} позволяют реализовывать
консервацию и восстановление расчета для продолжения.


Изначально {\tt RACS} был построен по асинхронной схеме, без центрального сервера (такая архитектура
представлялась более надежной). Появившийся со временем сервер 
занимался лишь даигностикой и сбором статистики загруженности ресурсов.
Практика показала, что при интенсивных разнородных расчетах в рамках локальной сети или кластера 
такая архитектура не позволяет 
должным образом распределять ресурсы, что приводит к эпизодическим конфликтам. 

Интерфейс подключения {\tt RACS} к приложениям численного моделирования так же может быть существенно улучшен.
В настоящий момент подключение {\tt RACS} к уже готовому коду требует рутинной переработки кода, что неизбежно приводит
к ошибкам. Представляется возможным организовать подключение с минимальными изменениями отлаженного ранее кода.

С другой стороны, за время эксплуатации был накоплен большой опыт по организации массовых расчетов и 
постобработке результатов моделирования,  сформулированна соответствующая идеология. Разработанные подходы должны быть 
особенно эффективны при решении инженерных задач, требующих проведения больших объемов однотипных расчетов и комплексного 
анализа их результатов для выбора
оптимальной конфигурации устройства.
