\section{Запуск расчетов подключенных к системе {\tt RACS}}
При запуске расчета, в момент создания объекта \verb'calc', происходит разбор аргументов командной строки.
При этом возможна настройка (отложенная) параметров расчета, организация серийного запуска и т.д.

Если в аргументах командной строки присутствует \verb'-h' или \verb'--help' запуск расчета не производится, вместо этого печатается
справочная информация и приложение завершает работу. В справочную информацию, кроме  описания синтаксиса задания параметров расчета
и различных специальных опций, для \verb'Python' может включаться информация о параметрах расчета~--- фактически
печатаются все строки из исходного файла содержащие последовательность \verb'#@'.

При задании параметров расчета используется синтаксис \verb'KEY=VALUE', где \verb'KEY'~--- имя параметра, \verb'VALUE'~--- значение параметра,
которое будет сконвертировано к нужному типу автоматически. Для приложений на \verb'Python' допустим синтаксис
\verb'KEY=@EXPRESSION', где выражение \verb'EXPRESSION' вычисляется 
функцией \verb'eval' в словаре модуля \verb'math' и на основе уже заданных параметров, затем 
конвертируется к типу значения параметра по умолчанию.

При серийном запуске возможны варианты
\begin{itemize}
\item\verb'KEY=[EXPRESSION]'  --- выражение \verb'EXPRESSION' вычисляется функцией \verb'eval' (только для языка \verb'Python');
\item\verb'KEY=[X1,X2..Xn]'   --- хаскелль--стиль для арифметической прогрессии;
\item\verb'KEY=[X1:STEP..Xn]' --- арифметическая прогрессия с шагом \verb'STEP';
\item\verb'KEY=[X1@STEP..Xn]' --- геометрическая прогрессия с множителем \verb'STEP';
\item\verb'KEY=[X1#SIZE..Xn]' --- арифметическая прогрессия из \verb'SIZE' элементов;
\item\verb'KEY=[X1^SIZE..Xn]' --- геометрическая прогрессия из \verb'SIZE' элементов;
\end{itemize}
В языке \verb'Python' значения \verb'X1', \verb'Xn', \verb'STEP', \verb'SIZE' вычисляются функцией \verb'eval' в словаре модуля \verb'math' и уже 
заданных аргументов. В языке \verb'C++' значения \verb'X1', \verb'Xn', \verb'STEP', \verb'SIZE'
приводятся к типу \verb'double' функцией \verb'::atof'.
Параметры \verb'X1' и \verb'Xn' {\bf всегда} включаются в серию. При явном задании 
параметра \verb'STEP' шаг всегда корректируется для точного попадания в \verb'Xn'. Параметр \verb'SIZE' 
должен быть целочисленным  (не менее двух). Если серии заданы для нескольких 
параметров, вычисляются все возможные комбинации значений (декартово произведение).

При серийном запуске на одном компьютере процесс дублируется при помощи вызова \verb'fork' нужное число раз, родительский процесс
передает каждому потомку необходимую комбинацию параметров расчета,
ожидает окончания работы всех потомков и завершается не выходя из конструктора объекта \verb'calc'. Максимальное число копий
процесса \verb'N' задается опцией \verb'-c=N' или \verb'--copies=N'.

При серийном запуске из под \verb'MPI' число копий задается
системной утилитой из под которой был запущен расчет, параметр \verb'copies' игнорируется. При запуске из под \verb'MPI'
процесс с нулевым идентификатором выполняет роль мастер--процесса~--- создает уникальные директории расчетов, раздает задания вычислителям, но
сам никаких вычислений не производит. На каждом из вычислителей в свою очередь процесс дублируется при помощи системного вызова \verb'fork',
дочерний процесс проводит расчет а родительский процесс дождавшись его завершения обращается к мастер--процессу за следующим заданием.

Аргумент командной строки вида \verb'TAG+' --- добавляет тэг \verb'TAG' к расчету. 

Для булевых параметров можно указывать значения
\verb'Y', \verb'y', \verb'YES', \verb'Yes', \verb'yes', \verb'ON',  \verb'On',  \verb'on',  \verb'TRUE',  \verb'True',  \verb'true',  \verb'V', \verb'v', \verb'1', 
\verb'N', \verb'n', \verb'NO',  \verb'No',  \verb'no',  \verb'OFF', \verb'Off', \verb'off', \verb'FALSE', \verb'False', \verb'false', \verb'X', \verb'x', \verb'0'.

Кроме того доступны следующие служебные параметры
\begin{itemize}
\item\verb'-r' или \verb'--repo' --- задает путь к репозиторию для создания расчета (по умолчанию значение \verb'repo');
\item\verb'path' --- явно задает путь к директории расчета, в языке \verb'Python' если расчет существовал 
  словарь расчета будет обновлен из директории расчета;
\item\verb'-p' или \verb'--clean-path' --- очищать явно заданную директорию расчета, при этом
                          словарь расчета не обновляется из директории (по умолчанию включено);
\item\verb'-s' или \verb'--symlink' --- создавать символическую ссылку \verb|'_'| на последнюю 
  директорию расчета (по умолчанию включено);
\item\verb'-d' или \verb'--daemonize' --- <<демонизировать>> расчет при запуске (освободить терминал, 
  вывод будет перенаправлен в \verb'logfile' в директории расчета),
  <<демонизация>> происходит при создание объекта \verb'calc' (по умолчанию выключено);
\item\verb'-S' или \verb'--statechecker' --- запускать демона, при необходимости фиксирующего в файле \verb'.RACS' аварийное
                            завершение расчета (по умолчанию включено);
\item\verb'-c' или \verb'--copies' --- число копий процесса при проведении расчетов с серийными 
  параметрами (по умолчанию 1);
\item\verb'-e' или \verb'--on-exit' --- по завершении расчета автоматически фиксировать время работы,
  состояние расчета и сохранять изменения параметров на диск (работает только если в процессе расчета была создана его уникальная директория, по умолчанию включено);
\item\verb'-n' или \verb'--calc-num' --- число знаков в номере расчета (в текущем дне) при 
  автоматической генерации имени уникальной директории расчета (по умолчанию 3);
\item\verb'-a' или \verb'--auto-pull' --- автоматически сохранять все параметры расчета из 
                         контролируемых расчетом объектов (по умолчанию включено, работает только в языке \verb'Python');
\item\verb'-m' или \verb'--commit-sources' --- сохранять исходные коды расчета (по умолчанию включено, работает только в языке \verb'Python').
\item\verb'-mpi' --- работа из под \verb'MPI'  (по умолчанию отключено, работает только в языке \verb'C++').
\end{itemize}

Булевы служебные параметры достаточно просто упомянуть что бы они приняли значение \verb'true', например \verb'-d' и \verb'-d=Y' эквивалентные записи.

Для всех параметров (кроме серийных) возможно дублирование, актуальным является 
последнее значение. Значения параметров по умолчанию могут быть изменены при вызове 
конструктора (создании объекта  \verb'calc', за исключением параметров \verb'daemonize' и \verb'copies'), но параметры 
командной строки их перекрывают. Длинные имена параметров 
могут задаваться как с одним, так и с двумя минусами, короткие имена
только с одним. Параметр \verb'path' задается без минусов.

